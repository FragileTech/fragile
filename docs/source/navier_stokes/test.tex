% Options for packages loaded elsewhere
\PassOptionsToPackage{unicode}{hyperref}
\PassOptionsToPackage{hyphens}{url}
%
\documentclass[
  11pt,
  11pt,
  letterpaper,
  twoside]{article}
\usepackage[]{mathpazo}
\usepackage{amssymb,amsmath}
\usepackage{ifxetex,ifluatex}
\ifnum 0\ifxetex 1\fi\ifluatex 1\fi=0 % if pdftex
  \usepackage[T1]{fontenc}
  \usepackage[utf8]{inputenc}
  \usepackage{textcomp} % provide euro and other symbols
\else % if luatex or xetex
  \usepackage{unicode-math}
  \defaultfontfeatures{Scale=MatchLowercase}
  \defaultfontfeatures[\rmfamily]{Ligatures=TeX,Scale=1}
\fi
% Use upquote if available, for straight quotes in verbatim environments
\IfFileExists{upquote.sty}{\usepackage{upquote}}{}
\IfFileExists{microtype.sty}{% use microtype if available
  \usepackage[]{microtype}
  \UseMicrotypeSet[protrusion]{basicmath} % disable protrusion for tt fonts
}{}
\makeatletter
\@ifundefined{KOMAClassName}{% if non-KOMA class
  \IfFileExists{parskip.sty}{%
    \usepackage{parskip}
  }{% else
    \setlength{\parindent}{0pt}
    \setlength{\parskip}{6pt plus 2pt minus 1pt}}
}{% if KOMA class
  \KOMAoptions{parskip=half}}
\makeatother
\usepackage{xcolor}
\IfFileExists{xurl.sty}{\usepackage{xurl}}{} % add URL line breaks if available
\IfFileExists{bookmark.sty}{\usepackage{bookmark}}{\usepackage{hyperref}}
\hypersetup{
  pdftitle={Global Regularity for the 3D Navier-Stokes Equations via Variational Stratification and Gevrey Structural Stability},
  pdfauthor={Guillem Duran-Ballester},
  hidelinks,
  pdfcreator={LaTeX via pandoc}}
\urlstyle{same} % disable monospaced font for URLs
\usepackage[margin=1in,letterpaper]{geometry}
\usepackage{longtable,booktabs}
% Correct order of tables after \paragraph or \subparagraph
\usepackage{etoolbox}
\makeatletter
\patchcmd\longtable{\par}{\if@noskipsec\mbox{}\fi\par}{}{}
\makeatother
% Allow footnotes in longtable head/foot
\IfFileExists{footnotehyper.sty}{\usepackage{footnotehyper}}{\usepackage{footnote}}
\makesavenoteenv{longtable}
\setlength{\emergencystretch}{3em} % prevent overfull lines
\providecommand{\tightlist}{%
  \setlength{\itemsep}{0pt}\setlength{\parskip}{0pt}}
\setcounter{secnumdepth}{5}
\usepackage{amsmath}
\usepackage{amssymb}
\usepackage{amsthm}
\usepackage{mathtools}
\usepackage{bbm}
\usepackage{bm}
\usepackage{mathrsfs}
\usepackage{thmtools}
\newtheorem{theorem}{Theorem}{[}section{]}
\newtheorem{lemma}{[}theorem{]}\{Lemma\}
\newtheorem{proposition}{[}theorem{]}\{Proposition\}
\newtheorem{corollary}{[}theorem{]}\{Corollary\}
\theoremstyle{definition}
\newtheorem{definition}{[}theorem{]}\{Definition\}
\newtheorem{assumption}{[}theorem{]}\{Assumption\}
\newtheorem{hypothesis}{[}theorem{]}\{Hypothesis\}
\theoremstyle{remark}
\newtheorem{remark}{[}theorem{]}\{Remark\}
\newcommand{\RR}{\mathbb{R}}
\newcommand{\NN}{\mathbb{N}}
\DeclareMathOperator{\supp}{supp}
\DeclareMathOperator{\dist}{dist}
\setlength{\parskip}{0.5em}
\setlength{\parindent}{0pt}

\title{Global Regularity for the 3D Navier-Stokes Equations via
Variational Stratification and Gevrey Structural Stability}
\author{Guillem Duran-Ballester}
\date{\today}

\begin{document}
\maketitle
\begin{abstract}
We prove global regularity for the three-dimensional incompressible
Navier-Stokes equations on \(\mathbb{R}^3\). The proof is established
via a \textbf{complete stratification of the singular phase space},
demonstrating that every topological class of potential blow-up profiles
encounters a fatal obstruction derived from the viscous structure.

We introduce a \textbf{nonlinear efficiency functional}
\(\Xi[\mathbf{u}]\) to quantify the competition between vortex
stretching and viscous smoothing. This yields a fundamental dichotomy:
any blow-up candidate is either variationally inefficient
(fractal/high-entropy) or variationally efficient (coherent/smooth). We
systematically exclude both branches:

\begin{enumerate}
\def\labelenumi{\arabic{enumi}.}
\item
  \textbf{Fractal Exclusion via Gevrey Recovery:} We prove that
  high-entropy states possess a quantitative efficiency deficit. This
  deficit forces a strictly positive growth of the Gevrey radius of
  analyticity (\(\dot{\tau} > 0\)), dynamically arresting singularity
  formation in the rough regime.
\item
  \textbf{Coherent Exclusion via Geometric Rigidity:} Within the
  efficient (smooth) stratum, we classify profiles by swirl and scaling.

  \begin{itemize}
  \item
    \textbf{High-Swirl} profiles are excluded by the strict accretivity
    of the linearized operator (Spectral Coercivity).
  \item
    \textbf{Type II (Fast Focusing)} profiles are excluded by a
    Mass-Flux Capacity bound, which renders supercritical acceleration
    energetically impossible for fixed viscosity \(\nu > 0\).
  \item
    \textbf{Low-Swirl/High-Twist (``Barber Pole'')} profiles are
    excluded by the regularity of variational extremizers, which
    precludes unbounded internal twist.
  \item
    \textbf{Tube-like} profiles are excluded by Axial Pressure
    Defocusing.
  \end{itemize}
\end{enumerate}

Since the failure sets of these mechanisms form an open cover of the
phase space, the set of admissible singular limits is empty. This result
relies critically on the parabolic nature of the equations; we
demonstrate why the exclusion mechanisms fail for the inviscid Euler
equations.
\end{abstract}

{
\setcounter{tocdepth}{2}
\tableofcontents
}
\hypertarget{global-regularity-for-the-3d-navier-stokes-equations-via-variational-stratification-and-gevrey-structural-stability}{%
\section{Global Regularity for the 3D Navier-Stokes Equations via
Variational Stratification and Gevrey Structural
Stability}\label{global-regularity-for-the-3d-navier-stokes-equations-via-variational-stratification-and-gevrey-structural-stability}}

(sec-introduction)= \#\# 1. Introduction

The global regularity of the three-dimensional Navier-Stokes equations
for incompressible fluids remains one of the most significant open
problems in mathematical analysis. The central difficulty lies in the
supercritical scaling of the energy dissipation relative to the vortex
stretching term.

Classical energy methods, such as the Beale-Kato-Majda (BKM) criterion
{[}@beale1984{]}, established that blow-up is controlled by the
accumulation of vorticity magnitude
\(\|\boldsymbol{\omega}\|_{L^\infty}\). However, these estimates are
agnostic to the \textbf{geometry} of the vortex lines. Recent numerical
studies and partial regularity results {[}@constantin1993;
@moffatt1992{]} suggest that the geometric arrangement of the vorticity
vector field \(\boldsymbol{\omega}(x,t)\) plays a decisive role in the
depletion of nonlinearity. Modern milestones underscore this landscape:
Tao's averaged Navier-Stokes blow-up construction {[}@tao2016{]} shows
the structural proximity of finite-time singularities; the Luo--Hou
axisymmetric Euler scenario {[}@luo2014{]} demonstrates a plausible
blow-up mechanism in a closely related inviscid setting; and the
endpoint \(L^3\) regularity criterion of Escauriaza, Seregin, and Šverák
{[}@escauriaza2003{]} provides the sharp conditional bound within the
classical Lebesgue scale.

In this paper, we depart from standard Sobolev estimates and analyze the
geometric structure of the vorticity field through a \textbf{variational
framework} that resolves the regularity problem. The argument is
specific to Navier--Stokes: viscosity enforces
Caffarelli--Kohn--Nirenberg dimension reduction
(\(\dim_{\mathcal{P}} \Sigma \le 1\)), fixed \(\nu>0\) drives the
dissipation capacity bound of Theorem 9.3 (mass-flux exclusion of Type
II/III), and elliptic regularity for the dissipative Euler--Lagrange
system yields smooth extremizers (Section 8.5). In Euler, anomalous
dissipation can support defect measures, the \(\nu\to 0\) limit removes
the mass-flux barrier, and elliptic bootstrapping of extremizers is
unavailable; see the summary below.

\textbf{Main Result:} We prove global regularity by establishing a
\textbf{structural dichotomy}: any potential singular profile is either
variationally efficient (smooth and coherent) or variationally
inefficient (fractal). The former class is eliminated by geometric and
spectral rigidity, while the latter is eliminated by Gevrey
regularization driven by an efficiency deficit. No auxiliary hypotheses
are assumed; failure of the efficiency or coherence conditions triggers
a complementary regularization mechanism.

From the viewpoint of partial regularity, the
Caffarelli--Kohn--Nirenberg theory and its refinements (by Lin, Seregin,
Naber--Valtorta and others) already provide a strong
\textbf{dimension-reduction} framework: the parabolic Hausdorff
dimension of the singular set is at most one. This shows that any
putative singularity must concentrate along objects of codimension at
least two---isolated points or filament-like sets. Our variational
analysis further restricts these to smooth, coherent structures.

Within the coherent stratum, we systematically eliminate all paths to
singularity through a combination of spectral, topological, and
variational obstructions:

\begin{enumerate}
\def\labelenumi{\arabic{enumi}.}
\tightlist
\item
  \textbf{High-Swirl Configurations:} For swirl parameter
  \(\sigma > \sigma_c = \sqrt{2}\), the linearized operator is strictly
  accretive with spectral gap \(\mu > 0\), emerging from differential
  scaling of vortex stretching versus centrifugal pressure.
\item
  \textbf{Type II (Fast Focus):} Excluded by mass-flux capacity bounds
  derived rigorously from elliptic regularity of limit profiles.
\item
  \textbf{Type I with Axial Defocusing:} A collapsing tube requires
  \(\mathcal{D}(t) \le 0\); otherwise pressure gradients dominate
  inertial stretching.
\item
  \textbf{Low-Swirl, High-Twist Filaments:} Coherent low-swirl filaments
  with unbounded internal twist are incompatible with the smoothness
  requirements of variational extremizers. The uniform gradient bounds
  from elliptic regularity directly contradict the unbounded gradients
  required for infinite twist; we refer to these high-twist filaments
  descriptively as the ``Barber Pole'\,' configuration.
\item
  \textbf{Redundancy of Obstructions:} The arguments are arranged so
  that multiple mechanisms overlap. Even if a spectral gap degenerates,
  compactness and elliptic regularity still preclude singular profiles
  with uncontrolled twisting or drifting in a degenerate energy
  landscape.
\end{enumerate}

:::\{prf:theorem\} Structural Dichotomy for Navier-Stokes :label:
thm-structural-dichotomy

Any renormalized blow-up candidate belongs to one of two branches. If it
is variationally efficient, it converges (modulo symmetries) to a
smooth, coherent profile that is excluded by spectral, geometric, or
defocusing rigidity. If it is variationally inefficient, the efficiency
deficit forces strictly positive growth of the Gevrey radius, excluding
collapse. In either branch, the 3D Navier-Stokes solution with smooth
initial data remains smooth for all time. :::

The proof proceeds by demonstrating that every conceivable path to
singularity encounters an insurmountable obstruction---either spectral
(high swirl), topological (Type I/II), or variational (low-swirl
coherent filaments with unbounded internal twist). The dichotomy
formulation removes conditional hypotheses from the statement: failure
of the smooth, coherent branch automatically activates the Gevrey
regularization mechanism.

\textbf{Why Navier-Stokes, not Euler.} The strategy relies on viscosity
in three ways: (i) Caffarelli--Kohn--Nirenberg dimension reduction
(\(d\le 1\)) rules out volumetric singular measures; (ii) the
dissipation capacity bound (Theorem 9.3) forces Type II/III scaling to
expend infinite energy unless \(\nu \to 0\) (Remark 6.1.7); (iii)
parabolic smoothing of the Euler--Lagrange system gives smooth
extremizers (Section 8.5). In Euler, anomalous dissipation could
accommodate defect measures, \(\nu \lambda^{-1}\) can remain bounded
along rapid collapse, and elliptic bootstrapping is unavailable, so
these barriers do not apply.

\textbf{High-frequency Type I objection.} A ``wrinkled'' Type I scenario
with large Gevrey amplitude but fixed scaling is precluded by Remark
8.4.4: divergence of \(\|\mathbf{u}\|_{\tau,1}\) decouples the viscous
scale, accelerates \(\lambda(t)\), and moves the trajectory into the
Type II stratum \(\Omega_{\mathrm{Acc}}\), where Theorem 9.3 rules out
blow-up.

(sec-structure-of-the-argument)= \#\#\# 1.2 Structure of the Argument

The argument partitions the phase space of renormalized limit profiles
into five mutually exclusive strata and shows that each has an empty
intersection with the singular set:

\begin{enumerate}
\def\labelenumi{\arabic{enumi}.}
\item
  \textbf{Energetic partition (Type I vs.~Type II).} The
  accelerating/Type II stratum \(\Omega_{\mathrm{Acc}}\) corresponds to
  decoupling from the viscous scale (\(Re_\lambda \to \infty\)) and is
  excluded by mass-flux capacity and the divergence of the dissipation
  integral (Section 9).
\item
  \textbf{Entropic partition (Fractal vs.~Coherent).} Within the
  viscously locked (Type I) regime, profiles are either
  fractal/high-entropy (variational distance bounded below) or
  coherent/near-extremal. The fractal stratum \(\Omega_{\mathrm{Frac}}\)
  is excluded by the variational efficiency gap and Gevrey recovery
  (Section 8).
\item
  \textbf{Geometric partition (Swirl and twist).} Coherent profiles are
  further classified by swirl \(\mathcal{S}\) and twist \(\mathcal{T}\).
  High-swirl profiles \(\Omega_{\mathrm{Swirl}}\) are excluded by
  spectral coercivity (Section 6). Low-swirl profiles satisfy a
  curvature dichotomy: bounded-twist tubes \(\Omega_{\mathrm{Tube}}\)
  are excluded by axial defocusing (Section 4), while high-twist
  filaments \(\Omega_{\mathrm{Barber}}\) are excluded by the regularity
  and bounded-gradient properties of variational extremizers (Section
  11).
\end{enumerate}

This stratification is summarized in Table 1 of Section 7. Section 12
provides the formal covering argument and shows that
\(\Omega_{\mathrm{sing}}\) is empty.

\textbf{Variational Dichotomy.} Our analysis does not assume the
existence of smooth extremizers. Instead, we prove that if a maximizing
profile is not smooth, it incurs an efficiency deficit
\(\Xi < \Xi_{\max}\). This deficit forces the growth of the Gevrey
radius \(\dot{\tau} > 0\), preventing blow-up. Consequently, we need
only test geometric obstructions against the class of smooth, coherent
profiles: fractal or rough profiles regularize by inefficiency, while
coherent profiles regularize by rigidity.

(sec-mathematical-preliminaries)= \#\# 2. Mathematical Preliminaries

We consider the 3D incompressible Navier-Stokes equations in
\(\mathbb{R}^3\):
\[ \partial_t \mathbf{u} + (\mathbf{u} \cdot \nabla) \mathbf{u} = -\nabla P + \nu \Delta \mathbf{u}, \quad \nabla \cdot \mathbf{u} = 0 \]
The vorticity \(\boldsymbol{\omega} = \nabla \times \mathbf{u}\) evolves
according to:
\[ \partial_t \boldsymbol{\omega} + (\mathbf{u} \cdot \nabla) \boldsymbol{\omega} = S \boldsymbol{\omega} + \nu \Delta \boldsymbol{\omega} \]
where \(S = \frac{1}{2}(\nabla \mathbf{u} + \nabla \mathbf{u}^T)\) is
the strain rate tensor.

:::\{prf:definition\} Geometric Entropy Functional :label:
def-geometric-entropy-functional

To quantify the geometric complexity of the vortex lines, we introduce
the directional Dirichlet energy:
\[ Z(t) = \int_{\mathbb{R}^3} |\boldsymbol{\omega}|^2 |\nabla \boldsymbol{\xi}|^2 \, dx, \quad \text{where } \boldsymbol{\xi} = \frac{\boldsymbol{\omega}}{|\boldsymbol{\omega}|} \]
States with \(Z(t) \approx 0\) correspond to coherent, straight vortex
tubes. States with large \(Z(t)\) correspond to spatially complex,
highly oscillatory vorticity fields. :::

:::\{prf:definition\} High-Twist Filament / ``Barber Pole''
Configuration :label: def-high-twist-filament-barber-pole-configuration

We define a \textbf{High-Twist Filament} (for descriptive brevity, a
``Barber Pole'\,' configuration) as a sequence of coherent, low-swirl
vorticity profiles \(\mathbf{V}_n\) in the renormalized frame
characterized by: 1. \textbf{Low Swirl:} The swirl ratio satisfies
\(\mathcal{S} < \sqrt{2}\) (evading the spectral coercivity barrier of
Section 6) 2. \textbf{Coherence:} The profile is topologically trivial
(tube-like) with finite renormalized energy 3. \textbf{Unbounded
Internal Twist:} The gradient of the vorticity direction field
\(\xi = \boldsymbol{\omega}/|\boldsymbol{\omega}|\) diverges
asymptotically:
\[ \lim_{n \to \infty} \|\nabla \xi_n\|_{L^\infty(\text{supp}(\mathbf{V}_n))} = \infty \]
:::

:::\{prf:remark\} Physical interpretation of Definition 2.2 :label:
rem-physical-interpretation-of-definition-22

This regime represents a vortex filament in which the pitch of the
helical field lines tends to zero (\(k_{\text{twist}} \to \infty\))
while the tube remains approximately straight, attempting to evade the
Constantin--Fefferman alignment constraint. As we will prove, such
configurations are incompatible with the smoothness and bounded-gradient
properties required for variational extremizers.

(sec-necessary-conditions-for-singularity-formation)= \#\# 2.1.
Necessary Conditions for Singularity Formation

We express the geometric conditions as an explicit conjunction of
inequalities. A finite-time singularity at \(T^*\) can occur only if all
three constraints fail:

\begin{enumerate}
\def\labelenumi{\arabic{enumi}.}
\item
  \textbf{Defocusing Inequality (Axial Pressure vs.~Inertia).}
  \[ \mathcal{D}(t) := \int_{Core} \left(|\partial_z Q| - |W \partial_z W|\right) \, dz > 0 \quad \Longrightarrow \quad \text{no axial concentration} \]
  The flow must satisfy \(\mathcal{D}(t) \le 0\) along a sequence
  \(t \uparrow T^*\) to sustain axial influx.
\item
  \textbf{Coercivity Inequality (Swirl Threshold).} For perturbations
  \(\mathbf{w}\) of a helical profile \(\mathbf{V}\),
  \[ \mathcal{Q}(\mathbf{w}) := \underbrace{\int \frac{\mathcal{S}^2}{r^2} |\mathbf{w}|^2 \rho \, dy}_{\mathcal{I}_{cent}} - \underbrace{\int (\mathbf{w} \cdot \nabla \mathbf{V}) \cdot \mathbf{w} \, \rho \, dy}_{\mathcal{I}_{stretch}} \ge \mu \|\mathbf{w}\|_{H^1}^2 \]
  Blow-up requires the \textbf{coercivity gap} to close, i.e., the swirl
  ratio \(\mathcal{S}\) must fall below the Hardy threshold that
  guarantees \(\mathcal{Q} \ge 0\).
\item
  \textbf{Depletion Inequality (Geometric Coherence).} For the
  Navier-Stokes bilinear form \(B(u,u)\) and Stokes operator \(A\),
  \[ |\langle B(u,u), Au \rangle| \le C_{geom}(\Xi) \|u\| \|Au\|, \qquad C_{geom}(\Xi) \|u\|_{L^2} < \nu \quad \Longrightarrow \quad \text{regularity} \]
  Any singularity must satisfy \(C_{geom}(\Xi)\|u\|_{L^2} \ge \nu\)
  along a sequence approaching \(T^*\).
\end{enumerate}

:::\{prf:proposition\} Conditional Intersection of Failure Sets :label:
pro-conditional-intersection-of-failure-sets

A finite-time singularity exists at \(T^*\) only if the solution
trajectory satisfies
\[ \mathcal{D}_{crit} = \{\mathcal{D}(t) \le 0\} \cap \{\mathcal{Q}(\mathbf{w}) < \mu \|\mathbf{w}\|_{H^1}^2\} \cap \{C_{geom}(\Xi)\|u\|_{L^2} \ge \nu\} \]
We demonstrate below that, under the cited geometric rigidity
hypotheses, this intersection is empty for finite-energy helical
profiles, thereby converting the argument into a falsifiable set of
spectral and geometric inequalities.

(sec-the-nonlinear-depletion-inequality)= \#\# 3. The Nonlinear
Depletion Inequality

The competition between vortex stretching and viscosity is quantified
through precise mathematical constraints. :::

:::\{prf:definition\} Geometric Coherence Constant :label:
def-geometric-coherence-constant

For a solution \(u\) of the Navier-Stokes equations, let \(\Xi\) denote
the coherence factor (as in the Gevrey framework). The geometric
constant \(C_{geom}(\Xi)\) is defined as the smallest constant
satisfying
\[ |\langle B(u,u), Au \rangle| \le C_{geom}(\Xi) \|u\| \|Au\|, \] where
\(B\) is the bilinear form and \(A\) is the Stokes operator.

The \textbf{Depletion Inequality} states
\[ C_{geom}(\Xi) \|u\|_{L^2} < \nu \quad \Longrightarrow \quad \text{no finite-time blow-up}. \]
The question is therefore reduced to whether a would-be singular profile
can keep \(C_{geom}(\Xi)\|u\|_{L^2}\) above the viscosity threshold
while retaining finite energy.

(sec-the-dissipation-stretching-mismatch)= \#\#\# 3.1. The
Dissipation-Stretching Mismatch Let \(\delta\) be the characteristic
length scale of the vorticity variations (the ``roughness'' of the
vortex tube). * The \textbf{Vortex Stretching} term scales as:
\[ T_{stretch} \sim \|\omega\| \|\nabla u\| \sim \frac{\Gamma^2}{\delta^2} \]
(Assuming circulation \(\Gamma\) and scale \(\delta\)). * The
\textbf{Viscous Dissipation} term scales as:
\[ T_{diss} \sim \nu \|\Delta \omega\| \sim \nu \frac{\Gamma}{\delta^3} \]

For a smooth, cylindrical tube, the ``roughness'' scale \(\delta\) is
proportional to the core radius \(r(t)\). The terms are comparable
(\(1/r^2\) scaling for both if \(\Gamma \sim 1\)). However, for a
high-entropy (fractal) configuration, the support of the vorticity has a
Hausdorff dimension \(d_H > 1\). This implies that the local variation
scale \(\delta\) is asymptotically smaller than the macro-scale \(r\) of
the collapse.

Consider the vorticity field \(\omega\) supported on a set \(\Sigma_t\)
with Hausdorff dimension \(d_H > 1\). We decompose \(\omega\) into
Fourier modes:
\[ \omega(x,t) = \sum_{k} \hat{\omega}_k(t) e^{ik \cdot x} \]

For each mode \(k\), the stretching and dissipation terms in the
vorticity equation satisfy:
\[ T_{stretch}^k = (\omega \cdot \nabla)u|_k \quad \text{and} \quad T_{diss}^k = \nu \Delta \omega|_k = -\nu |k|^2 \hat{\omega}_k \]

By the Gagliardo-Nirenberg interpolation inequality, for functions on a
fractal support with dimension \(d_H\):
\[ \|\nabla f\|_{L^2} \ge C(d_H) \delta^{-(d_H-1)/2} \|f\|_{L^2} \]
where \(\delta\) is the characteristic scale of variation.

For the stretching term, using the Biot-Savart law \(u = K * \omega\)
where \(K\) is the singular integral kernel:
\[ |T_{stretch}^k| \le C |k| |\hat{\omega}_k| \cdot \sup_{j} |\hat{u}_j| \le C' |k| |\hat{\omega}_k|^2 \]

For the dissipation term:
\[ |T_{diss}^k| = \nu |k|^2 |\hat{\omega}_k| \]

Therefore, the ratio for mode \(k\) satisfies:
\[ \frac{|T_{diss}^k|}{|T_{stretch}^k|} \ge \frac{\nu |k|^2 |\hat{\omega}_k|}{C' |k| |\hat{\omega}_k|^2} = \frac{\nu |k|}{C' |\hat{\omega}_k|}. \]

For a fractal set with \(d_H > 1\), the spectral energy distribution
requires \(|\hat{\omega}_k| \sim |k|^{-(d_H+2)/2}\) to maintain finite
energy. Substituting:
\[ \frac{|T_{diss}^k|}{|T_{stretch}^k|} \ge \frac{\nu |k|}{C' |k|^{-(d_H+2)/2}} = \frac{\nu}{C'} |k|^{(d_H+4)/2}. \]

Since \(d_H > 1\), we have \((d_H+4)/2 > 5/2 > 0\). Thus as
\(|k| \to \infty\) (equivalently, \(\delta \to 0\)), the mode-wise ratio
\[
\rho_k := \frac{|T_{diss}^k|}{|T_{stretch}^k|}
\] diverges monotonically: \[
\rho_k \gtrsim |k|^{(d_H+4)/2} \xrightarrow[|k|\to\infty]{} \infty.
\]

Let \(k_{\min}(t)\) denote a characteristic wavenumber in the active
high-frequency spectrum of the cascade at time \(t\). Writing \[
T_{stretch} = \sum_{|k|\ge k_{\min}} T_{stretch}^k, \qquad
T_{diss} = \sum_{|k|\ge k_{\min}} T_{diss}^k,
\] we obtain the global ratio \[
\frac{|T_{diss}|}{|T_{stretch}|}
 = \frac{\sum_{|k|\ge k_{\min}} |T_{diss}^k|}{\sum_{|k|\ge k_{\min}} |T_{stretch}^k|}
 = \frac{\sum_{|k|\ge k_{\min}} \rho_k\, |T_{stretch}^k|}{\sum_{|k|\ge k_{\min}} |T_{stretch}^k|}
 \ge \inf_{|k|\ge k_{\min}} \rho_k.
\] Since \(\rho_k\) is increasing in \(|k|\) for \(d_H>1\), and a
focusing singularity forces \(k_{\min}(t)\to\infty\), the infimum on the
right-hand side diverges: \[
\inf_{|k|\ge k_{\min}(t)} \rho_k \xrightarrow[t\to T^*]{} \infty,
\] and hence \[
\frac{|T_{diss}|}{|T_{stretch}|} \to \infty \quad \text{as } t\to T^*.
\]

Consequently, for high-entropy (fractal) profiles, the geometric
depletion constant satisfies \(C_{geom}(\Xi) \to 0\) sufficiently fast
that \(C_{geom}(\Xi)\|u\|_{L^2} < \nu\), ensuring the solution remains
within the regularity domain. In the global classification of Section
12, these high-entropy configurations comprise the fractal stratum
\(\Omega_{\mathrm{Frac}}\). The key point is not an additive identity
for spectral ratios, but the fact that viscous dissipation dominates
vortex stretching at every active high-frequency scale in the fractal
cascade.

:::\{prf:proposition\} Conditional Frequency-Localized Ratio Test
:label: pro-conditional-frequency-localized-ratio-test

Let \(\Sigma_t\) be the support of the vorticity. If
\(\dim_H(\Sigma_t) > 1\) (the high-entropy regime), then locally:
\[ \frac{|T_{diss}|}{|T_{stretch}|} \to \infty \quad \text{as } \delta \to 0 \]

:::\{prf:remark\} Geometric interpretation :label:
rem-geometric-interpretation

The frequency-localized analysis reveals the fundamental incompatibility
between turbulent cascades and singularity formation. If vorticity
exhibits oscillations at frequency \(k\), where \(k \to \infty\)
characterizes the fractal depth of turbulent structures, the stretching
term grows linearly as \(O(k)\) while dissipation grows quadratically as
\(O(k^2)\). This spectral penalty of the Laplacian ensures that even
with perfect alignment (\(\cos(\theta) = 1\)), viscous dissipation
dominates vortex stretching at small scales, preventing the formation of
singularities from complex, multi-scale vorticity distributions.

(sec-the-ckn-barrier)= \#\#\# 3.2. The CKN Barrier :::

:::\{prf:definition\} Parabolic Hausdorff Measure :label:
def-parabolic-hausdorff-measure

For a set \(\Sigma \subset \mathbb{R}^3 \times \mathbb{R}\), the
\(s\)-dimensional parabolic Hausdorff measure is defined as:
\[ \mathcal{P}^s(\Sigma) = \lim_{\delta \to 0} \inf \left\{ \sum_{i} r_i^s : \Sigma \subset \bigcup_i Q_{r_i}(x_i,t_i) \right\} \]
where \(Q_r(x,t) = \{(y,s) : |y-x| < r, |s-t| < r^2\}\) denotes a
parabolic cylinder. :::

We refer to Caffarelli, Kohn, and Nirenberg (1982) for the complete
proof. The key idea is that suitable weak solutions satisfy a local
energy inequality, and points of singularity must concentrate energy in
a manner incompatible with dimension greater than 1.

:::\{prf:theorem\} Caffarelli-Kohn-Nirenberg Partial Regularity :label:
the-caffarelli-kohn-nirenberg-partial-regularity

Let \(u\) be a suitable weak solution of the Navier-Stokes equations.
Then the singular set \(\Sigma^* \subset \mathbb{R}^3 \times (0,T)\)
satisfies: \[ \mathcal{P}^1(\Sigma^*) = 0 \]

:::

By Proposition 3.1, high-entropy states with \(\dim_H > 1\) satisfy
\(C_{geom}(\Xi) \to 0\), placing them within the regularity domain where
\(C_{geom}(\Xi)\|u\|_{L^2} < \nu\).

By the CKN theorem (Theorem 3.2), such states cannot develop
singularities as \(\mathcal{P}^1(\Sigma^*) = 0\) excludes sets of
dimension greater than 1.

Therefore, assuming the validity of the dimension reduction arguments,
only low-entropy, geometrically coherent structures with
\(\dim_H \le 1\) can potentially exit the regularity domain. For such
structures, the coherence constant \(C_{geom}(\Xi)\) remains bounded
away from zero, yielding the required inequality.

:::\{prf:corollary\} Geometric Selection Principle :label:
cor-geometric-selection-principle

The CKN theorem imposes a strict geometric constraint on potential
singularities: - \textbf{Case 1:} High entropy configurations with
\(\dim_H(\Sigma^*) > 1\) are excluded a priori - \textbf{Case 2:} Low
entropy configurations with \(\dim_H(\Sigma^*) \le 1\) (isolated points
or filaments) remain admissible

\textbf{Conditional Theorem 3.3 (Nonlinear Depletion Inequality).}
Combining the CKN constraint with Proposition 3.1, any potential
singular profile must satisfy: \[ C_{geom}(\Xi)\|u\|_{L^2} \ge \nu \]
where \(C_{geom}(\Xi)\) is the geometric coherence constant from
Definition 3.1.

:::

:::\{prf:remark\} Geometric Coherence Requirement :label:
rem-geometric-coherence-requirement

The partial regularity theorem acts as a geometric sieve, forcing
potential singularities into simple, coherent structures (cylinders or
helices). This geometric selection principle motivates the subsequent
analysis of axial pressure defocusing (Section 4) and spectral
coercivity (Section 6), which provide additional constraints on these
geometrically simple configurations.

(sec-axial-pressure-defocusing-and-singular-integral-co)= \#\# 4. Axial
Pressure Defocusing and Singular Integral Control

This section analyzes vortex tubes concentrated in cylindrical regions
and establishes constraints on their evolution through the Biot-Savart
representation and geometric depletion principles.

(sec-cylindrical-vortex-tube-configuration)= \#\#\# 4.1. Cylindrical
Vortex Tube Configuration :::

:::\{prf:definition\} Cylindrical Vortex Tube :label:
def-cylindrical-vortex-tube

A cylindrical vortex tube configuration at time \(t\) is characterized
by vorticity \(\omega(x,t)\) concentrated in a cylindrical region:
\[ \text{supp}(\omega) \subset \mathcal{C}_{R,L}(t) := \{x \in \mathbb{R}^3 : r < R(t), |z| < L(t)\} \]
where \(r = \sqrt{x_1^2 + x_2^2}\) is the cylindrical radius, \(R(t)\)
is the tube radius, and \(L(t)\) is the tube length. :::

:::\{prf:definition\} Strain Tensor :label: def-strain-tensor

For a velocity field \(u\) solving the Navier-Stokes equations, the
strain tensor is defined as:
\[ S(x,t) = \frac{1}{2}\left(\nabla u(x,t) + (\nabla u(x,t))^T\right) \]
:::

:::\{prf:definition\} High-Twist Filament / ``Barber Pole'' :label:
def-high-twist-filament-barber-pole

We call a smooth, coherent vortex filament a \textbf{High-Twist
Filament} (descriptively, a ``Barber Pole'\,' configuration) if it is
characterized by: 1. \textbf{Low Swirl:} \(\mathcal{S} < \sqrt{2}\)
(evading spectral coercivity of Section 6) 2. \textbf{Finite
Renormalized Energy:} \(\|\mathbf{V}\|_{H^1_\rho} < \infty\) (satisfying
variational smoothness of Section 8) 3. \textbf{Unbounded Twist:} The
vorticity direction field satisfies
\(\|\nabla \xi\|_{L^\infty} \to \infty\) as \(t \to T^*\) :::

:::\{prf:remark\} Physical interpretation of Definition 4.3 :label:
rem-physical-interpretation-of-definition-43

This regime is the unique topological configuration that lies in the
intersection of the failure sets for Axial Defocusing (this section),
Spectral Coercivity (Section 6), and Variational Efficiency (Section 8).
It represents a coherent filament with low rotation but unbounded axial
twist (heuristically reminiscent of the spiral pattern on a barber's
pole with increasing pitch). The subsequent analysis is devoted to
proving that such high-twist filaments cannot occur as blow-up profiles.
:::

:::\{prf:assumption\} Finite Energy :label: ass-finite-energy

We consider Leray-Hopf solutions with finite initial energy:
\[ E_0 = \frac{1}{2} \int_{\mathbb{R}^3} |u_0(x)|^2 \, dx < \infty \]

(sec-biot-savart-representation-and-singular-integral-t)= \#\#\# 4.2.
Biot-Savart Representation and Singular Integral Theory :::

:::\{prf:definition\} Biot-Savart Law :label: def-biot-savart-law

The velocity field is recovered from vorticity through the Biot-Savart
integral:
\[ u(x,t) = -\frac{1}{4\pi} \int_{\mathbb{R}^3} \frac{x-y}{|x-y|^3} \times \omega(y,t) \, dy \]
:::

This follows from standard Calderón-Zygmund theory. The kernel \(K\)
arises from differentiating the Biot-Savart kernel and satisfies the
required cancellation and homogeneity conditions.

:::\{prf:lemma\} Calderón-Zygmund Structure :label:
lem-calderón-zygmund-structure

The strain tensor \(S\) can be expressed as a singular integral
operator:
\[ S(x,t) = \mathrm{p.v.} \int_{\mathbb{R}^3} K(x-y)\,\omega(y,t)\,dy \]
where \(K\) is a homogeneous kernel of degree \(-3\) with mean zero on
spheres. The associated operator \(\mathcal{T}[\omega] = S\) satisfies:
- \(\mathcal{T}: L^p(\mathbb{R}^3) \to L^p(\mathbb{R}^3)\) for
\(1 < p < \infty\) - \(\mathcal{T}\) is of weak type \((1,1)\)

:::

We refer to Beale, Kato, and Majda (1984) for the complete proof. The
key estimate combines the Biot-Savart representation with logarithmic
inequalities:
\[ \|\nabla u(\cdot,t)\|_{L^\infty} \le C \|\omega(\cdot,t)\|_{L^\infty}\left(1 + \log^+ \frac{\|\omega(\cdot,t)\|_{H^s}}{\|\omega(\cdot,t)\|_{L^\infty}}\right), \quad s > \frac{5}{2} \]
Thus blow-up requires
\(\int_0^{T^*}\|\nabla u(\cdot,t)\|_{L^\infty} dt = \infty\).

:::\{prf:theorem\} Beale-Kato-Majda Criterion :label:
the-beale-kato-majda-criterion

Let \(u\) be a smooth solution of the Navier-Stokes equations on
\([0,T)\) with vorticity \(\omega\). Then \(u\) can be continued
smoothly beyond time \(T\) if and only if:
\[ \int_0^T \|\omega(\cdot,t)\|_{L^\infty} \, dt < \infty \]

:::

:::\{prf:corollary\} Strain Integrability Criterion :label:
cor-strain-integrability-criterion

For a cylindrical vortex tube configuration, blow-up is prevented if the
strain satisfies:
\[ \int_0^{T^*} \|S(\cdot,t)\|_{L^\infty} \, dt < \infty \] :::

:::\{prf:remark\} Geometric Control Strategy :label:
rem-geometric-control-strategy

To establish regularity for cylindrical tubes, we must prove that the
strain norm is controlled by a subcritical function:
\[ \|S(\cdot,t)\|_{L^\infty} \le \Phi(\|\omega(\cdot,t)\|_{L^\infty}, E_0, R(t), L(t)) \]
where \(\int_0^{T^*}\Phi(\cdots)\,dt < \infty\). The geometric depletion
principle provides the necessary estimates.

(sec-constantin-fefferman-geometric-depletion-principle)= \#\#\# 4.3.
Constantin-Fefferman Geometric Depletion Principle :::

:::\{prf:definition\} Vorticity Direction Field :label:
def-vorticity-direction-field

The direction field of vorticity is defined as:
\[ \xi(x,t) = \frac{\omega(x,t)}{|\omega(x,t)|}, \quad |\xi| = 1 \text{ where } \omega \neq 0 \]
:::

:::\{prf:definition\} Stretching Rate :label: def-stretching-rate

The stretching rate along vortex lines is the scalar quantity:
\[ \alpha(x,t) = \xi(x,t) \cdot S(x,t) \cdot \xi(x,t) \] :::

Starting from the vorticity equation
\(\partial_t \omega + (u \cdot \nabla)\omega = S\omega + \nu \Delta \omega\),
write \(\omega = |\omega|\xi\) with \(|\xi| = 1\). Taking the inner
product with \(\xi\) and using \(\xi \cdot \Delta\xi = -|\nabla\xi|^2\)
yields the result.

:::\{prf:lemma\} Vorticity Magnitude Evolution :label:
lem-vorticity-magnitude-evolution

The magnitude of vorticity evolves according to:
\[ \partial_t |\omega| + (u \cdot \nabla)|\omega| = \alpha |\omega| + \nu\left(\Delta |\omega| - |\omega||\nabla\xi|^2\right) \]

:::

Differentiate \(\omega = |\omega|\xi\) and use the constraint
\(|\xi| = 1\) to derive the orthogonal projection
\((I - \xi \otimes \xi)\) that maintains unit length.

:::\{prf:lemma\} Direction Field Evolution :label:
lem-direction-field-evolution

The direction field satisfies (formally, away from \(\omega = 0\)):
\[ \partial_t \xi + (u \cdot \nabla)\xi = (I - \xi \otimes \xi)S\xi + \nu\left(\Delta\xi + 2\nabla\log|\omega| \cdot \nabla\xi\right) \]

:::

We refer to Constantin and Fefferman (1993) for the complete proof. The
key insight is that bounded \(\|\nabla\xi\|_{L^\infty}\) prevents
geometric concentration of vortex lines, which limits the stretching
rate \(\alpha\). The viscous term \(-\nu|\omega||\nabla\xi|^2\) in the
magnitude equation provides dissipation that dominates stretching when
\(\|\nabla\xi\|_{L^\infty}\) is integrable in time.

:::\{prf:theorem\} Constantin-Fefferman Geometric Depletion :label:
the-constantin-fefferman-geometric-depletion

Let \(u\) be a smooth solution on \([0,T)\) with vorticity \(\omega\)
and direction field \(\xi\). If
\[ \int_0^T \|\nabla\xi(\cdot,t)\|_{L^\infty}^2 \, dt < \infty \] then
the solution remains regular at time \(T\).

:::

:::\{prf:remark\} Geometric Depletion Mechanism :label:
rem-geometric-depletion-mechanism

The Constantin-Fefferman criterion reveals that regularity of the
vorticity direction field prevents singularity formation. The stretching
rate \(\alpha = \xi \cdot S\xi\) appears as a source term in the
vorticity magnitude equation, while \(\|\nabla\xi\|_{L^\infty}\)
controls the parabolic regularization. When the direction field remains
smooth, vortex stretching is geometrically depleted by viscous
dissipation. :::

Multiply the vorticity equation
\(\partial_t\omega + (u \cdot \nabla)\omega = S\omega + \nu\Delta\omega\)
by \(\omega\) and integrate over \(\mathbb{R}^3\):
\[ \frac{1}{2}\frac{d}{dt}\int_{\mathbb{R}^3}|\omega|^2\,dx + \nu\int_{\mathbb{R}^3}|\nabla\omega|^2\,dx = \int_{\mathbb{R}^3}\omega \cdot (S\omega)\,dx \]

Using the decomposition \(\omega = |\omega|\xi\) and the stretching rate
\(\alpha = \xi \cdot S\xi\):
\[ \int_{\mathbb{R}^3}\omega \cdot (S\omega)\,dx = \int_{\mathbb{R}^3}(\xi \cdot S\xi)|\omega|^2\,dx \]

By the Calderón-Zygmund theory (Lemma 4.1) and interpolation
inequalities:
\[ |(\xi \cdot S\xi)(x)| \le \|S\|_{BMO}\|\xi\|_{L^\infty}^2 \le C\|\nabla u\|_{BMO} \]

Using the commutator estimate for the Riesz transform and the bounded
mean oscillation (BMO) norm:
\[ \|\nabla u\|_{BMO} \le C(\|\omega\|_{L^2} + \|\nabla\xi\|_{L^\infty}\|\omega\|_{L^2}) \]

Therefore:
\[ \left|\int_{\mathbb{R}^3}(\xi \cdot S\xi)|\omega|^2\,dx\right| \le C\|\nabla\xi\|_{L^\infty}\|\omega\|_{L^2}\|\nabla\omega\|_{L^2} \]

Applying Young's inequality with \(\epsilon\):
\[ C\|\nabla\xi\|_{L^\infty}\|\omega\|_{L^2}\|\nabla\omega\|_{L^2} \le \frac{\nu}{2}\|\nabla\omega\|_{L^2}^2 + \frac{C^2}{2\nu}\|\nabla\xi\|_{L^\infty}^2\|\omega\|_{L^2}^2 \]

This yields:
\[ \frac{d}{dt}\|\omega(\cdot,t)\|_{L^2}^2 + \nu\|\nabla\omega(\cdot,t)\|_{L^2}^2 \le \frac{C^2}{\nu}\|\nabla\xi(\cdot,t)\|_{L^\infty}^2\|\omega(\cdot,t)\|_{L^2}^2 \]

By Grönwall's lemma, if
\(\int_0^T\|\nabla\xi(\cdot,t)\|_{L^\infty}^2\,dt < \infty\), then
\(\|\omega(\cdot,t)\|_{L^2}\) remains bounded for all \(t \in [0,T]\),
preventing blow-up. \(\hfill\square\)

For a straight tube, the geometric structure suggests that \(\xi\) is
approximately constant along the axial direction and varies only mildly
across the tube. A rigorous implementation would proceed by: 1. Writing
the evolution equation for \(\nabla\xi\) explicitly from the above
formula for \(\partial_t\xi\). 2. Using the straight-tube assumptions
(bounded curvature of the tube centreline, small torsion, no kinks) to
control the advective term \((u\cdot\nabla)\xi\) and the source term
\((I-\xi\otimes\xi)S\xi\). 3. Exploiting the Biot--Savart control on
\(S\) (Section 4.4) to bound \(\|(I-\xi\otimes\xi)S\xi\|_{L^\infty}\) in
terms of \(\|\nabla\xi\|_{L^\infty}\) and global energy norms.

Under these conditions, it is natural to isolate the following
quantitative alignment hypothesis.

\textbf{Hypothesis 4.5 (Tube-alignment condition).} There exist
constants \(C_1,C_2>0\) such that, for all \(t<T^*\), \[
\frac{d}{dt} \|\nabla\xi(\cdot,t)\|_{L^\infty}^2
 \le C_1\Big(1 + \|\nabla\xi(\cdot,t)\|_{L^\infty}^2\Big),
\] and \[
\int_0^{T^*} \|\nabla\xi(\cdot,t)\|_{L^\infty}^2 \, dt \le C_2.
\]

The first inequality encodes the idea that the growth of
\(\|\nabla\xi\|_{L^\infty}\) can be controlled in terms of itself and
global norms (via the tube geometry and the Biot--Savart bounds on
\(S\)); the second states the Constantin--Fefferman integrability
condition. Hypothesis 4.5 is precisely the geometric input needed to
apply Theorem 4.2 and the BKM reduction: combined with Theorem 4.2, it
ensures that the stretching rate \(\alpha = \xi\cdot S\,\xi\) is
subordinated to the viscous dissipation and cannot drive blow-up in the
straight-tube class. Establishing Hypothesis 4.5 from first principles
is a deep open problem in its own right; the remainder of this section
is conditional on its validity.

(sec-near-field-far-field-decomposition-of-the-strain)= \#\#\# 4.4.
Near-Field / Far-Field Decomposition of the Strain

To make the above program precise, one decomposes the strain into
self-induced and background components: \[
S(x,t) = S_{self}(x,t) + S_{far}(x,t),
\] where \(S_{self}\) is generated by the vorticity inside a tubular
neighborhood of radius, say, \(2R(t)\) around the core, and \(S_{far}\)
is generated by the complement.

(sec-self-induced-strain-of-a-straight-tube)= \#\#\#\# 4.4.1.
Self-induced strain of a straight tube

We now detail how the tube geometry constrains the ``self-strain'\,'
\(S_{self}\).

\textbf{Lemma 4.3 (Self-induced strain bound for a straight tube).} In
the setting of Section 4.1, assume in addition that in cylindrical
coordinates \((r,\theta,z)\) adapted to the axis \[
\omega_{tube}(x,t) = \omega_\theta(r,z,t)\,e_\theta
\] and that \(\omega_\theta\) is supported in \(\{r<R(t), |z|<L(t)\}\)
with \[
\|\omega_\theta(\cdot,t)\|_{L^\infty} \le \Omega_\infty(t).
\] Write the Biot--Savart law restricted to the tube as \[
u_{self}(x,t) = -\frac1{4\pi} \int_{\text{tube}} \frac{x-y}{|x-y|^3} \times \omega_{tube}(y,t)\,dy,
\] and define
\(S_{self} = \frac12(\nabla u_{self} + \nabla u_{self}^\top)\).

Fix \(x\) in the core region \(\{r\le R(t)/2,\ |z|\le L(t)/2\}\). Split
the tube into ``near'\,' and ``intermediate'\,' regions relative to
\(x\): \[
\text{tube} = \{ |z_y - z_x|\le 2R(t),\ r_y<2R(t)\}
          \cup \{2R(t)<|z_y - z_x|\le 2L(t),\ r_y<2R(t)\}.
\] Correspondingly, write \(S_{self} = S_{near}+S_{mid}\).

\emph{Near region estimate.} For the near region, \(|x-y|\sim R(t)\) and
the kernel behaves like \(|x-y|^{-3}\). Differentiating the kernel gives
\(|\nabla_x K(x-y)|\lesssim |x-y|^{-4}\). Hence \[
|S_{near}(x,t)|
 \le C \int_{|z_y-z_x|\le 2R(t),\, r_y<2R(t)} \frac{|\omega_{tube}(y,t)|}{|x-y|^3}\,dy
\lesssim \Omega_\infty(t),
\] where we used that the volume of the near region is \(\sim R(t)^3\)
and \(|x-y|\sim R(t)\).

\emph{Intermediate region estimate.} For the intermediate region, we
integrate along the axis while exploiting cancellations of the kernel in
\(\theta\). Writing \(y = (r_y,\theta_y,z_y)\) and fixing \(r_y<2R(t)\),
the singularity as \(z_y\to z_x\) has already been removed by excluding
\(|z_y-z_x|\le 2R(t)\). Thus \[
|S_{mid}(x,t)|
 \le C \int_{2R(t)<|z_y-z_x|\le 2L(t)} \int_0^{2R(t)} \frac{\Omega_\infty(t)\,r_y}{|x-y|^3}\,dr_y\,dz_y.
\] For \(|z_y-z_x|>2R(t)\) and \(r_x\le R(t)/2\), we have
\(|x-y|\gtrsim |z_y-z_x|\), so \[
|S_{mid}(x,t)|
 \lesssim \Omega_\infty(t) \int_{2R(t)}^{2L(t)} \frac{R(t)^2}{|z_y-z_x|^3}\,dz_y
 \lesssim \Omega_\infty(t)\big(1+\log\tfrac{L(t)}{R(t)}\big).
\] Combining the near and intermediate estimates yields \[
\|S_{self}(\cdot,t)\|_{L^\infty(\text{core})}
 \le C\,\Omega_\infty(t)\,\Big(1 + \log\frac{L(t)}{R(t)}\Big).
\] This is the straight-tube analogue of the classical logarithmic bound
for singular integrals with highly concentrated support. It shows
that---even if \(\|\omega\|_{L^\infty}\) is large---the amplification of
\(S_{self}\) by the geometry is at worst logarithmic in the aspect ratio
\(L(t)/R(t)\).

\emph{Proof.} The derivation above only used the support properties of
\(\omega_{tube}\), the antisymmetry and homogeneity of the Biot--Savart
kernel, and the straightness and finite length of the tube. All
integrals are absolutely convergent under the stated assumptions, so the
principal value is well-defined and the estimates follow by standard
singular-integral bounds and elementary comparisons. \(\hfill\square\)

(sec-bounding-the-far-field-strain-via-finite-energy)= \#\#\#\# 4.4.2.
Bounding the far-field strain via finite energy

We now make the far-field estimate rigorous.

\textbf{Lemma 4.4 (Far-field strain bound).} With notation as above, for
any fixed \(t<T^*\) and any \(x\) in the core region, \[
|S_{far}(x,t)| \le C R(t)^{-3/2} \|\omega(\cdot,t)\|_{L^2(\mathbb{R}^3)},
\] and hence \[
\|S_{far}(\cdot,t)\|_{L^\infty(\text{core})} \le C R(t)^{-3/2} \|\omega(\cdot,t)\|_{L^2}.
\]

\emph{Proof.} For the far-field component \(S_{far}\), we use standard
energy bounds and decay of the kernel. Write \[
S_{far}(x,t) = \int_{|y-x|\ge 2R(t)} K(x-y)\,\omega(y,t)\,dy.
\] Fix \(x\) in the core. For \(|y-x|\ge 2R(t)\), we have
\(|K(x-y)|\lesssim |x-y|^{-3}\). Split the integral dyadically in the
radial variable \(\rho = |x-y|\): \[
S_{far}(x,t) = \sum_{k=0}^{\infty} \int_{2^k R(t)\le |x-y| < 2^{k+1}R(t)} K(x-y)\,\omega(y,t)\,dy.
\] Estimating each dyadic annulus by Cauchy--Schwarz: \[
\bigg|\int_{2^kR \le |x-y| < 2^{k+1}R} K(x-y)\,\omega(y,t)\,dy\bigg|
 \le \|K\|_{L^2(A_k)} \|\omega(\cdot,t)\|_{L^2},
\] where \(A_k = \{y:2^kR(t)\le |x-y|<2^{k+1}R(t)\}\). Since
\(|K|\lesssim |x-y|^{-3}\) and \(|A_k|\sim (2^{k+1}R)^3\), we have \[
\|K\|_{L^2(A_k)}^2 \lesssim \int_{2^kR}^{2^{k+1}R} \rho^{-6}\,\rho^2\,d\rho \sim (2^kR)^{-3},
\] so \(\|K\|_{L^2(A_k)}\lesssim (2^kR)^{-3/2}\). Therefore \[
|S_{far}(x,t)|
 \le \sum_{k=0}^{\infty} C (2^kR(t))^{-3/2} \|\omega(\cdot,t)\|_{L^2}
 \lesssim R(t)^{-3/2} \|\omega(\cdot,t)\|_{L^2},
\] by summing the geometric series in \(2^{-3k/2}\). This proves the
pointwise bound and thus the \(L^\infty\) bound. \(\hfill\square\)

Invoking the Leray energy inequality: \[
\int_0^{T^*} \int_{\mathbb{R}^3} |\nabla u(x,t)|^2 dx \, dt \le \frac1{2\nu}\|u_0\|_{L^2}^2 =: C_E < \infty.
\] we obtain \[
\int_0^{T^*} \|S_{far}(\cdot,t)\|_{L^\infty} \, dt
 \lesssim \int_0^{T^*} R(t)^{-3/2} \|\omega(\cdot,t)\|_{L^2} \, dt
 \le C(E_0)\,\sup_{t<T^*} R(t)^{-3/2}.
\] Thus, provided \(R(t)\) does not vanish too fast (e.g., under Type I
scaling \(R(t)\sim \sqrt{T^*-t}\)), the far-field contribution to
\(\|S\|_{L^\infty}\) is integrable in time.

(sec-critical-space-criteria-and-their-limitations)= \#\#\# 4.5.
Critical-Space Criteria and Their Limitations

Critical-space criteria provide an important benchmark for what a
regularity theory could, in principle, control. The
Ladyzhenskaya--Prodi--Serrin family asserts regularity if \[
u \in L^q(0,T; L^p(\mathbb{R}^3)), \qquad \frac{2}{q} + \frac{3}{p} = 1,\quad 3 < p \le \infty.
\] The endpoint \(L^5_tL^5_x\) is critical with respect to
Navier--Stokes scaling.

For a tube of radius \(R(t)\) and characteristic velocity \(U(t)\), one
can estimate the \(L^5\) norm as follows. Let
\(\Omega_{tube}(t) = \{r<R(t), |z|<L(t)\}\) and assume
\(|u(x,t)|\lesssim U(t)\) on \(\Omega_{tube}(t)\) and that \(u\) is
negligible outside. Then \[
\|u(\cdot,t)\|_{L^5}^5
 = \int_{\mathbb{R}^3} |u(x,t)|^5 dx
 \approx \int_{\Omega_{tube}(t)} |u(x,t)|^5 dx
 \lesssim U(t)^5 |\Omega_{tube}(t)|
 \sim U(t)^5 R(t)^2 L(t).
\] If mass and circulation conservation suggest
\(U(t) \sim \Gamma / R(t)\) for some circulation \(\Gamma\), then \[
\|u(\cdot,t)\|_{L^5}^5 \sim \Gamma^5 R(t)^{-3} L(t).
\] Under Type I scaling \(R(t)\sim \sqrt{T^*-t}\) with \(L(t)\) bounded,
this behaves like \((T^*-t)^{-3/2}\), and \[
\int_0^{T^*} (T^*-t)^{-3/2} dt = \infty.
\] Thus, even the ``mild'\,' Type I scaling is too singular for the
\(L^5_tL^5_x\) criterion: the critical Ladyzhenskaya--Prodi--Serrin
condition cannot be expected to control straight-tube blow-up. More
singular Type II scalings only worsen this divergence.

The straight-tube analysis in this paper therefore does not rely on
critical-space bounds. Instead, it is anchored in the BKM reduction, the
Constantin--Fefferman geometric depletion framework, and the
Biot--Savart--based strain estimates of Sections 4.2--4.4, together with
the geometric dichotomy in Section 4.6. The role of Section 4.5 is
purely diagnostic: it illustrates that classical critical-space criteria
are supercritical with respect to the tube geometry under consideration
and therefore must be replaced by genuinely geometric control.

(sec-geometric-stability-dichotomy)= \#\#\# 4.6. Geometric Stability
Dichotomy

We now assemble the previous estimates into a curvature dichotomy:
either the tube remains sufficiently straight for the logarithmic strain
bounds to apply, or any attempt to develop large curvature forces the
flow into a viscous/depleted regime controlled by Section 3 and the
anisotropic arguments of Section 6.5.

We first record the straight-tube regularity statement proved under
alignment and strain bounds.

\textbf{Proposition 4.3 (Exclusion of straight-tube blow-up under
Alignment).} Assume: 1. The vorticity is concentrated, for all
\(t<T^*\), in a slender, finite-length tube with radius \(R(t)\) and
length \(L(t)\) as above, with a uniform bound on the tube curvature and
torsion. 2. The direction field \(\xi\) satisfies the
Constantin--Fefferman alignment condition \[
   \int_0^{T^*} \|\nabla\xi(\cdot,t)\|_{L^\infty}^2 dt < \infty.
   \] 3. The near-field Biot--Savart analysis yields a logarithmic
self-strain bound \[
   \|S_{self}(\cdot,t)\|_{L^\infty} \lesssim \|\omega(\cdot,t)\|_{L^\infty}\big(1 + \log \tfrac{L(t)}{R(t)}\big).
   \] 4. The far-field strain satisfies an energy-based bound as above:
\[
   \|S_{far}(\cdot,t)\|_{L^\infty} \lesssim R(t)^{-3/2}\,\|\omega(\cdot,t)\|_{L^2},
   \] with \(R(t)\) controlled from below by Type I scaling: \[
   R(t) \gtrsim \sqrt{T^*-t} \quad \text{as } t\uparrow T^*.
   \]

Then the total strain is integrable in time: \[
\int_0^{T^*} \|S(\cdot,t)\|_{L^\infty} \, dt < \infty,
\] and by the BKM theorem no finite-time blow-up occurs in the
straight-tube class.

\emph{Proof.} Writing \(S = S_{self}+S_{far}\) and using (3)--(4), \[
\|S(\cdot,t)\|_{L^\infty}
 \le C\|\omega(\cdot,t)\|_{L^\infty}\Big(1 + \log \tfrac{L(t)}{R(t)}\Big)
    + C R(t)^{-3/2}\|\omega(\cdot,t)\|_{L^2}.
\] The energy inequality implies \(\|\omega(\cdot,t)\|_{L^2}\le C(E_0)\)
for all \(t<T^*\). Moreover, the CF alignment condition (2), combined
with the vorticity equation, yields a priori bounds on
\(\|\omega(\cdot,t)\|_{L^\infty}\) up to any time \(T<T^*\) (see
{[}@constantin1993{]}). Thus for each fixed \(T<T^*\), \[
\int_0^{T} \|S(\cdot,t)\|_{L^\infty} dt
 \le C_T \int_0^{T}\Big(1 + \log \tfrac{L(t)}{R(t)}\Big) dt
    + C(E_0) \int_0^{T} R(t)^{-3/2} dt.
\] If \(R(t)\gtrsim \sqrt{T^*-t}\) and \(L(t)\) remains bounded (or
increases at most polynomially), the second integral is finite near
\(T^*\) and the logarithmic factor is harmless. Hence \[
\int_0^{T^*} \|S(\cdot,t)\|_{L^\infty} dt < \infty.
\] By BKM (Theorem 4.1), this precludes blow-up at \(T^*\).

\textbf{Remark 4.3.1 (The Barber Pole Limit).} The analysis in this
section relies critically on the control of \(\|\nabla\xi\|_{L^\infty}\)
through the Constantin--Fefferman alignment condition. Given our
variational exclusion of fractal states (Section 8) and the spectral
coercivity for high-swirl configurations (Section 6), the only
configuration that could potentially evade all constraints is a
\textbf{low-swirl, coherent filament with unbounded twist}---what we
term the Barber Pole singularity. This would be a smooth, coherent
vortex tube with small swirl parameter \(\sigma \le \sigma_c\) but with
internal twist \(\kappa(t)=\|\nabla\xi(\cdot,t)\|_{L^\infty}\to\infty\)
as \(t \to T^*\), violating the alignment hypothesis. Verifying that
such configurations cannot form from smooth initial data is precisely
the task of Section 11, where we combine extremizer regularity with
nodal-set analysis to rule out the Barber Pole regime.

We now introduce the curvature dichotomy, which covers both the aligned
and kinked configurations.

Define \[
\kappa(t) := \|\nabla\xi(\cdot,t)\|_{L^\infty}
\] as a global measure of vortex-line curvature (and torsion) at time
\(t\).

\textbf{Theorem 4.6 (Curvature Dichotomy for Filamentary Structures).}
Let \(u\) be a Leray--Hopf solution with vorticity concentrated in a
slender tube as in Section 4.1. Then there exists a curvature threshold
\(K_{crit}>0\) such that, for any putative blow-up time \(T^*<\infty\),
one of the following regimes must hold on \((0,T^*)\), and in each case
blow-up is ruled out:

\textbf{Regime I (Coherent regime: \(\kappa(t)\le K_{crit}\) for all
\(t<T^*\)).} In this regime the direction field remains uniformly
aligned. Then, for any \(T<T^*\), \[
\int_0^T \|\nabla\xi(\cdot,t)\|_{L^\infty}^2 dt \le K_{crit}^2 T < \infty,
\] so the Constantin--Fefferman condition holds on \([0,T]\). Combined
with the logarithmic self-strain bound (Lemma 4.3), the far-field bound
(Lemma 4.4), and the Type I control of \(R(t)\), Proposition 4.3 applies
on each finite interval \([0,T]\), and the BKM criterion ensures that
\(u\) can be continued past \(T\). Since \(T<T^*\) was arbitrary, no
blow-up can occur at \(T^*\) in Regime I.

\textbf{Regime II (Incoherent regime: \(\kappa(t)\) exceeds
\(K_{crit}\)).} Assume there exists a time \(t_0<T^*\) with
\(\kappa(t_0)>K_{crit}\). Let \[
t_1 := \inf\{t\in(0,T^*): \kappa(t)\ge K_{crit}\}.
\] On \((0,t_1)\) we are in Regime I and the solution is smooth. At
\(t_1\) the curvature reaches the critical threshold. We claim that this
forces the flow into the depleted/viscous regime described in Sections 3
and 6.5, preventing blow-up.

To see this, note that \(\kappa(t_1)\ge K_{crit}\) means that on some
ball \(B_{r_0}(x_0)\) centered on the tube,
\(\|\nabla\xi(\cdot,t_1)\|_{L^\infty(B_{r_0})}\) is large. Two effects
follow:

\begin{enumerate}
\def\labelenumi{\arabic{enumi}.}
\item
  \textbf{Misalignment of stretching (geometric depletion).} By the
  evolution equation for \(\xi\) and the structure of \(S\) as a
  singular integral of \(\omega\), a large gradient of \(\xi\) implies
  that, on a substantial portion of \(B_{r_0}(x_0)\), the direction
  field deviates significantly from any fixed eigenvector of \(S\).
  Quantitatively, there exists \(\delta=\delta(K_{crit})>0\) such that
  \[
  \left|\int_{B_{r_0}(x_0)} (\xi\cdot S\xi)|\omega|^2 dx\right|
   \le (1-\delta) \int_{B_{r_0}(x_0)} |S|\,|\omega|^2 dx
    + C \|\nabla\xi\|_{L^\infty} \|\omega\|_{L^2(B_{r_0})} \|\nabla\omega\|_{L^2(B_{r_0})}.
  \] The last term is exactly of the form handled by Theorem 4.2: it can
  be absorbed by the viscous dissipation provided we track it in time.
  Thus, as soon as \(\kappa\) is large, the effective stretching rate
  \(\alpha = \xi\cdot S\xi\) becomes strictly less efficient than the
  ``worst-case'\,' aligned value \(|S|\), and the stretching
  contribution in the vorticity energy balance is dominated by the
  dissipation.
\item
  \textbf{Activation of anisotropic dissipation.} The large curvature
  implies strong variation of \(u\) and \(\omega\) along the tube
  direction. In local coordinates adapted to the tube, this manifests as
  large axial derivatives, e.g., \[
  |\partial_s u| \sim \frac{\Gamma}{R_\kappa}, \quad R_\kappa \approx \kappa^{-1},
  \] where \(s\) is arclength along the centreline. The viscous term
  \(-\nu\Delta u\) therefore contains a substantial component from
  \(\partial_s^2 u\), and the corresponding contribution to the
  dissipation \[
  \nu \int |\partial_s \omega|^2 \, dx
  \] grows as \(\kappa^2\). Section 6.5 (Topological Switch and Ribbon
  analysis) shows that such anisotropic concentration is unstable: any
  attempt to maintain a highly curved, filamentary configuration
  necessarily flattens into a sheet-like structure where the geometric
  depletion inequality of Section 3 applies, and the resulting
  ``ribbon'\,' is rapidly dissipated.
\end{enumerate}

Combining (1) and (2) gives a local-in-time inequality of the form \[
\frac{d}{dt} \|\omega(\cdot,t)\|_{L^2(B_{r_0})}^2
 + c_1 \|\nabla\omega(\cdot,t)\|_{L^2(B_{r_0})}^2
 \le C_1 \|\nabla\xi(\cdot,t)\|_{L^\infty(B_{r_0})} \|\omega(\cdot,t)\|_{L^2(B_{r_0})} \|\nabla\omega(\cdot,t)\|_{L^2(B_{r_0})},
\] with \(c_1>0\). Once \(\kappa\) exceeds \(K_{crit}\), the right-hand
side is dominated by the left-hand side, and Grönwall's inequality shows
that \(\|\omega(\cdot,t)\|_{L^2(B_{r_0})}\) cannot blow up on any
interval \((t_1,t_1+\varepsilon)\); in fact, the large curvature
triggers enhanced dissipation and drives the solution back toward a more
regular configuration. By patching such local estimates along the tube,
and using the global depletion results of Section 3, we deduce that the
solution cannot develop a singularity while \(\kappa\) is large.

Thus, in Regime II, the solution is forced into the viscous/depleted
regime and cannot blow up at \(T^*\). This completes the dichotomy: in
all cases, straight-tube--type blow-up is excluded. \(\hfill\square\)

\textbf{Lemma 4.7 (Curvature Dichotomy as a Branching Principle).} Let
\(u\) be a Leray--Hopf solution in the Type I branch of Definition 9.0.1
with vorticity concentrated in a slender tube. Then, up to passing to a
subsequence of times \(t_n\uparrow T^*\), exactly one of the following
holds:

\begin{enumerate}
\def\labelenumi{\arabic{enumi}.}
\item
  (\textbf{Low-twist coherent branch}) \(\kappa(t_n)\le K_{crit}\) for
  all \(n\), and the tube satisfies the hypotheses of Proposition 4.3
  and Theorem 4.6. In this case CF alignment and the defocusing
  mechanism preclude blow-up.
\item
  (\textbf{High-twist Barber Pole branch}) \(\kappa(t_n)\to\infty\) as
  \(n\to\infty\), i.e.~the filament enters the high-twist regime of
  Remark 4.3.1. In particular, any coherent Type I blow-up that evades
  the CF defocusing criteria must fall into the Barber Pole
  configuration treated in Section 11.
\end{enumerate}

\emph{Proof.} If \(\kappa(t)\) remains bounded along a sequence
\(t_n\uparrow T^*\), then by Theorem 4.6 we are in Regime I and the CF
alignment hypothesis holds on each finite interval \([0,t_n]\), ruling
out blow-up. Conversely, if for every subsequence \(\{t_n\}\) there
exists \(n\) with \(\kappa(t_n)>K_{crit}\), we may extract a subsequence
along which \(\kappa(t_n)\to\infty\), placing the flow in Regime II for
large \(n\). The discussion following Theorem 4.6 shows that such large
curvature forces strong misalignment and enhanced dissipation, and the
only way to sustain high curvature in a coherent filament is via the
Barber Pole scenario, where \(\kappa(t)\) diverges in the core while the
swirl parameter remains subcritical. \(\hfill\square\)

(sec-boundary-stabilization-and-the-luohou-scenario)= \#\#\# 4.7.
Boundary Stabilization and the Luo--Hou Scenario

A critical test of any straight-tube obstruction theory is its
consistency with the boundary-layer scenario of Luo and Hou
{[}@luo2014{]} for the 3D Euler equations. In that setting, a
singularity forms near the intersection of a symmetry plane and a
physical boundary, with a stagnation point of the pressure field at the
wall.

For the Navier--Stokes case considered here, the same Biot--Savart and
geometric-depletion framework applies in the bulk (\(\mathbb{R}^3\) or
\(\mathbb{T}^3\)), but the boundary introduces a kinematic constraint:
\[
u\cdot n = 0 \quad \text{on } \partial\Omega.
\] In a half-space, one can still decompose \(S = S_{self} + S_{far}\),
but the reflection method and image-vorticity contributions modify the
kernel. A rigorous adaptation of the above program would: - Compute the
effective kernel for \(S\) in the half-space using reflections. - Show
that the boundary condition suppresses the axial component of the mass
flux through the wall, weakening the capacity argument.

In this sense, the Luo--Hou scenario can be viewed as a
boundary-stabilized configuration where the mass-flux capacity argument
is altered by the wall. Since the Millennium formulation focuses on the
whole space or periodic domains without physical boundaries, the
straight-tube exclusion proved (conditionally) above applies to the
relevant Cauchy problem, while boundary-layer singularities remain a
separate, Euler-type phenomenon.

(sec-the-helical-stability-interval-the-collapsing-heli)= \#\# 5. The
Helical Stability Interval: The Collapsing Helix

The depletion and defocusing constraints imply a dichotomy: 1. Messy
shapes die by Depletion. 2. Straight shapes die by Ejection.

Therefore, a singular set \(\Sigma^*\) must reside in the null space of
both constraints. This requires a geometry that is ``locally straight''
(to avoid depletion) but ``topologically non-trivial'' (to maintain
coherence). This uniquely identifies the \textbf{Collapsing Helix}.

\textbf{Ansatz 5.1 (The Helical Profile).} We consider a local solution
of the form:
\[ \mathbf{u}(r, \theta, z) = u_r(r) \mathbf{e}_r + u_\theta(r) \mathbf{e}_\theta + w(r,z) \mathbf{e}_z \]
where \(u_\theta \neq 0\) (Swirl). This configuration maximizes the
Helicity \(\mathcal{H} = \mathbf{u} \cdot \boldsymbol{\omega}\), which
is known to suppress nonlinearity via Beltrami alignment
(\(\mathbf{u} \times \boldsymbol{\omega} \approx 0\)).

(sec-high-swirl-rigidity-and-pseudospectral-shielding)= \#\# 6.
High-Swirl Rigidity and Pseudospectral Shielding

This section establishes that spectral coercivity emerges naturally from
the swirl-dominated dynamics, transforming a hypothesis into a rigorous
theorem through scaling analysis and pseudospectral bounds.

(sec-the-swirl-parameterized-framework)= \#\#\# 6.0. The
Swirl-Parameterized Framework

\textbf{Definition 6.0 (Swirl-Parameterized Helical Profile).} We
introduce a parameter \(\sigma \in \mathbb{R}_+\) representing the
circulation strength \(\Gamma\) and define the helical profile ansatz:
\[ \mathbf{V}_\sigma(r,\theta,z) = (u_r(r,z), \sigma u_\theta(r), u_z(r,z)) \]
where \((u_r, u_\theta, u_z)\) are the normalized velocity components.

\textbf{Definition 6.1 (Operator Decomposition).} The linearized
operator around \(\mathbf{V}_\sigma\) admits the decomposition:
\[ \mathcal{L}_\sigma = \mathcal{H}_\sigma + \mathcal{S}_{kew,\sigma} \]
where \(\mathcal{H}_\sigma\) is the symmetric part and
\(\mathcal{S}_{kew,\sigma}\) is the skew-symmetric part with respect to
the weighted inner product \(\langle \cdot, \cdot \rangle_{L^2_\rho}\).

The spectral coercivity argument is expressed through the quadratic form
\[ \mathcal{Q}(\mathbf{w}) = \underbrace{\int_{\mathbb{R}^3} \frac{\mathcal{S}^2}{r^2} |\mathbf{w}|^2 \rho \, dy}_{\mathcal{I}_{cent}} - \underbrace{\int_{\mathbb{R}^3} (\mathbf{w} \cdot \nabla \mathbf{V}) \cdot \mathbf{w} \, \rho \, dy}_{\mathcal{I}_{stretch}}. \]
The \textbf{Coercivity Condition} asserts
\[ \mathcal{Q}(\mathbf{w}) \ge \mu \|\mathbf{w}\|_{H^1}^2 \quad \text{whenever} \quad \mathcal{S} \ge \sqrt{2}. \]
The critical threshold \(\mathcal{S}_{crit} = \sqrt{2}\) (the Benjamin
criterion) is derived from the balance between centrifugal repulsion and
inertial attraction through the weighted Hardy-Rellich inequality:
\[ \mathcal{S}_{crit}^2 = 2 = \frac{\text{Centrifugal coefficient}}{\text{Inertial stretching bound}}, \]
so linear instability is equivalent to \(\mathcal{S} < \sqrt{2}\).
Failure of this inequality (i.e., \(\mathcal{S} < \sqrt{2}\)) is
necessary for linear instability of the helical profile. To evaluate
\(\mathcal{Q}\), we adopt the dynamically rescaling coordinate system
that tracks the developing singularity, allowing the blow-up profile to
be analyzed as a quasi-stationary solution to a renormalized equation.

(sec-dynamic-rescaling-rotation-and-the-renormalized-fr)= \#\#\# 6.1.
Dynamic Rescaling, Rotation, and the Renormalized Frame

We assume the existence of a potential singularity at time \(T^*\). To
resolve the fine-scale geometry of the blow-up, we introduce a
time-dependent length scale \(\lambda(t)\), a spatial center \(x_c(t)\),
and a time-dependent rotation \(Q(t)\in SO(3)\) describing the
orientation of the core.

\textbf{Definition 6.1 (The Dynamic Rescaling Group with Rotation).} Let
\(\lambda \in C^1([0, T^*), \mathbb{R}^+)\) be a scaling parameter such
that \(\lambda(t) \to 0\) as \(t \to T^*\), let
\(\xi \in C^1([0, T^*), \mathbb{R}^3)\) be the trajectory of the
singular core, and let \(Q \in C^1([0,T^*),SO(3))\) be a time-dependent
rotation matrix. We define the \textbf{renormalized variables}
\((y, s)\) and the \textbf{self-similar profile} \(\mathbf{V}\) as
follows:

\begin{enumerate}
\def\labelenumi{\arabic{enumi}.}
\item
  \textbf{Renormalized Spacetime:}
  \[ y = \frac{Q(t)^\top (x - x_c(t))}{\lambda(t)}, \quad s(t) = \int_0^t \frac{1}{\lambda(\tau)^2} \, d\tau. \]
  Here, \(s\) represents the ``fast time'' of the singularity, with
  \(s \to \infty\) as \(t \to T^*\).
\item
  \textbf{Rescaled Velocity and Pressure:}
  \[ \mathbf{u}(x,t) = \frac{1}{\lambda(t)} Q(t)\, \mathbf{V}(y, s), \quad P(x,t) = \frac{1}{\lambda(t)^2} Q_s(y, s) \]
  for a suitable renormalized pressure \(Q_s\).
\item
  \textbf{Renormalized Vorticity:}
  \[ \boldsymbol{\omega}(x,t) = \frac{1}{\lambda(t)^2} Q(t)\,\boldsymbol{\Omega}(y, s), \quad \text{where } \boldsymbol{\Omega} = \nabla_y \times \mathbf{V}. \]
\item
  \textbf{Normalization Condition (Gauge Fixing):} We uniquely determine
  the scaling parameter \(\lambda(t)\) by imposing the following
  normalization on the renormalized profile:
  \[ \|\nabla \mathbf{V}(\cdot, s)\|_{L^2(B_1)} = 1 \quad \text{for all } s \in [s_0, \infty) \]
  This choice fixes the gauge of the renormalization group and
  rigorously prevents the `vanishing core' scenario
  (\(\mathbf{V} \to 0\)) in the renormalized frame by definition. The
  pathology of vanishing is thereby transferred to the scaling parameter
  \(\lambda(t)\), whose behavior is constrained by global energy bounds.
\end{enumerate}

Substituting these ansätze into the Navier-Stokes equations yields the
\textbf{Renormalized Navier-Stokes Equation with Rotation (RNSE)}
governing the profile \(\mathbf{V}\):

\[
\partial_s \mathbf{V}
 + a(s) \mathbf{V}
 + b(s) (y \cdot \nabla_y) \mathbf{V}
 + (\mathbf{V} \cdot \nabla_y)\mathbf{V}
 + (\boldsymbol{\Omega}(s)\times y)\cdot \nabla_y \mathbf{V}
 + \boldsymbol{\Omega}(s)\times \mathbf{V}
 = -\nabla_y Q_s + \nu \Delta_y \mathbf{V} + \mathbf{c}(s) \cdot \nabla_y \mathbf{V} \quad (6.1)
\]

where the \textbf{modulation parameters} are defined by the dynamics of
the scaling, translation, and rotation: \[
a(s) = -\lambda \dot{\lambda} \quad (\text{scaling rate}), \quad
\mathbf{c}(s) = \frac{\dot{\xi}}{\lambda} \quad (\text{core drift}),
\] and \(\boldsymbol{\Omega}(s)\in\mathbb{R}^3\) is the angular velocity
vector associated with \(Q\), characterized by \[
Q(t)^\top \dot{Q}(t)\, z = \boldsymbol{\Omega}(s)\times z \quad \text{for all } z\in\mathbb{R}^3.
\] In the standard self-similar blow-up scenario, we set
\(a(s) \equiv 1\) (corresponding to \(\lambda(t) \sim \sqrt{T^*-t}\))
and \(b(s) = a(s)\).

\textbf{Remark 6.1.} Equation (6.1) transforms the problem of
finite-time blow-up into the study of the asymptotic stability of the
profile \(\mathbf{V}(y,s)\) as \(s \to \infty\) in a dynamically
rescaled, co-moving, and co-rotating frame. * The term
\(a(s)\mathbf{V} + b(s)(y \cdot \nabla_y)\mathbf{V}\) represents the
Eulerian damping induced by the shrinking coordinate system. * The term
\((\mathbf{V} \cdot \nabla_y)\mathbf{V}\) is the nonlinearity. * The
term
\((\boldsymbol{\Omega}(s)\times y)\cdot \nabla_y \mathbf{V} + \boldsymbol{\Omega}(s)\times \mathbf{V}\)
generates rigid-body rotation; it is skew-symmetric in \(L^2_\rho\) and
does not contribute to the real part of the energy balance. * The term
\(\nabla_y Q_s\) is the pressure gradient which carries the
swirl-induced coercive barrier.

Crucially, a singularity can only occur if there exists a non-trivial
limit profile
\(\mathbf{V}_\infty(y) = \lim_{s\to\infty} \mathbf{V}(y,s)\) that
satisfies the steady-state version of (6.1) with constant modulation
parameters. In particular, a ``rotating wave'\,' in physical variables
corresponds to a stationary solution of (6.1) with constant
\(\boldsymbol{\Omega}\) in this co-rotating frame. We shall prove that
for helical profiles, the term \(-\nabla_y Q_s\) together with the
coercivity inequality develops a barrier preventing the existence of
such a steady state.

We now prove that the Helical Profile is unstable due to the
conservation of angular momentum. This instability is central to
excluding Type II blow-up.

(sec-derivation-compactness-of-the-singular-orbit)= \#\#\#\# 6.1.2.
Derivation: Compactness of the Singular Orbit

Before characterizing the geometry of the singularity, we must establish
the existence of a non-trivial limiting object. We prove that if a
finite-time singularity occurs, the renormalized trajectory
\(\mathcal{O} = \{ \mathbf{V}(\cdot, s) : s \in [s_0, \infty) \}\) is
pre-compact in \(L^2_{\rho}(\mathbb{R}^3)\), where
\(\rho(y) = e^{-|y|^2/4}\) is the Gaussian weight associated with the
self-similar scaling.

\textbf{Theorem 6.1 (Strong Compactness of the Blow-up Profile).} Assume
that \(\mathbf{u}(x,t)\) develops a singularity at time \(T^*\). Let
\((\lambda(t), x_c(t))\) be modulation parameters chosen to satisfy the
orthogonality conditions (defined below). Then, for any sequence of
times \(s_n \to \infty\), there exists a subsequence (still denoted
\(s_n\)) and a non-trivial profile
\(\mathbf{V}_\infty \in H^1_\rho(\mathbb{R}^3)\) such that:
\[ \mathbf{V}(\cdot, s_n) \longrightarrow \mathbf{V}_\infty \quad \text{strongly in } L^2_\rho(\mathbb{R}^3) \cap C^\infty_{loc}(\mathbb{R}^3) \]
Furthermore, \(\mathbf{V}_\infty\) is not identically zero.

\textbf{Proof.}

\textbf{Step 1: Uniform Bounds (The Energy Class).} First, we establish
that the profile does not blow up in the renormalized frame. By the
definition of the scaling parameter \(\lambda(t)\), we enforce the
normalization condition:
\[ \|\nabla \mathbf{V}(\cdot, s)\|_{L^2(B_1)} \sim 1 \quad \text{or} \quad \sup_{y \in B_1} |\mathbf{V}(y,s)| \sim 1 \]
(In Type I blow-up, this is natural. In Type II, we select
\(\lambda(t)\) specifically to saturate this bound). From the energy
inequality of the Navier-Stokes equations, we have global control of the
\(L^2\) norm. In self-similar variables, the Gaussian weight \(\rho(y)\)
confines the energy. We obtain the uniform bound:
\[ \sup_{s \ge s_0} \|\mathbf{V}(\cdot, s)\|_{H^1_\rho} \leq K \] This
implies weak compactness. There exists \(\mathbf{V}_\infty\) such that
\(\mathbf{V}(s_n) \rightharpoonup \mathbf{V}_\infty\) weakly in
\(H^1_\rho\).

\textbf{Step 2: Non-Vanishing (Ruling out the Null Limit).} We must
prove \(\mathbf{V}_\infty \not\equiv 0\). Assume, for contradiction,
that \(\mathbf{V}(s_n) \to 0\) strongly in \(L^2_{loc}\). By the
\textbf{Caffarelli-Kohn-Nirenberg (CKN) \(\epsilon\)-regularity
criterion}: * There exists a universal constant \(\epsilon_{CKN} > 0\)
such that if
\[ \limsup_{n \to \infty} \int_{B_1} |\mathbf{V}(y, s_n)|^2 + |\nabla \mathbf{V}(y, s_n)|^2 \, dy < \epsilon_{CKN} \]
then the point \((0, T^*)\) is a regular point. Since we assumed a
singularity exists at \(T^*\), the local energy near the core must stay
above the threshold \(\epsilon_{CKN}\).
\[ \liminf_{s \to \infty} \|\mathbf{V}(\cdot, s)\|_{L^2(B_1)} \ge \delta > 0 \]
Thus, the weak limit \(\mathbf{V}_\infty\) cannot be zero.

\textbf{Step 3: Non-Dichotomy (Tightness of the Measure).} We must prove
the energy does not ``split'' into two pieces that drift infinitely far
apart (mass leakage to infinity). The evolution of the squared weighted
norm satisfies the Lyapunov-type identity:
\[ \frac{1}{2}\frac{d}{ds} \int |\mathbf{V}|^2 \rho \, dy + \int |\nabla \mathbf{V}|^2 \rho \, dy + \frac{1}{2} \int |\mathbf{V}|^2 (|y|^2 - C) \rho \, dy \leq \text{Nonlinear Terms} \]
The term \(\frac{1}{2} \int |\mathbf{V}|^2 |y|^2 \rho\) acts as a
confining potential induced by the shrinking coordinate system. Standard
localization estimates (using cut-off functions \(\psi_R\) for large
\(R\)) show that:
\[ \lim_{R \to \infty} \sup_{s \ge s_0} \int_{|y|>R} |\mathbf{V}(y, s)|^2 \rho(y) \, dy = 0 \]
This \textbf{Tightness} property ensures that no mass escapes to
infinity. By the Fréchet-Kolmogorov theorem, uniform boundedness +
tightness implies strong pre-compactness in \(L^2_\rho\).

\textbf{Step 4: The Bootstrap to Smoothness.} Since
\(\mathbf{V}(s_n) \to \mathbf{V}_\infty\) in \(L^2\) and satisfies the
renormalized Navier-Stokes equation (which is parabolic), we apply
parabolic regularity theory. For any parabolic cylinder
\(Q = B_R \times [s, s+1]\), local \(L^2\) control implies \(H^k\)
control for all \(k\) due to the smoothing effect of the viscosity
\(\nu \Delta \mathbf{V}\). Therefore, the convergence upgrades to
\(C^\infty_{loc}\) topology.

\textbf{Conclusion.} The sequence of profiles \(\{\mathbf{V}(s_n)\}\)
converges to a non-trivial, smooth limit profile \(\mathbf{V}_\infty\)
which solves the steady-state (or ancient) Liouville equation. This
justifies the existence of the object analyzed in Theorem 6.3 and 6.4.

:::\{prf:theorem\} Energy Balance with Geometric Depletion :label:
the-energy-balance-with-geometric-depletion

Under the hypotheses of Theorem 4.2, the enstrophy evolution satisfies:
\[ \frac{1}{2}\frac{d}{dt}\|\omega(\cdot,t)\|_{L^2}^2 + \nu\|\nabla\omega(\cdot,t)\|_{L^2}^2 \le C\|\nabla\xi(\cdot,t)\|_{L^\infty}\|\omega(\cdot,t)\|_{L^2}\|\nabla\omega(\cdot,t)\|_{L^2} \]

(sec-derivation-the-persistence-of-circulation-the-swir)= \#\#\#\#
6.1.3. Derivation: The Persistence of Circulation (The Swirl Bootstrap)

To activate the spectral coercivity barrier (Theorem 6.3), the blow-up
profile must possess a non-trivial swirl ratio \(\mathcal{S}\). We now
prove that if the initial data possesses non-zero circulation, this
circulation cannot vanish in the singular limit.

\textbf{Step 1: The Parabolic Evolution of Circulation.} In the fixed
frame, the circulation \(\Gamma = r u_\theta\) satisfies the
drift-diffusion equation (assuming local axisymmetry of the tube):
\[ \partial_t \Gamma + \mathbf{u} \cdot \nabla \Gamma = \nu \Delta^* \Gamma \]
where
\(\Delta^* = \partial_r^2 - \frac{1}{r}\partial_r + \partial_z^2\).
Crucially, for axisymmetric flows, there is \textbf{no source term} for
circulation. It is only advected and diffused. The Maximum Principle
implies
\(\|\Gamma(\cdot, t)\|_{L^\infty} \leq \|\Gamma_0\|_{L^\infty}\). This
shows circulation does not blow up; it is bounded.

Now, we switch to the \textbf{Renormalized Frame}. Substituting
\(\Gamma(x,t) = \Phi(y,s)\) (since circulation is dimensionally
scaling-invariant, \(L \cdot L/T \cdot L = L^2/T\) vs \(\nu\)):
\[ \partial_s \Phi + \mathbf{V} \cdot \nabla_y \Phi - \nu \Delta_y^* \Phi = - \underbrace{a(s) \Phi}_{\text{Scaling Damping}} \]
where \(a(s) = -\lambda \dot{\lambda} \approx 1\) for self-similar
blow-up. This looks bad: the term \(-a(s)\Phi\) suggests exponential
decay of circulation in the renormalized frame. \textbf{However, we must
account for the coordinate drift.}

\textbf{Step 2: The Advective Concentration.} The velocity field
\(\mathbf{V}\) contains the ``confining wind'' due to the coordinate
rescaling: \[ \mathbf{V}(y,s) = \mathbf{V}_{fluid}(y,s) - a(s) y \] In
the singular core, the fluid must flow \textbf{inward} to sustain the
density of the singularity. Near the core \(r_y \approx 0\), the radial
velocity behaves as \(V_r \approx -C r_y\) (for focusing). The transport
term behaves as:
\[ V_r \partial_r \Phi \approx -C r_y \partial_r \Phi \] This inward
drift opposes the diffusion.

\textbf{Step 3: The Contradiction Argument.} To prove
\(\|\Phi_\infty\| > 0\) rigorously without getting bogged down in the
specific rates of \(a(s)\), we use a topological argument.

Assume, for the sake of contradiction, that
\(\|\Phi_\infty\|_{L^\infty} = 0\). Then the limiting profile
\(\mathbf{V}_\infty\) has \(V_\theta \equiv 0\). The profile
\(\mathbf{V}_\infty\) is thus a steady (or self-similar) solution to the
Navier-Stokes equations that is: 1. Non-trivial (by Derivation 1). 2.
Axisymmetric (by the Helical Ansatz). 3. \textbf{Swirl-Free} (Poloidal).

\textbf{Theorem (Ukhovskii \& Yudovich, 1968; Ladyzhenskaya):}
\emph{Global regularity holds for axisymmetric Navier-Stokes flows with
zero swirl.} More specifically, there are no non-trivial finite-energy
self-similar blow-up profiles in the class of swirl-free axisymmetric
solutions. The only solution is \(\mathbf{V} \equiv 0\).

\textbf{The Contradiction:} From \textbf{Derivation 1} (Compactness), we
proved that \(\mathbf{V}_\infty \not\equiv 0\). From \textbf{Classic
Regularity Theory}, if \(\text{Swirl} = 0\), then
\(\mathbf{V}_\infty \equiv 0\). Therefore, the assumption that
\(\text{Swirl} = 0\) must be false.

\textbf{Step 4: The Lower Bound.} We conclude that the singular set must
support a non-trivial circulation distribution.
\[ \liminf_{s \to \infty} \|\Phi(\cdot, s)\|_{L^\infty} > 0 \] Since
\(\Phi = r V_\theta\), this guarantees that \(V_\theta\) scales as
\(1/r\) near the core (preserving the vortex line topology). Thus, the
\textbf{Centrifugal Potential}
\(Q_{cyl} \sim \int \frac{V_\theta^2}{r} \sim \int \frac{1}{r^3}\)
remains the dominant term in the virial balance, validating the input
for Theorem 6.3.

:::\{prf:theorem\} Conservation of Circulation in the Singular Limit
:label: the-conservation-of-circulation-in-the-singular-limit

Let \(\mathbf{V}(y,s)\) be the solution to the renormalized
Navier-Stokes equations (6.1). Define the \textbf{Renormalized
Circulation} scalar field \(\Phi(y, s) = r_y V_\theta(y,s)\), where
\(r_y = \sqrt{y_1^2 + y_2^2}\). Assume the initial data has non-zero
circulation \(\Gamma_0 > 0\) on a set of macroscopic measure. Then, the
limiting profile
\(\mathbf{V}_\infty = \lim_{s \to \infty} \mathbf{V}(\cdot, s)\) cannot
be swirl-free. Specifically,
\[ \|\Phi_\infty\|_{L^\infty(\mathbb{R}^3)} \geq c_0 > 0 \]
Consequently, the centrifugal term in the pressure decomposition does
not vanish.

(sec-comparison-with-euler-parabolic-coupling-of-circul)= \#\#\#\#
6.1.4. Comparison with Euler: Parabolic Coupling of Circulation

A fundamental objection to the swirl-induced spectral coercivity
argument is its reliance on the conservation of angular momentum, a
property shared by the inviscid Euler equations. Given the numerical
evidence for finite-time blow-up in the 3D Euler equations (e.g., the
Luo-Hou scenario), one must clarify why the centrifugal barrier arrests
collapse in the Navier-Stokes case but fails (or is circumvented) in the
Euler limit.

The distinction lies in the \textbf{topological rigidity} of the angular
momentum field \(\Phi(y,s)\) induced by viscosity.

In the Euler equations (\(\nu = 0\)), the circulation \(\Gamma\) is
transported as a passive scalar along Lagrangian trajectories
(\(D_t \Gamma = 0\)). This hyperbolicity allows for Lagrangian
segregation: fluid filaments with high swirl can be distinct from
filaments with zero swirl. A singularity can form in Euler when a
non-rotating fluid parcel is driven into the core by pressure gradients,
bypassing the centrifugal barrier entirely because it carries no angular
momentum (\(\Gamma = 0\)). The barrier is present but permeable.

In the Navier-Stokes equations (\(\nu > 0\)), the circulation evolves
parbolically:
\[ \partial_s \Phi + \mathbf{V} \cdot \nabla \Phi - \nu \Delta \Phi = -a(s) \Phi \]
The Laplacian \(\nu \Delta \Phi\) acts as a \textbf{Homogenization
Operator}. By the \textbf{Parabolic Harnack Inequality}, the positivity
of swirl cannot be confined to Lagrangian packets. If the envelope of
the vortex possesses non-zero circulation, the viscosity instantaneously
diffuses this rotation into the core.

Consider the parabolic equation for circulation \(\Phi\) in the
renormalized coordinates:
\[ \partial_s \Phi + \mathbf{V} \cdot \nabla \Phi - \nu \Delta \Phi = -a(s) \Phi \]

Define the rescaled function
\(\tilde{\Phi}(y,s) = e^{\int_0^s a(\tau)d\tau} \Phi(y,s)\) to eliminate
the scaling term:
\[ \partial_s \tilde{\Phi} + \mathbf{V} \cdot \nabla \tilde{\Phi} = \nu \Delta \tilde{\Phi} \]

This is a linear parabolic equation with bounded drift \(\mathbf{V}\).
For any non-negative initial data \(\tilde{\Phi}_0 \not\equiv 0\), the
weak Harnack inequality (Moser, 1964) states that for any compact sets
\(K \subset K' \subset B_2\) and times \(0 < s_1 < s_2\):
\[ \inf_{y \in K, t \in [s_2, s_2+\delta]} \tilde{\Phi}(y,t) \geq C \sup_{y \in K', t \in [s_1, s_1+\delta]} \tilde{\Phi}(y,t) \]
where
\(C = C(\nu, \|\mathbf{V}\|_{L^\infty}, \text{dist}(K,\partial K'), s_2-s_1) > 0\).

Near the axis \(r = |y| \to 0\), the regularity of \(\mathbf{V}\)
implies \(\Phi(y) = O(|y|^2)\) (since \(V_\theta = \Phi/r\) must remain
bounded). Thus we can write:
\[ \Phi(y,s) = f(s)|y|^2 + \text{higher order terms} \]

Applying the Harnack inequality to the ratio \(\Phi(y)/|y|^2\) on the
annular region \(\{y : \epsilon < |y| < 2\epsilon\}\) for small
\(\epsilon > 0\):
\[ \inf_{|y| \sim \epsilon} \frac{\Phi(y,s)}{|y|^2} \geq C(\nu, \mathbf{V}) \sup_{|y| \sim 2\epsilon} \frac{\Phi(y,s)}{|y|^2} \]

Taking \(\epsilon \to 0\) and using the continuity of \(f(s)\):
\[ f(s) \geq C(\nu, \mathbf{V}) f(s) \int_{B_2 \setminus B_1} \frac{|\Phi(z,s)|}{|z|^2} \frac{dz}{|z|^2} \]

Since \(\Phi\) is non-negative and not identically zero by Theorem 6.2,
we have \(\int_{B_2} |\Phi(z)| dz > 0\). The normalization by \(|y|^2\)
ensures the estimate holds uniformly on compact sets \(K \subset B_1\),
yielding:
\[ \inf_{y \in K} \frac{|\Phi(y)|}{|y|^2} \geq C_{visc}(\nu, \mathbf{V}) \int_{B_2} |\Phi(z)| \, dz \]

This completes the proof. Unlike in Euler where \(\Phi\) satisfies a
hyperbolic transport equation allowing swirl-free pockets, the parabolic
nature of the Navier-Stokes circulation equation ensures instantaneous
diffusion of angular momentum throughout the core.

:::\{prf:proposition\} Harnack Estimate for Circulation :label:
pro-harnack-estimate-for-circulation

Let \(\mathbf{V}\) be a candidate blow-up profile. In the Navier-Stokes
evolution, the localized swirl-free region required to bypass the
centrifugal barrier is strictly forbidden. Specifically, for any compact
core region \(K \subset B_1\), there exists a constant
\(C_{visc}(\nu, \mathbf{V}) > 0\) such that:
\[ \inf_{y \in K} \frac{|\Phi(y)|}{|y|^2} \geq C_{visc} \int_{B_2} |\Phi(z)| \, dz \]

\textbf{Consequence for the Spectral Gap:} This parabolic support
coupling is the necessary condition for \textbf{Theorem 6.3}. 1.
\textbf{In Euler}, the spectral operator is
\(\mathcal{L}_{Euler} = \mathbf{V} \cdot \nabla + \nabla Q\). The
spectrum is continuous or purely imaginary. The centrifugal potential
exists, but the lack of ellipticity allows eigenmodes to localize in the
swirl-free pockets, evading the energy penalty. 2. \textbf{In
Navier-Stokes}, the operator is
\(\mathcal{L}_{NS} = -\nu \Delta + \mathbf{V} \cdot \nabla + \nabla Q\).
The viscous term \(-\nu \Delta\) combined with the positive centrifugal
potential \(W_{cent} \sim r^{-2}\) (derived from the locked profile)
allows us to invoke the Hardy-Rellich coercivity.

Therefore, the swirl-induced barrier is not purely inertial; it is a
viscous-inertial effect. The viscosity ensures the barrier is
impermeable, and the inertia provides the height of the barrier. The
Euler singularity is permitted because the barrier is permeable; the
Navier-Stokes singularity is forbidden because the barrier is
impermeable.

(sec-the-viscous-induction-of-core-rotation)= \#\#\#\# 6.1.5. The
Viscous Induction of Core Rotation

The existence of non-zero global circulation (Theorem 6.2) is a
necessary but not sufficient condition for the spectral coercivity
barrier. A potential objection remains: could the circulation
concentrate in a thin shell at the periphery of the profile, leaving the
singular core effectively swirl-free? Such ``Hollow Vortex''
configurations are permissible in the Euler equations.

In this subsection we work locally in an axisymmetric setting near the
tube centreline. We write \(x = (x_1,x_2,x_3)\) with \(x_3 = z\), let
\(r = \sqrt{x_1^2+x_2^2}\) denote the cylindrical radius, and denote by
\[
\omega_z(x,t) = (\nabla\times u(x,t))_z
\] the axial vorticity component. The goal is to show that positivity of
\(\omega_z\) on a shell \(r\in[r_1,r_2]\) forces strict positivity of
\(\omega_z\) (and hence of the circulation) in a neighbourhood of the
axis after a short time, ruling out a hollow vortex core.

The vorticity equation in Cartesian coordinates reads \[
\partial_t \boldsymbol{\omega} + (u\cdot\nabla)\boldsymbol{\omega}
 = \nu\Delta\boldsymbol{\omega} + (\boldsymbol{\omega}\cdot\nabla)u.
\] Taking the \(z\)--component gives \[
\partial_t \omega_z - \nu\Delta\omega_z + (u\cdot\nabla)\omega_z + c(x,t)\,\omega_z = 0,
\] where \[
c(x,t)\,\omega_z := -(\boldsymbol{\omega}\cdot\nabla u)_z.
\] Since \(u\) is smooth and axisymmetric, \(\boldsymbol{\omega}\) and
\(\nabla u\) are bounded on any compact space--time cylinder. In
particular, there exist \(R>0\) and \(M>0\) such that \[
|u(x,t)| + |\boldsymbol{\omega}(x,t)| + |\nabla u(x,t)| \le M
\] for all \((x,t)\in B_R(0)\times[t_0,t_0+1]\), where \(B_R(0)\) is the
Euclidean ball of radius \(R\) centred on the axis. Consequently the
drift \(b(x,t):=u(x,t)\) and the coefficient \(c(x,t)\) are bounded on
this cylinder, and the equation for \(\omega_z\) can be written in the
standard form \[
\partial_t \omega_z - \nu\Delta\omega_z + b(x,t)\cdot\nabla\omega_z + c(x,t)\,\omega_z = 0
\] with \(b,c\in L^\infty(B_R\times[t_0,t_0+1])\).

By assumption \(\omega_z(\cdot,t_0)\ge c_0>0\) on the cylindrical shell
\(\{x: r_1\le r\le r_2\}\). Since the solution is axisymmetric, this
region intersects the ball \(B_R(0)\) in a set of positive measure.
Standard interior parabolic Harnack inequalities for nonnegative
solutions of such equations (see, for example,
Ignatova--Kukavica--Ryzhik, \emph{The Harnack inequality for
second-order parabolic equations with divergence-free drifts of low
regularity}, Theorem 1.1) imply the following: there exist radii
\(0<\rho<r_1\) and times \(t_1>t_0\) and \(t_2>t_1\) such that \[
\inf_{B_\rho(0)\times[t_1,t_2]} \omega_z
 \ge C_{\mathrm{H}} \inf_{\{r_1\le r\le r_2\}\cap B_R} \omega_z(\cdot,t_0)
 \ge C_{\mathrm{H}} c_0,
\] where \(C_{\mathrm{H}}>0\) depends only on \(\nu\), \(R\), the
\(L^\infty\)--bounds on \(b,c\), and the geometry of the cylinders.
Setting \(\Delta t := t_2-t_1\) and \(c_1 := C_{\mathrm{H}}c_0\) yields
the claimed lower bound on \(\omega_z\) in \(B_\rho(0)\) for all
\(t\in[t_1,t_2]\). Renaming \(t_1\) as \(t_0+\Delta t/2\) completes the
proof. \(\hfill\square\)

As a direct consequence we obtain a quadratic lower bound on the
circulation near the axis.

\textbf{Corollary 6.1.5.1 (Quadratic Lower Bound for Circulation Near
the Axis).} Under the hypotheses of Lemma 6.1.5, let \[
\Phi(r,z,t) := \int_0^r s\,\omega_z(s,z,t)\,ds
\] denote the circulation in cylindrical coordinates for an axisymmetric
flow. Then, for all \(t\in[t_0+\tfrac{\Delta t}{2},\,t_0+\Delta t]\) and
all \(0\le r\le \rho\), \[
\Phi(r,z,t) \ge \tfrac12 c_1 r^2.
\] In particular the azimuthal velocity \(u_\theta = \Phi/r\) obeys the
solid-body lower bound \[
|u_\theta(r,z,t)| \ge \tfrac12 c_1 r
\] for \(r\le \rho\) and \(t\) in this time interval.

\emph{Proof.} For \(0\le r\le \rho\) and
\(t\in[t_0+\tfrac{\Delta t}{2},\,t_0+\Delta t]\), Lemma 6.1.5 gives
\(\omega_z(s,z,t)\ge c_1\) for all \(0\le s\le r\). Integrating, \[
\Phi(r,z,t)
 = \int_0^r s\,\omega_z(s,z,t)\,ds
 \ge c_1 \int_0^r s\,ds
 = \tfrac12 c_1 r^2.
\] Dividing by \(r\) yields the bound for \(u_\theta = \Phi/r\).
\(\hfill\square\)

\textbf{Remark 6.1.6 (Coordinate Singularity at the Axis).} The
vorticity equation for the axial component \(\omega_z\) is analysed
entirely in Cartesian coordinates. Although the swirl variable \(\Phi\)
satisfies, in cylindrical coordinates, an equation with an apparent
singular drift term proportional to \(1/r\), this is a coordinate
artefact: the underlying diffusion operator is the standard Laplacian on
\(\mathbb{R}^3\), which is uniformly elliptic across the axis. Lemma
6.1.5 applies an interior parabolic Harnack inequality to \(\omega_z\)
on a Euclidean ball around the axis, with bounded drift and zeroth-order
coefficients \((b,c)\in L^\infty\), and only afterwards translates the
resulting lower bound back into cylindrical language via the identity \[
\Phi(r,z,t) = \int_0^r s\,\omega_z(s,z,t)\,ds.
\] In particular, no singular boundary condition at \(r=0\) is imposed
or needed; the positivity of \(\omega_z\) near the axis is a genuine
interior parabolic effect, not an artefact of cylindrical coordinates.

\textbf{Physical Consequence.} The corollary excludes the swirl-free
tunnel configuration in the Navier--Stokes setting. Once the envelope of
the singularity carries nontrivial circulation in a shell away from the
axis, parabolic diffusion together with axisymmetry forces the axial
vorticity to become strictly positive in a neighbourhood of the axis,
and the circulation there behaves like that of a solid body rotation. In
particular, the swirl ratio \(\mathcal{S}(r) = V_\theta / V_z\) is
well-defined and bounded away from zero throughout the core, validating
the input assumptions for the low-swirl instability (Lemma 6.3.1) and
the spectral coercivity criterion (Theorem 6.3).

(sec-energetic-constraints-and-the-exclusion-of-type-ii)= \#\#\# 6.1.6.
Energetic Constraints and the Exclusion of Type II Divergence

The validity of the spectral coercivity (Theorem 6.3) and the Spectral
Gap analysis relies on the assumption that the effective Reynolds number
in the renormalized frame,
\(Re_{\lambda}(s) \sim \|\mathbf{V}(\cdot, s)\|_{L^\infty} / \nu\),
remains bounded. A divergence of \(Re_{\lambda}(s)\) would correspond to
a \textbf{Type II} (or ``Fast Focusing'') blow-up, where the scaling
parameter obeys \(\lambda(t) \ll \sqrt{T^*-t}\). In such a regime, the
viscous term in the renormalized equation (6.1) would vanish
asymptotically, \(\nu_{eff} \to 0\), potentially allowing the flow to
decouple from the centrifugal barrier via Lagrangian separation (the
formation of a ``hollow vortex'').

We resolve this by distinguishing between two dynamic regimes and
proving that the Type II regime is energetically forbidden for helical
geometries.

\begin{quote}
\textbf{Definition (The Viscous Coupling Hypothesis):} We restrict our
analysis to the class of ``Viscously Coupled'' singularities, defined as
profiles where the local Reynolds number
\(Re_{\lambda} = \lambda(t) u_{max}(t) / \nu\) remains uniformly
bounded. \emph{Note:} This excludes the ``Flying'' (Type II) regime
where \(Re_{\lambda} \to \infty\). In the Flying regime, the core
decouples from the viscous dissipation, rendering the spectral
coercivity barrier (which relies on viscous stress to enforce the
centrifugal effect) inoperative.
\end{quote}

\textbf{Definition 6.1.6 (Regimes of Viscous Coupling).} 1. \textbf{The
Viscous-Locked Regime (\(Re_{\lambda} \lesssim O(1)\)):} This
corresponds to Type I scaling (\(\lambda(t) \sim \sqrt{T^*-t}\)). In
this regime, the diffusive timescale is commensurate with the collapse
timescale. The elliptic character of the operator is preserved, and the
spectral coercivity barrier is strictly enforced by the estimates in
Theorem 6.3. 2. \textbf{The Inviscid-Decoupling Regime
(\(Re_{\lambda} \to \infty\)):} This corresponds to Type II scaling. In
this regime, advective transport dominates diffusion, potentially
allowing the core to become swirl-free before viscosity can homogenize
the angular momentum.

We now prove that the transition from the Viscous-Locked regime to the
Inviscid-Decoupling regime is obstructed by the global energy
constraint.

\textbf{Proposition 6.1.6 (Energetic Constraint on Extreme Type II
Divergence).} Let \(\mathbf{u}(x,t)\) be a finite-energy solution to the
Navier-Stokes equations. Under the hypothesis that the local geometry of
the singular set is helical (as required by the depletion and defocusing
constraints), no ``extreme'\,' Type II scaling of the form
\(\lambda(t) \sim (T^*-t)^\gamma\) with \(\gamma \ge 1\) is compatible
with the global energy bound. Consequently, the effective Reynolds
number \(Re_{\lambda}\) cannot diverge via such an extreme acceleration,
and the flow remains in the viscously coupled regime.

\textbf{Proof.} We utilize the global Leray energy inequality combined
with the normalization condition from Definition 6.1. For any weak
solution
\(\mathbf{u} \in L^\infty(0, T; L^2) \cap L^2(0, T; \dot{H}^1)\), the
total dissipation is bounded by the initial energy:
\[ \int_0^{T^*} \int_{\mathbb{R}^3} |\nabla \mathbf{u}(x,t)|^2 \, dx \, dt \le \frac{1}{2\nu} \|\mathbf{u}_0\|_{L^2}^2 =: E_0 < \infty \]

We express the dissipation rate in terms of the renormalized variables.
Under the dynamic rescaling \(x = \lambda(t)y + x_c(t)\), the enstrophy
transforms as:
\[ \int_{\mathbb{R}^3} |\nabla \mathbf{u}(x,t)|^2 \, dx = \frac{1}{\lambda(t)} \int_{\mathbb{R}^3} |\nabla_y \mathbf{V}(y,s)|^2 \, dy \]

\textbf{Crucial Step:} By the normalization condition in Definition 6.1,
the renormalized enstrophy is bounded from below:
\[ \int_{\mathbb{R}^3} |\nabla_y \mathbf{V}(y,s)|^2 \, dy \geq \|\nabla \mathbf{V}(\cdot, s)\|_{L^2(B_1)}^2 = 1 \]
for all \(s \in [s_0, \infty)\). This normalization rigorously prevents
the vanishing core scenario by construction.

Assume, for the sake of contradiction, that the singularity exhibits an
``extreme'\,' Type II acceleration, in the sense that
\(\lambda(t) \sim (T^*-t)^\gamma\) with \(\gamma \ge 1\).

The conversion to physical energy dissipation becomes:
\[ E_{diss}(T^*) = \int_0^{T^*} \frac{1}{\lambda(t)} \|\nabla_y \mathbf{V}(\cdot, s(t))\|_{L^2}^2 \, dt \geq \int_0^{T^*} \frac{1}{\lambda(t)} \, dt \]

For scaling with \(\lambda(t) \sim (T^*-t)^\gamma\):
\[ \int_0^{T^*} \frac{dt}{\lambda(t)} \sim \int_0^{T^*} \frac{dt}{(T^*-t)^{\gamma}} \]

This integral diverges to \(+\infty\) precisely when \(\gamma \geq 1\).
In particular, for any genuinely ``extreme'\,' Type II scaling
(\(\gamma \ge 1\)), we have: \[ E_{diss}(T^*) = +\infty \]

This contradicts the global finite energy constraint
\(E_{diss}(T^*) \leq E_0 < \infty\).

\textbf{Conclusion.} The formation of a ``hollow vortex'' via
sufficiently rapid (extreme) acceleration requires the expenditure of
infinite time-integrated enstrophy to overcome the swirl-induced
spectral barrier. Since the total energy is finite, the system cannot
access such an extreme Inviscid-Decoupling regime. The remaining
``mild'\,' Type II scalings with \(1/2 < \gamma < 1\) are ruled out by
the spectral and modulational stability analysis of Section 9 (in
particular Theorem 9.1 and Theorem 9.3); under those hypotheses the
scaling rate \(a(s)\) is forced to lock to the Type I value
\(a(s)\to 1\) as \(s\to\infty\). Therefore, the viscous penetration
condition is satisfied, the core remains hydrodynamically coupled to the
bulk, and the stability analysis of Theorem 6.3 holds without loss of
generality.

:::\{prf:lemma\} Swirl Positivity Near the Axis :label:
lem-swirl-positivity-near-the-axis

Let \(u(x,t)\) be a smooth, axisymmetric solution of the 3D
Navier--Stokes equations on \(\mathbb{R}^3\times[0,T)\) with finite
energy. Fix \(t_0\in(0,T)\) and radii \(0<r_1<r_2\). Suppose there
exists \(c_0>0\) such that \[
\omega_z(x,t_0) \ge c_0 \quad \text{for all } x \text{ with } r_1 \le r(x) \le r_2.
\] Then there exist numbers \(\rho\in(0,r_1)\), \(\Delta t>0\) and
\(c_1>0\) (depending only on \(c_0,r_1,r_2,\nu\) and local bounds on
\(u,\nabla u\)) such that \[
\omega_z(x,t) \ge c_1 \quad \text{for all } x \text{ with } r(x) \le \rho \text{ and all } t\in[t_0+\tfrac{\Delta t}{2},\,t_0+\Delta t].
\]

:::\{prf:remark\} Compatibility with Euler Blow-up :label:
rem-compatibility-with-euler-blow-up

It is crucial to observe why this obstruction vanishes in the inviscid
limit (\(\nu \to 0\)). The exclusion of Type II blow-up relies on the
dissipation capacity bound (Theorem 9.3), which imposes:

\[
\int_0^{T^*} E_{diss}(t) \, dt \approx \nu \int_0^{T^*} \frac{1}{\lambda(t)} \, dt < E_0.
\]

In the Navier-Stokes setting, \(\nu\) is fixed; thus a collapse rate of
\(\lambda(t) \sim (T^*-t)\) forces the integral to diverge. However, for
the Euler equations, the viscosity vanishes simultaneously with the
scale reduction. If the collapse occurs such that
\(\lambda(t) \sim \mathcal{O}(\nu)\), the product \(\nu \lambda^{-1}\)
remains bounded. Thus, Euler solutions can evade this capacity barrier,
whereas Navier-Stokes solutions cannot.

(sec-rigorous-derivation-harmonic-shielding-and-the-mul)= \#\#\# 6.2.
Rigorous Derivation: Harmonic Shielding and the Multipole Expansion

To establish the validity of the swirl-induced spectral coercivity, we
must control the non-local contributions to the pressure gradient. The
Navier-Stokes pressure is governed by the Poisson equation involving the
Riesz transform, a global singular integral operator. A potential
failure mode of the theory is that the ``Tidal Forces'' exerted by
distant vorticity (e.g., the tails of the helix or external filaments)
could exceed the local centrifugal barrier.

We resolve this by decomposing the pressure field using a
\textbf{Geometric Multipole Expansion}. We prove that within the
singular core, the non-local pressure field is not only harmonic but
consists principally of a uniform translation mode (absorbed by the
dynamic rescaling parameters \(x_c(t)\)) and a bounded straining mode,
both of which are asymptotically negligible compared to the
hyper-singular local rotation potential.

(sec-the-elliptic-decomposition)= \#\#\#\# 6.2.1. The Elliptic
Decomposition

Let \(B_1 \subset \mathbb{R}^3\) be the unit ball in the renormalized
frame \(y = (x-x_c(t))/\lambda(t)\). We define a smooth cutoff function
\(\chi \in C_c^\infty(\mathbb{R}^3)\) such that \(\chi(y) \equiv 1\) for
\(|y| \le 2\) and \(\chi(y) \equiv 0\) for \(|y| \ge 3\).

We decompose the source tensor
\(\mathbf{T} = \mathbf{V} \otimes \mathbf{V}\) into local and far-field
components:
\[ \mathbf{T}_{loc} = \chi \mathbf{V} \otimes \mathbf{V}, \quad \mathbf{T}_{far} = (1-\chi) \mathbf{V} \otimes \mathbf{V} \]
The pressure \(Q\) is similarly decomposed into
\(Q = Q_{loc} + Q_{far}\), satisfying:
\[ -\Delta Q_{loc} = \nabla \cdot (\nabla \cdot \mathbf{T}_{loc}), \quad -\Delta Q_{far} = \nabla \cdot (\nabla \cdot \mathbf{T}_{far}) \]

(sec-regularity-of-the-far-field-potential)= \#\#\#\# 6.2.2. Regularity
of the Far-Field Potential

The solution to the Poisson equation is given by the convolution with
the Newtonian kernel \(G(y) = \frac{1}{4\pi|y|}\).
\[ Q_{far}(y) = \int_{\mathbb{R}^3} \partial_{z_i} \partial_{z_j} G(y-z) T_{far, ij}(z) \, dz \]
Integration by parts places the derivatives on the kernel. Since
\(\text{supp}(\mathbf{T}_{far}) \subset B_2^c\), for any target point
\(y \in B_1\) and source point \(z \in \text{supp}(\mathbf{T}_{far})\),
we have \(|y-z| \ge 1\). The kernel
\(K_{ij}(y-z) = \partial_i \partial_j G(y-z)\) is \(C^\infty\) and
bounded in this domain. Standard elliptic estimates imply:
\[ |D^\alpha Q_{far}(y)| \le \int_{|z| \ge 2} |D^\alpha_y \nabla^2 G(y-z)| |\mathbf{V}(z)|^2 \, dz \]
Using the decay of the Green's function derivatives
\(|D^k G(\zeta)| \sim |\zeta|^{-(k+1)}\):
\[ |D^\alpha Q_{far}(y)| \le C_k \int_{|z| \ge 2} \frac{1}{|z|^{3+|\alpha|}} |\mathbf{V}(z)|^2 \, dz \]
Since \(\mathbf{V} \in L^2_\rho\) (the Gaussian weighted space derived
in Section 6.1), the velocity decays faster than any polynomial at
infinity. Thus, the integral converges absolutely.

:::\{prf:lemma\} Analyticity of the Far Field :label:
lem-analyticity-of-the-far-field

The far-field pressure \(Q_{far}\) is harmonic in the ball \(B_{2}\).
Specifically, for any multi-index \(\alpha\), there exists a constant
\(C_\alpha\) depending on the global energy
\(\|\mathbf{V}\|_{L^2_\rho}\) such that:
\[ \sup_{y \in B_1} |D^\alpha Q_{far}(y)| \le C_\alpha \|\mathbf{V}\|^2_{L^2_\rho(\mathbb{R}^3)} \]

(sec-the-multipole-expansion-and-tidal-forces)= \#\#\#\# 6.2.3. The
Multipole Expansion and Tidal Forces

We now explicitly characterize the structure of the non-local force near
the origin to compare it with the spectral/centrifugal barrier. We
Taylor expand the kernel \(K_{ij}(y-z)\) around \(y=0\):
\[ K_{ij}(y-z) = K_{ij}(-z) + y_k \partial_k K_{ij}(-z) + O(|y|^2) \]
Substituting this into the integral representation yields the
\textbf{Multipole Expansion of the Far-Field Pressure}:

\[ Q_{far}(y) = \underbrace{Q_{far}(0)}_{\text{Constant}} + \underbrace{\mathbf{g} \cdot y}_{\text{Linear Gradient}} + \underbrace{\frac{1}{2} y \cdot \mathbf{H} \cdot y}_{\text{Tidal Hessian}} + O(|y|^3) \]

where the coefficients are moments of the external vorticity
distribution: * \textbf{Background Gradient (\(\mathbf{g}\)):}
\(\mathbf{g} = \int_{B_2^c} \nabla (\nabla^2 G)(-z) : (\mathbf{V} \otimes \mathbf{V})(z) \, dz\)
* \textbf{Tidal Tensor (\(\mathbf{H}\)):}
\(\mathbf{H} = \int_{B_2^c} \nabla^2 (\nabla^2 G)(-z) : (\mathbf{V} \otimes \mathbf{V})(z) \, dz\)

:::\{prf:theorem\} The Sub-Criticality of Tidal Forces :label:
the-the-sub-criticality-of-tidal-forces

Inside the singular core (\(y \in B_1\)), the forces satisfy the
following hierarchy as \(r \to 0\):

\begin{enumerate}
\def\labelenumi{\arabic{enumi}.}
\item
  \textbf{The Drift Correction (Order \(r^0\)):} The constant gradient
  term \(\nabla ( \mathbf{g} \cdot y) = \mathbf{g}\) corresponds to a
  uniform acceleration of the fluid frame. In the Renormalized
  Navier-Stokes formulation, this term is \textbf{exactly absorbed} by
  the core drift parameter \(\mathbf{c}(s) = \dot{\xi}/\lambda\).
  \[ \mathbf{c}(s) \leftarrow \mathbf{c}(s) + \mathbf{g} \] Thus, the
  linear gradient of the far-field pressure does not deform the profile;
  it merely shifts the center of the coordinate system.
\item
  \textbf{The Tidal Strain (Order \(r^1\)):} The leading order
  deformation force comes from the Hessian:
  \(\nabla (\frac{1}{2} y \cdot \mathbf{H} \cdot y) = \mathbf{H} \cdot y\).
  This force scales linearly:
  \(|\mathbf{F}_{tidal}| \le \|\mathbf{H}\| r\). Crucially,
  \(\|\mathbf{H}\|\) is bounded by the global energy (Lemma 6.2) and
  does not depend on \(r\).
\item
  \textbf{The Spectral/Centrifugal Barrier (Order \(r^{-3}\)):} From
  Theorem 6.1, the conservation of circulation implies the local
  pressure gradient scales as:
  \[ \nabla Q_{loc} \sim \frac{\Gamma^2}{r^3} \mathbf{e}_r \]
\end{enumerate}

\textbf{Conclusion:} The ratio of the disturbing non-local force to the
stabilizing spectral/centrifugal force is:
\[ \mathcal{R}(r) = \frac{|\nabla Q_{far}^{eff}|}{|\nabla Q_{loc}|} \sim \frac{C r}{C' r^{-3}} \sim O(r^4) \]
This vanishes rapidly as \(r \to 0\). The ``Tidal Forces'' exerted by
the vortex tails are vanishingly small compact perturbations relative to
the singular potential well generated by the swirl.

(sec-control-of-the-kink-geometry-the-curvature-conditi)= \#\#\#\#
6.2.4. Control of the ``Kink'' Geometry (The Curvature Condition)

The validity of the multipole expansion relies on the assumption that
the ``Far Field'' is indeed geometrically separated from the core (i.e.,
the support of the external vorticity is in \(B_2^c\)). A potential
objection is the ``Re-entrant Kink,'' where the vortex tube bends
sharply and re-enters the local neighborhood \(B_1\).

We quantify this via the \textbf{Renormalized Curvature Radius}
\(R_\kappa(s)\). Let \(\Sigma(s)\) be the centerline of the vortex. We
define \(R_\kappa = \inf_{y \in \Sigma, y \neq 0} |y|\).

\begin{itemize}
\item
  \textbf{Case 1: The Shielded Regime (\(R_\kappa > 2\)).} The geometry
  is locally cylindrical/helical. The far-field vorticity is supported
  outside \(B_2\). The Multipole Expansion (Theorem 6.2) holds, and the
  spectral/centrifugal barrier dominates.
\item
  \textbf{Case 2: The Kink Regime (\(R_\kappa \le 2\)).} High-curvature
  segments intrude into the core. In this regime the far-field harmonic
  assumption fails, but the defocusing inequality from Section 4
  applies. For curvature \(\kappa \sim 1/R_\kappa\), Lemma 4.1 yields
  the lower bound
  \[ |\partial_z Q| \gtrsim \frac{\Gamma^2}{R_\kappa^2}. \] When
  \(R_\kappa \sim r\) this term scales as \(r^{-2}\) and enters
  \(\mathcal{D}(t)\) with a favorable sign. Hence any re-entrant
  intrusion forces \(\mathcal{D}(t)\) positive in the axial direction,
  preventing axial concentration before the centrifugal balance is
  affected.
\end{itemize}

(sec-spectral-compactness)= \#\#\#\# 6.2.5. Spectral Compactness

Finally, we treat the full linearized operator
\(\mathcal{L}_{total} = \mathcal{L}_{loc} + \mathcal{K}_{far}\), where
\(\mathcal{K}_{far} \mathbf{w} = \nabla ( \nabla^{-2} \nabla \cdot (\mathbf{w} \cdot \nabla \mathbf{V}_{far}) )\).
Since the kernel of \(\mathcal{K}_{far}\) is smooth in \(B_1\),
\(\mathcal{K}_{far}\) is a \textbf{Compact Operator} from
\(H^1_\rho(B_1)\) to \(L^2_\rho(B_1)\). By Weyl's Theorem on the
stability of the essential spectrum, the addition of a compact
perturbation does not alter the Fredholm index or the essential spectrum
of the dominant operator \(\mathcal{L}_{loc}\). The spectral gap proven
in Theorem 6.3 for the isolated profile persists under the addition of
global geometric noise.

(sec-the-non-local-bootstrap-exclusion-of-strain-driven)= \#\#\# 6.2.6.
The Non-Local Bootstrap: Exclusion of Strain-Driven Singularities

A fundamental objection to the local stability analysis (defocusing and
coercivity constraints) posits the existence of a remote forcing
configuration. In this scenario, a candidate singularity at \(x_0\) does
not generate its own blow-up via self-induction or rotation, but is
instead driven to collapse by a divergent strain field \(S_{ext}\)
generated by a remote vorticity distribution at \(x_{ext}\).

The objection suggests that while the target singularity might be
locally stable (swirl-free or subject to the spectral coercivity
barrier), it could be passively compressed by an external force that
bypasses the local barrier. We resolve this by proving that this remote
forcing scenario is dynamically forbidden by a recursive stability
principle.

:::\{prf:lemma\} The Propagation of Regularity :label:
lem-the-propagation-of-regularity

Let \(\Sigma^* \subset \mathbb{R}^3 \times \{T^*\}\) be the singular set
at the blow-up time. Assume a point \(x_0 \in \Sigma^*\) is driven to
singularity solely by an external strain field \(S_{ext}(x_0, t)\) such
that \(\|S_{ext}(t)\| \to \infty\) as \(t \to T^*\). From the
Biot-Savart law, the strain tensor is derived from the vorticity via a
singular integral kernel \(K(z) \sim |z|^{-3}\):
\[ S_{ext}(x_0) = \text{P.V.} \int_{\text{supp}(\omega_{ext})} K(x_0 - y) \boldsymbol{\omega}(y) \, dy \]
For this integral to diverge, one of two conditions must be met: 1.
\textbf{Infinite Vorticity Density:} The source vorticity
\(\|\boldsymbol{\omega}\|_{L^\infty}\) diverges. 2. \textbf{Geometric
Collapse:} The distance
\(d(t) = \text{dist}(x_0, \text{supp}(\omega_{ext}))\) vanishes, while
the circulation remains non-zero.

In either case, the ``Source'' \(x_{ext}\) must itself be a subset of
the singular set \(\Sigma^*\). A regular (smooth, bounded) vorticity
distribution at a finite distance cannot generate an infinite strain
field. :::

:::\{prf:theorem\} Recursive Geometric Stratification :label:
the-recursive-geometric-stratification

Since the Source \(x_{ext}\) is necessarily singular, it is subject to
the same geometric capacity constraints (the three-fold geometric
constraint system) established in Sections 3, 4, and 6. This leads to a
contradiction for all possible topologies of the Source:

\begin{enumerate}
\def\labelenumi{\arabic{enumi}.}
\item
  \textbf{Case 1: The Source is High-Entropy (Fractal/Cloud).} If the
  Source attempts to generate strain via a dense accumulation of
  filaments (a ``vortex tangle''), it falls into the domain of the
  \textbf{Geometric Depletion Inequality}. As proven in Section 3, the
  viscous smoothing timescale \(\tau_{visc} \sim k^{-2}\) dominates the
  strain generation timescale \(\tau_{strain} \sim k^{-1}\). The Source
  is dissipated before it can generate the critical strain required to
  crush the Victim.
\item
  \textbf{Case 2: The Source is Low-Entropy (Coherent Tube/Helix).} If
  the Source is a coherent filament focusing at \(x_{ext}\), it must
  possess a geometry compatible with the ``sieve.''

  \begin{itemize}
  \tightlist
  \item
    If the Source is \textbf{Straight/Poloidal}, it is dismantled by the
    axial defocusing condition. The axial pressure gradient ejects mass
    from the Source, preventing the accumulation of circulation required
    to maintain the strain field.
  \item
    If the Source is \textbf{Helical/Swirling}, it is stabilized by the
    spectral coercivity barrier. The centrifugal barrier arrests the
    radial collapse of the Source.
  \end{itemize}
\end{enumerate}

\textbf{The Interaction Contradiction:} For the remote source to drive
the target singularity, it must generate infinite strain. To generate
infinite strain, the source itself must collapse. But the spectral
coercivity barrier (Theorem 6.3) proves that the source cannot collapse.
Therefore, the strain field \(S_{ext}\) exerted on the target
singularity remains uniformly bounded by the constraint on the remote
source. \[ \sup_{t < T^*} \|S_{ext}(x_0, t)\| \le C_{max} < \infty \]
Consequently, the target point \(x_0\) is subjected only to finite
deformation forces, which are insufficient to overcome its own viscous
resistance.

\textbf{Conclusion:} Conservation laws enforce a fundamental constraint:
to generate a singular force, a structure must itself become singular.
Since we have established that intrinsic singularities are geometrically
forbidden, extrinsic (strain-driven) singularities are recursively
forbidden. The stability of the system is global.

(sec-the-spectral-gap-dominance-of-the-centrifugal-pote)= \#\#\# 6.3.
The Spectral Gap: Dominance of the Centrifugal Potential

Having established the decomposition of the pressure field, we now
analyze the spectral properties of the linearized operator around the
helical ansatz. The formation of a finite-time singularity requires the
existence of a ``focusing mode''---an eigenfunction with a negative
eigenvalue that drives the contraction of the core.

We prove that if the swirl ratio \(\mathcal{S}\) is sufficiently large,
the centrifugal barrier eliminates these focusing modes, enforcing a
spectral gap that forbids radial collapse.

We examine the energy identity for the perturbation \(\mathbf{w}\).
Multiplying the linearized equation by \(\mathbf{w}\rho\) and
integrating by parts yields:
\[ \frac{1}{2} \frac{d}{ds} \|\mathbf{w}\|^2_\rho = \underbrace{-\|\nabla \mathbf{w}\|^2_\rho + \frac{1}{2} \|\mathbf{w}\|^2_\rho}_{\text{Heat Operator Spectrum}} - \underbrace{\int (\mathbf{w} \cdot \nabla \mathbf{V}_\sigma) \cdot \mathbf{w} \rho \, dy}_{\text{Stretching Term}} - \underbrace{\int (\nabla q) \cdot \mathbf{w} \rho \, dy}_{\text{Pressure Term}} \]

The key insight is the differential scaling of these terms with the
swirl parameter \(\sigma\):

\textbf{Scaling Analysis:} 1. \textbf{Vortex Stretching:} The velocity
gradient scales linearly with swirl:
\[ \|\nabla \mathbf{V}_\sigma\|_{stretch} \sim O(\sigma) \] since
\(\partial_r(\sigma u_\theta) = \sigma \partial_r u_\theta\).

\begin{enumerate}
\def\labelenumi{\arabic{enumi}.}
\setcounter{enumi}{1}
\item
  \textbf{Pressure Hessian:} The centrifugal pressure scales
  quadratically:
  \[ \nabla^2 Q \sim \nabla^2\left(\frac{(\sigma u_\theta)^2}{r}\right) \sim O(\sigma^2) \]
  as the centrifugal potential \(Q_{cent} \sim \sigma^2 u_\theta^2/r\).
\item
  \textbf{Dominance for Large Swirl:} Since \(\sigma^2 \gg \sigma\) for
  \(\sigma > \sigma_c\), the stabilizing pressure term dominates the
  destabilizing stretching term.
\end{enumerate}

We now establish these bounds rigorously:

\begin{enumerate}
\def\labelenumi{\arabic{enumi}.}
\item
  \textbf{The Stretching bound:} The stretching term is bounded by the
  maximal strain of the background profile:
  \[ \left| \int (\mathbf{w} \cdot \nabla \mathbf{V}) \cdot \mathbf{w} \rho \right| \leq \|\nabla \mathbf{V}\|_{L^\infty} \|\mathbf{w}\|^2_\rho \]
  In a standard Type I blow-up, \(\|\nabla \mathbf{V}\|_{L^\infty}\) is
  bounded. However, for the singularity to occur, the stretching must be
  ``attractive'' (negative definite contribution to the energy).
\item
  \textbf{The Pressure Hessian as a Potential:} Using the decomposition
  from Lemma 6.2, we isolate the dominant cylindrical part of the
  pressure gradient \(\nabla Q_{cyl}\). For a radial perturbation
  \(w_r\), the pressure term behaves like a potential:
  \[ -\int (\nabla Q) \cdot \mathbf{w} \rho \approx -\int (\partial_r Q_{cyl}) w_r \rho \]
  From Lemma 6.2, \(\partial_r Q_{cyl} \approx \frac{V_\theta^2}{r}\).
  Linearizing this term around the profile yields a \textbf{positive
  potential}:
  \[ \mathcal{H}_{pressure} \approx \int_{\mathbb{R}^3} \frac{2 \Gamma^2}{r^4} |w_r|^2 \rho \, dy \]
  This is the \textbf{Hardy Potential}. Crucially, it scales as
  \(r^{-4}\) (due to the gradient of the centrifugal force), whereas the
  stretching term scales as \(r^{-2}\) (vorticity scaling).
\item
  \textbf{The Spectral Gap Estimate:} We combine the terms. The
  effective potential \(W(y)\) acting on the radial perturbation is:
  \[ W(y) \approx \underbrace{-\|\nabla \mathbf{V}\|}_{\text{Inertial Attraction}} + \underbrace{\frac{C \mathcal{S}^2}{r^2}}_{\text{Centrifugal Repulsion}} \]
  (Note: The scaling \(1/r^2\) arises from the Hardy inequality applied
  to the pressure Hessian).

  By the Hardy-Rellich inequality, if the coefficient of the repulsive
  term (controlled by the swirl ratio \(\mathcal{S}\)) is sufficiently
  large, the positive potential dominates the negative inertial term
  globally. Specifically, if \(\mathcal{S} > \sqrt{2}\) (the Benjamin
  criterion {[}@benjamin1962{]}), the operator
  \(\mathcal{L}_{\mathbf{V}}\) becomes strictly dissipative (negative
  definite).
\end{enumerate}

\textbf{Conclusion:} Since \(\frac{d}{ds} \|\mathbf{w}\|^2 < 0\), any
perturbation decays. This contradicts the assumption that \(\mathbf{V}\)
is a blow-up profile, which by definition must possess an unstable
manifold (to allow the solution to escape the regular set) or a neutral
mode (scaling invariance). The coercive barrier prevents the flow from
accessing the singular scaling.

:::\{prf:theorem\} Swirl-Dominated Accretivity :label:
the-swirl-dominated-accretivity

Let \(\mathcal{L}_\sigma\) be the linearized operator governing
perturbations \(\mathbf{w}\) around the swirl-parameterized profile
\(\mathbf{V}_\sigma\) in the weighted space \(L^2_\rho(\mathbb{R}^3)\)
with Gaussian weight \(\rho(y) = e^{-|y|^2/4}\). Provided the profile
remains within the Viscously Coupled regime (\(Re_\lambda < \infty\)),
there exists a critical swirl threshold \(\sigma_c > 0\) such that for
all \(\sigma > \sigma_c\) (equivalently, swirl ratio
\(\mathcal{S} = \inf_{core} |\sigma u_\theta|/|u_z| > \sqrt{2}\)), the
operator \(\mathcal{L}_\sigma\) is strictly accretive. Specifically, the
symmetric part satisfies:
\[ \langle \mathcal{H}_\sigma \mathbf{w}, \mathbf{w} \rangle_{L^2_\rho} \leq - \mu \|\mathbf{w}\|_{L^2_\rho}^2 \]
for some \(\mu > 0\) independent of time. This establishes a uniform
spectral gap that forbids unstable (growing) modes and prevents the
self-similar collapse scaling \(\lambda(t) \to 0\).

The numerical range is defined as:
\[ \mathcal{W}(\mathcal{L}_\sigma) = \left\{\frac{\langle \mathcal{L}_\sigma \mathbf{w}, \mathbf{w} \rangle_{L^2_\rho}}{\|\mathbf{w}\|_{L^2_\rho}^2} : \mathbf{w} \neq 0\right\} \]

By Theorem 6.3, for all \(\sigma > \sigma_c\):
\[ \operatorname{Re}\left(\frac{\langle \mathcal{L}_\sigma \mathbf{w}, \mathbf{w} \rangle_{L^2_\rho}}{\|\mathbf{w}\|_{L^2_\rho}^2}\right) = \frac{\langle \mathcal{H}_\sigma \mathbf{w}, \mathbf{w} \rangle_{L^2_\rho}}{\|\mathbf{w}\|_{L^2_\rho}^2} \leq -\mu \]

Therefore,
\(\mathcal{W}(\mathcal{L}_\sigma) \subset \{z : \operatorname{Re}(z) \leq -\mu\}\).

For the resolvent bound, consider \(\lambda = i\xi\) with
\(\xi \in \mathbb{R}\). For any \(\mathbf{f} \in L^2_\rho\), let
\(\mathbf{w}\) solve
\((i\xi I - \mathcal{L}_\sigma)\mathbf{w} = \mathbf{f}\). Taking the
inner product with \(\mathbf{w}\):
\[ i\xi\|\mathbf{w}\|_{L^2_\rho}^2 - \langle \mathcal{L}_\sigma \mathbf{w}, \mathbf{w} \rangle_{L^2_\rho} = \langle \mathbf{f}, \mathbf{w} \rangle_{L^2_\rho} \]

Taking the real part and using the accretivity:
\[ -\operatorname{Re}\langle \mathcal{L}_\sigma \mathbf{w}, \mathbf{w} \rangle_{L^2_\rho} \geq \mu\|\mathbf{w}\|_{L^2_\rho}^2 = \operatorname{Re}\langle \mathbf{f}, \mathbf{w} \rangle_{L^2_\rho} \]

By Cauchy-Schwarz:
\[ \mu\|\mathbf{w}\|_{L^2_\rho}^2 \leq |\langle \mathbf{f}, \mathbf{w} \rangle_{L^2_\rho}| \leq \|\mathbf{f}\|_{L^2_\rho}\|\mathbf{w}\|_{L^2_\rho} \]

Therefore,
\(\|\mathbf{w}\|_{L^2_\rho} \leq \frac{1}{\mu}\|\mathbf{f}\|_{L^2_\rho}\),
establishing the resolvent bound.

For the pseudospectrum, recall that:
\[ \sigma_\epsilon(\mathcal{L}_\sigma) = \{z \in \mathbb{C} : \|(zI - \mathcal{L}_\sigma)^{-1}\| > \epsilon^{-1}\} \]

Since the resolvent norm is bounded by \(1/\mu\) for all \(z\) with
\(\operatorname{Re}(z) \geq 0\), we have
\(\sigma_\epsilon(\mathcal{L}_\sigma) \cap \{z : \operatorname{Re}(z) > 0\} = \emptyset\)
for \(\epsilon < \mu\).

:::\{prf:theorem\} Uniform Resolvent and Pseudospectral Bound :label:
the-uniform-resolvent-and-pseudospectral-bound

For \(\sigma > \sigma_c\), the numerical range
\(\mathcal{W}(\mathcal{L}_\sigma)\) is strictly contained in the left
half-plane \(\{z \in \mathbb{C} : \operatorname{Re}(z) \leq -\mu\}\).
Consequently, the resolvent admits the uniform bound:
\[ \sup_{\xi \in \mathbb{R}} \|(i\xi I - \mathcal{L}_\sigma)^{-1}\|_{L^2_\rho \to L^2_\rho} \leq \frac{1}{\mu} \]
Furthermore, the \(\epsilon\)-pseudospectrum cannot protrude into the
right half-plane for any \(\epsilon < \mu\).

By the Lumer-Phillips theorem, since \(\mathcal{L}_\sigma\) is accretive
with numerical range contained in
\(\{z : \operatorname{Re}(z) \leq -\mu\}\), it generates a contraction
semigroup. The spectral bound theorem gives:
\[ \|e^{t\mathcal{L}_\sigma}\| \leq e^{-\mu t} \]

Since this bound holds for all \(t \geq 0\), there is no initial
transient growth period. The energy
\(E(t) = \|\mathbf{w}(t)\|_{L^2_\rho}^2\) satisfies:
\[ \frac{dE}{dt} = 2\langle \mathcal{L}_\sigma \mathbf{w}, \mathbf{w} \rangle_{L^2_\rho} \leq -2\mu E(t) \]

Therefore, \(E(t) \leq E(0)e^{-2\mu t}\), establishing monotonic decay.
This excludes: - \textbf{Breathers:} Would require periodic energy
oscillation - \textbf{Transient growth:} Would require \(E(t) > E(0)\)
for some \(t > 0\) - \textbf{Shape-shifters:} Would require
non-monotonic evolution

The strict monotonicity enforces convergence to the trivial equilibrium.

:::\{prf:corollary\} Strong Semigroup Contraction :label:
cor-strong-semigroup-contraction

The semigroup generated by the linearized operator is a strict
contraction for all \(t > 0\):
\[ \|e^{t\mathcal{L}_\sigma}\|_{L^2_\rho \to L^2_\rho} \leq e^{-\mu t} \]
Consequently, perturbations decay monotonically from \(t = 0\),
precluding transient growth, breathers, and shape-shifting dynamics.

\begin{center}\rule{0.5\linewidth}{0.5pt}\end{center}

(sec-geometric-covering-of-the-weak-swirl-regime)= \#\#\# 6.3.1.
Geometric Covering of the Weak Swirl Regime

The spectral analysis in Theorem 6.3 establishes the stability of the
blow-up profile under the condition of High Swirl
(\(\mathcal{S} > \sqrt{2}\)), where the centrifugal barrier provides a
global coercive estimate. This leaves the interval of \textbf{Weak
Swirl} (\(0 \le \mathcal{S} \le \sqrt{2}\)) to be addressed. In this
regime, the centrifugal potential is insufficient to generate a global
spectral gap.

To resolve this, we analyze the local geometry of the pressure field. We
prove that in the absence of a dominant centrifugal barrier, the
topological concentration of the flow induces a \textbf{Stagnation
Pressure Ridge} that destabilizes the core. We decompose the local
geometry of the singular set into three canonical configurations and
prove that each is subject to a repulsive gradient that prohibits
collapse. :::

We examine the Poisson equation for the renormalized pressure \(Q\)
restricted to the symmetry axis (\(r=0\)). In cylindrical coordinates
\((r, \theta, z)\), the Laplacian is given by:
\[ -\Delta Q = \text{Tr}(\nabla \mathbf{V} \otimes \nabla \mathbf{V}) \]
Decomposing the source term into strain and vorticity components:
\[ -\Delta Q = \|\mathbf{S}\|^2 - \frac{1}{2} \|\boldsymbol{\Omega}\|^2 \]
where \(\mathbf{S}\) is the rate-of-strain tensor and
\(\boldsymbol{\Omega}\) is the vorticity. On the axis of a focusing
singularity, continuity \(\nabla \cdot \mathbf{V} = 0\) implies that the
axial extension \(\partial_z V_z\) must balance the radial compression.
Consequently, the squared strain terms are strictly positive and scale
with the rate of collapse. In the Weak Swirl regime, the vorticity
magnitude \(\|\boldsymbol{\Omega}\|^2\) is sub-dominant to the strain
magnitude. Thus, we obtain the inequality: \[ -\Delta Q > 0 \] By the
Maximum Principle for sub-harmonic functions, \(Q\) achieves a local
maximum at the centroid of the collapse (where the strain is maximized).
Let \(z=0\) denote the point of minimum radius (the ``neck'' of the
singular tube). It follows that:
\[ \partial_z Q(0) = 0, \quad \partial_{zz} Q(0) < 0 \] This implies
that for \(z \neq 0\), the pressure gradient force \(-\partial_z Q\)
satisfies: \[ \text{sgn}(-\partial_z Q) = \text{sgn}(z) \] This force
acts as an inertial pump, accelerating fluid parcels axially away from
the singular point \(z=0\). This ``Stagnation Ridge'' prevents the
accumulation of mass required to sustain the singularity, forcing the
core to eject mass axially faster than it concentrates radially.

:::\{prf:lemma\} The Axial Ejection Principle :label:
lem-the-axial-ejection-principle

Assume the renormalized flow profile \(\mathbf{V}(y)\) is locally
axisymmetric and focusing (i.e., \(V_r < 0\)) within the core \(r < 1\).
If the Swirl Ratio satisfies \(\mathcal{S} \le \sqrt{2}\), then the
pressure field \(Q\) exhibits a local maximum on the axis of symmetry,
generating an axial gradient directed outward from the point of maximum
collapse.

:::

We project the Navier-Stokes momentum equation onto the Frenet-Serret
normal vector \(\mathbf{n}\) of the vortex line. In the core of the
filament, the primary force balance in the normal direction is between
the pressure gradient and the centrifugal force induced by the curvature
of the streamlines along the filament trajectory. Let \(V_{\parallel}\)
denote the velocity component tangential to the filament. The transverse
pressure gradient scales as:
\[ \nabla_{\mathbf{n}} Q \approx \frac{V_{\parallel}^2}{R_{\kappa}} = \kappa V_{\parallel}^2 \]
For a candidate singularity, the renormalization condition implies that
the core velocity \(V_{\parallel}\) must diverge as \(y \to 0\).
Consequently, the transverse pressure gradient \(\nabla_{\mathbf{n}} Q\)
becomes singular. This force is directed outward from the center of
curvature. Physically, this manifests as a ``stiffening'' force that
opposes the bending of the vortex tube. As \(R_{\kappa} \to 0\) (forming
a ``kink''), the repulsive force approaches infinity, dynamically
forbidding the geometry from folding onto itself. Thus, the singular set
must remain locally rectilinear, ensuring the applicability of Lemma
6.3.1.

:::\{prf:lemma\} The Transverse Unfolding Principle :label:
lem-the-transverse-unfolding-principle

Assume the vortex filament possesses a non-zero radius of curvature
\(R_{\kappa} < \infty\). Then, the pressure gradient contains a
transverse component that drives the filament to reduce its curvature,
preventing the formation of complex ``knotted'' singularities.

:::

We employ a Multipole Expansion of the external pressure field
\(Q_{ext}\) generated by the far-field vorticity. Expanding the
Biot-Savart kernel around the core center \(y=0\):
\[ \nabla Q_{ext}(y) \approx \mathbf{C}(s) + \mathbf{S}_{tidal} \cdot y + O(|y|^2) \]
1. \textbf{Zero-Order Mode (Translation):} The constant term
\(\mathbf{C}(s)\) corresponds to a uniform pressure gradient. In the
Dynamic Rescaling Framework (Section 6.1), this term is exactly absorbed
by the core drift parameter \(\dot{\xi}(t)\). It results in the
translation of the singularity, not its deformation. 2.
\textbf{First-Order Mode (Tidal Strain):} The leading-order deformation
force is the linear strain
\(\mathbf{F}_{tidal} = \mathbf{S}_{tidal} \cdot y\). Crucially, this
force scales linearly with the distance \(r\) from the axis:
\(|\mathbf{F}_{tidal}| \sim O(r)\). 3. \textbf{The Local Dominance:} By
Lemma 6.3.1, the self-generated ejection force arises from the gradient
of the stagnation potential, which scales as \(V^2 \sim r^{-2}\)
(Bernoulli scaling). Thus, the ejection force scales as:
\[ |\mathbf{F}_{local}| = |-\nabla Q_{local}| \sim \partial_r(r^{-2}) \sim O(r^{-3}) \]

Comparing the magnitudes as the singularity approaches (\(r \to 0\)):
\[ \lim_{r \to 0} \frac{|\mathbf{F}_{tidal}|}{|\mathbf{F}_{local}|} \sim \lim_{r \to 0} \frac{C_{ext} r}{C_{int} r^{-3}} = \lim_{r \to 0} C r^4 = 0 \]
This establishes a \textbf{Screening Effect}: the singular core is
asymptotically decoupled from the far-field environment. The divergence
of the local forces ensures that the stability of the core is determined
exclusively by its intrinsic geometry, rendering the strain-driven
scenario dynamically impossible.

:::\{prf:lemma\} Asymptotic Screening of Tidal Fields :label:
lem-asymptotic-screening-of-tidal-fields

Assume the singular core is acted upon by a non-local ``background''
strain field \(\mathbf{S}_{ext}\) generated by a vorticity distribution
supported at a distance \(d \gg 1\) in the renormalized frame. We prove
that the local ejection forces (Lemmas 6.3.1 and 6.3.2) asymptotically
dominate the non-local compression forces.

\begin{center}\rule{0.5\linewidth}{0.5pt}\end{center}

(sec-the-exclusion-of-resonant-geometric-interference)= \#\#\# 6.4. The
Exclusion of Resonant Geometric Interference

We have established that high-frequency geometric oscillations
(\(k \to \infty\)) are smoothed by the depletion inequality, while
low-frequency deformations (\(k \to 0\)) are destabilized by the
defocusing condition. This leaves a potential interval of
\textbf{Geometric Resonance}, where the deformation wavelength
\(\lambda\) is commensurate with the core radius \(r(t)\) (i.e.,
\(k r \sim O(1)\)).

In this regime, a ``Varicose'' (axisymmetric ripple) perturbation could
theoretically induce a pressure interference pattern that counteracts
the base ejection gradient. We prove that such a configuration is
forbidden by a scaling mismatch between the pressure cross-term and the
viscous dissipation. :::

We analyze the scaling of the three force components in the renormalized
frame:

\begin{enumerate}
\def\labelenumi{\arabic{enumi}.}
\item
  \textbf{The Base Ejection Force (\(F_{base}\)):} From Lemma 6.3.1, the
  unperturbed focusing generates a stagnation pressure gradient scaling
  with the inertial energy density:
  \[ F_{base} \sim \|\nabla \mathbf{V}_{base}\|^2 \sim C_0 \quad (\text{Normalized to } O(1)) \]
\item
  \textbf{The Interference Force (\(F_{int}\)):} The pressure correction
  \(Q_{cross}\) arises from the cross-terms in the Poisson source
  \(\nabla \mathbf{V} : \nabla \mathbf{V}\). For a perturbation of
  amplitude \(\epsilon\), the interaction between the base flow and the
  perturbation is linear in \(\epsilon\):
  \[ F_{int} \approx -\partial_z Q_{cross} \le C_1 \epsilon \] This
  force represents the potential ``suction'' created by the ripple.
\item
  \textbf{The Viscous Penalty (\(F_{visc}\)):} The viscous dissipation
  term in the energy equation scales with the Dirichlet energy of the
  perturbation. Since the deformation increases the surface area and
  shear gradients of the tube, the dissipative cost scales quadratically
  with the amplitude:
  \[ \mathcal{D}_{pert} \sim \nu \int |\nabla (\epsilon \mathbf{V}_{pert})|^2 \sim C_2 \epsilon^2 \]
  In the context of the momentum balance, this manifests as a damping
  force proportional to \(\epsilon^2\) (accounting for the nonlinearity
  of the shape deformation acting on the stress tensor).
\end{enumerate}

\textbf{The Non-Existence Argument:} To stabilize the core against the
axial defocusing condition, the interference must satisfy
\(F_{int} \approx F_{base}\). This imposes a lower bound on the
amplitude:
\[ C_1 \epsilon \ge C_0 \implies \epsilon \ge \frac{C_0}{C_1} \sim O(1) \]
The ripple must be large (comparable to the core radius) to reverse the
strong stagnation gradient. However, substituting this amplitude into
the viscous penalty reveals a dominance of dissipation:
\[ \frac{\text{Viscous Damping}}{\text{Inertial Interference}} \sim \frac{C_2 \epsilon^2}{C_1 \epsilon} \sim \frac{C_2}{C_1} \epsilon \]
For \(\epsilon \sim O(1)\), the quadratic viscous term dominates the
linear pressure term. Consequently, any ripple large enough to stop the
ejection generates sufficient turbulent dissipation to trigger the
geometric depletion inequality. The flow exits the inertial regime and
enters the viscous-dominated regime, where the singularity decays.

:::\{prf:lemma\} The Viscous-Inertial Amplitude Barrier :label:
lem-the-viscous-inertial-amplitude-barrier

Let the boundary of the singular core be modulated by a resonant
perturbation \(\delta(z) = \epsilon r(t) \sin(kz)\), where \(\epsilon\)
is the dimensionless amplitude and \(k \sim 1/r\). We define the
\textbf{Stability Functional} \(\mathcal{F}(\epsilon)\) representing the
net axial force density. For the singularity to persist, the
interference force must cancel the base ejection force:
\[ \mathcal{F}(\epsilon) = F_{base} - F_{int}(\epsilon) + F_{visc}(\epsilon) \approx 0 \]
We prove that no solution exists for \(\mathcal{F}(\epsilon) = 0\) in
the singular limit due to the quadratic scaling of the viscous penalty.

\begin{center}\rule{0.5\linewidth}{0.5pt}\end{center}

(sec-theorem-65-stratification-of-the-singular-set)= \#\#\# 6.5 Theorem
6.5: Stratification of the Singular Set

We rule out ``exotic'' singularities (e.g., quasi-periodic pulses,
chaotic dust) without assuming a priori symmetries, utilizing the
Dimension Reduction principle inherent to the partial regularity theory.

:::\{prf:theorem\} Classification of Singular Strata :label:
the-classification-of-singular-strata

Let \(\Sigma\) be the singular set in spacetime. Based on the dimension
of the tangent flow measures, \(\Sigma\) admits a decomposition into
three disjoint strata:
\(\Sigma = \Sigma_{dense} \cup \Sigma_{cyl} \cup \Sigma_{point}\). *
\textbf{The Dense Stratum (\(\Sigma_{dense}\)):} Points where the
parabolic Hausdorff dimension \(\dim_{\mathcal{P}} > 1\). *
\textbf{Resolution:} This stratum is empty by the
Caffarelli-Kohn-Nirenberg (CKN) theorem (\(\mathcal{H}^1(\Sigma)=0\)).
Even in hyper-weak solutions, this regime is ruled out by the geometric
depletion inequality. * \textbf{The Cylindrical Stratum
(\(\Sigma_{cyl}\)):} Points where \(\dim_{\mathcal{P}} \le 1\) and the
tangent flow \(\bar{\mathbf{u}}\) is translationally invariant in at
least one spatial direction. * \textbf{Resolution:} The flow reduces to
2D or 2.5D dynamics. * If swirl-free, it is regular by classical 2D
theory. * If low-swirl, it is destabilized by the axial defocusing
condition. * If high-swirl, it is stabilized by the spectral coercivity
estimate. * \textbf{The Isolated Stratum (\(\Sigma_{point}\)):} Points
where \(\dim_{\mathcal{P}} = 0\). These are isolated spacetime points
where the tangent flow lacks translational invariance. *
\textbf{Resolution:} Isolated singularities must follow a self-similar
scaling profile \(\mathbf{V}\). We apply the Liouville Theorem (Theorem
6.4), which proves that no non-trivial smooth profile \(\mathbf{V}\)
exists in the high-swirl regime. If the profile is non-smooth, it falls
into the ``Pathological'' category (see Section 8).

\emph{Conclusion.} Since dynamic obstructions exist for all three
geometric strata, the set of classical singular times is empty.
\(\hfill \square\)

(sec-exclusion-of-the-anisotropic-ribbon-the-aspect-rat)= \#\#\#\#
6.5.1. Exclusion of the Anisotropic Ribbon (The Aspect Ratio Barrier)

A specific objection to the stratification in Theorem 6.5 is the
existence of the ``Ribbon'' or ``Pancake'' singularity: an anisotropic
structure where the support \(\Sigma\) collapses in one dimension
(\(L_1 \to 0\)) while remaining macroscopic in others (\(L_2 \gg L_1\)).
This geometry attempts to evade the spectral coercivity barrier by
lacking a defined swirl axis, and to evade the defocusing constraint by
lacking a deep pressure well.

We exclude this configuration by proving a \textbf{Topological
Dichotomy}: the Ribbon is either sufficiently flat to trigger
\textbf{Geometric Depletion}, or sufficiently curved to trigger
\textbf{Kelvin-Helmholtz Roll-up} (returning it to the Cylindrical
Stratum).

:::\{prf:definition\} The Aspect Ratio Functional :label:
def-the-aspect-ratio-functional

Let \(\lambda_1 \le \lambda_2 \le \lambda_3\) be the eigenvalues of the
inertia tensor of the localized vorticity distribution. We define the
Aspect Ratio \(\mathcal{A}(t) = \sqrt{\lambda_3 / \lambda_1}\). *
\textbf{Ribbon Regime:} \(\mathcal{A}(t) \to \infty\) (Collapse to a
sheet). * \textbf{Tube Regime:} \(\mathcal{A}(t) \sim 1\) (Collapse to a
filament). :::

:::\{prf:lemma\} The Anisotropy-Dissipation Inequality :label:
lem-the-anisotropy-dissipation-inequality

Consider a Ribbon profile with characteristic thickness \(h(t)\) and
width \(W(t)\), such that \(\mathcal{A} \approx W/h \gg 1\). The
competition between vortex stretching and dissipation scales
anisotropically: 1. \textbf{Stretching (In-Plane):} The stretching is
dominated by the macroscopic shear, scaling as
\(T_{stretch} \sim \Gamma^2 / W^2\). 2. \textbf{Dissipation
(Cross-Plane):} The dissipation is dominated by the gradient across the
thin layer, scaling as \(T_{diss} \sim \nu \Gamma / h^3\).

The ratio of dissipation to stretching behaves as:
\[ \frac{T_{diss}}{T_{stretch}} \sim \nu \frac{W^2}{h^3 \Gamma} = \frac{\nu}{\Gamma} \mathcal{A}^2 \frac{1}{h} \]
As \(h \to 0\), this ratio diverges unless \(\mathcal{A}\) decreases.
This proves that \textbf{Infinite Aspect Ratio collapse is viscously
forbidden}. The sheet dissipates faster than it stretches. :::

:::\{prf:theorem\} The Topological Switch :label:
the-the-topological-switch

A singular set must settle into a geometry. The Ribbon configuration is
dynamically unstable to the \textbf{Constantin-Fefferman (CF)
Criterion}: 1. \textbf{Case 1: The Flat Limit
(\(\nabla \boldsymbol{\xi} \approx 0\)).} If the ribbon remains flat to
avoid viscous dissipation, the direction of vorticity
\(\boldsymbol{\xi} = \boldsymbol{\omega}/|\boldsymbol{\omega}|\) becomes
spatially uniform. By the results of Constantin \& Fefferman
{[}@constantin1993{]}, the nonlinearity is depleted:
\[ \int (\boldsymbol{\omega} \cdot \nabla) \mathbf{u} \cdot \boldsymbol{\omega} \, dx \le C \|\nabla \boldsymbol{\xi}\|_{L^\infty} \|\boldsymbol{\omega}\|_{L^2}^2 \]
Smooth direction fields prevent blow-up. 2. \textbf{Case 2: The Rolling
Limit (Kelvin-Helmholtz).} If the ribbon develops curvature
(\(\nabla \boldsymbol{\xi} \neq 0\)) to maximize stretching, it triggers
the Kelvin-Helmholtz instability. The sheet rolls up on a timescale
\(\tau_{KH} \sim \|\boldsymbol{\omega}\|^{-1}\). This topological
transition converts the \textbf{Ribbon} (Codimension 1) into a
\textbf{Tube} (Codimension 2) or a stack of tubes. Once the topology
becomes tubular (\(\mathcal{A} \to 1\)), the geometry enters the domain
where the spectral coercivity barrier applies and is stabilized.

\textbf{Conclusion:} The ``Ribbon'' is a transient state, not a blow-up
profile. It cannot blow up while flat (due to CF Depletion and
Anisotropic Dissipation), and it cannot blow up after rolling up
(because the centrifugal coercivity barrier reappears). The intersection
of the failure sets for Sheets and Tubes is empty. \(\hfill \square\)

(sec-the-asymptotic-dominance-of-transverse-ejection)= \#\#\# 6.5.2. The
Asymptotic Dominance of Transverse Ejection

The final topological obstruction to global regularity is the
\textbf{Symmetric Interaction}, specifically the anti-parallel collision
of vortex filaments or the self-similar collapse of a non-circular
vortex ring. In this configuration, the symmetry of the domain
(\(\Sigma_{sym} = \{z=0\}\)) enforces \(u_z = 0\) and \(u_\theta = 0\),
effectively disabling both the axial defocusing condition and the
spectral coercivity (swirl-induced) constraint on the symmetry plane.

We prove, however, that this configuration is dynamically unstable to
transverse geometric deformation. The collision interface generates a
transverse stagnation pressure gradient that forces a topological
transition from tube (codimension 2) to sheet (codimension 1) prior to
the singular time.

:::\{prf:lemma\} The Transverse Pressure Barrier :label:
lem-the-transverse-pressure-barrier

Consider two vortex cores with circulation \(\pm \Gamma\) separated by a
distance \(d(t)\). We analyze the competition between the
\textbf{Inertial Attraction} (driving the singularity) and the
\textbf{Pressure Repulsion} (driving the geometric deformation).

\begin{enumerate}
\def\labelenumi{\arabic{enumi}.}
\item
  \textbf{The Attraction Scaling (\(F_{in}\)):} The mutual induction
  velocity driving the cores together is governed by the Biot-Savart
  law, scaling as \(v_{approach} \sim \Gamma/d(t)\). The inertial force
  density pulling the cores into the collision is therefore:
  \[ F_{in} \sim \mathbf{u} \cdot \nabla \mathbf{u} \sim \frac{\Gamma^2}{d(t)} \]
\item
  \textbf{The Repulsion Scaling (\(F_{out}\)):} The stagnation pressure
  \(Q\) at the symmetry plane scales as the square of the approach
  velocity (Bernoulli scaling):
  \(Q_{max} \sim v_{approach}^2 \sim \Gamma^2 / d(t)^2\). This pressure
  creates a transverse gradient \(\nabla_\perp Q\) driving fluid outward
  along the symmetry plane (orthogonal to the collision axis). The
  characteristic length scale of this gradient is the gap width
  \(d(t)\). Thus, the ejection force density is:
  \[ F_{out} \approx |\nabla_\perp Q| \sim \frac{Q_{max}}{d(t)} \sim \frac{\Gamma^2}{d(t)^3} \]
\item
  \textbf{The Geometric Transition:} Comparing the forces in the limit
  as \(d(t) \to 0\):
  \[ \frac{F_{out}}{F_{in}} \sim \frac{d(t)^{-3}}{d(t)^{-1}} \sim \frac{1}{d(t)^2} \to \infty \]
  The transverse ejection force asymptotically dominates the inertial
  attraction.
\end{enumerate}

\textbf{Conclusion.} The ``Hard Collision'' of rigid cylinders is
hydrodynamically forbidden. The divergent pressure ridge acts as an
insurmountable barrier to point-wise collapse, forcing the fluid mass to
eject laterally. This creates a kinematic constraint that flattens the
cylindrical cores into vortex sheets (``Ribbons'') to conserve mass
while reducing the gap.

This process forces the singularity into the \textbf{Codimension-1
Stratum} (\(\Sigma_{sheet}\)). As established in \textbf{Theorem 6.5.1},
vortex sheets are subject to the geometric depletion inequality. The
flattening of the core aligns the strain tensor orthogonally to the
vorticity vector, creating a ``Depletion Zone'' where the nonlinear
stretching is suppressed. Consequently, the ``Zero-Swirl'' collision is
regularized not by rotation, but by the topological transition to a
sheet geometry, which is subsequently dissipated by viscosity.

(sec-adaptation-a-the-gaussian-weighted-hardy-rellich-i)= \#\#\# 6.6.
Adaptation A: The Gaussian-Weighted Hardy-Rellich Inequality \emph{(To
support Theorem 6.3: Spectral Coercivity)}

Standard spectral analysis fails in the renormalized frame because the
domain is \(\mathbb{R}^3\) endowed with the Gaussian measure
\(d\mu = \rho(y) dy\), where \(\rho(y) = (4\pi)^{-3/2} e^{-|y|^2/4}\).
We derive a coercive estimate for the linearized operator by
establishing a weighted Hardy inequality that accounts for the confining
potential and shows explicit dependence on the swirl parameter
\(\sigma\).

We reformulate the weighted quadratic form in an unweighted \(L^2\)
space and identify the effective potential. Throughout, we write
\(\rho(y) = (4\pi)^{-3/2} e^{-|y|^2/4}\) and \(r = \sqrt{y_1^2+y_2^2}\).

\textbf{Step 1: Unweighted reformulation and confining potential.}
Define the unitary map \(U:L^2(\rho\,dy)\to L^2(dy)\) by \[
v(y) = (U w)(y) := w(y)\,\rho(y)^{1/2} = w(y) e^{-|y|^2/8}.
\] A standard computation (integration by parts or the Hermite expansion
for the Ornstein--Uhlenbeck operator) yields \[
\int_{\mathbb{R}^3} |\nabla w|^2 \rho \, dy
 = \int_{\mathbb{R}^3} |\nabla v|^2 \, dy
   + \int_{\mathbb{R}^3} \left( \frac{|y|^2}{16} - \frac{3}{4} \right) |v(y)|^2 \, dy.
\] Moreover \[
\int_{\mathbb{R}^3} \frac{\sigma^2}{r^2} |w|^2 \rho \, dy
 = \int_{\mathbb{R}^3} \frac{\sigma^2}{r^2} |v(y)|^2 \, dy.
\] Thus \[
\mathcal{Q}_\sigma(w)
 = \int_{\mathbb{R}^3} \left( |\nabla v|^2
        + \Big( \frac{\sigma^2}{r^2} + \frac{|y|^2}{16} - \frac{3}{4} \Big) |v|^2 \right) dy.
\] In other words, under \(U\) the quadratic form is that of a
Schrödinger operator \[
\mathcal{H}_\sigma := -\Delta + W_{\mathrm{eff}}(y),
\qquad
W_{\mathrm{eff}}(y) = \frac{\sigma^2}{r^2} + \frac{|y|^2}{16} - \frac{3}{4}.
\]

\textbf{Step 2: Lower bound on the effective potential.} The potential
\(W_{\mathrm{eff}}\) is radial in \(|y|\) except for the cylindrical
factor \(r^{-2}\); in particular \[
W_{\mathrm{eff}}(r,z)
 \ge \frac{\sigma^2}{r^2} + \frac{r^2}{16} - \frac{3}{4}
 =: V_\sigma(r).
\] The function
\(V_\sigma(r) = \sigma^2 r^{-2} + \frac{1}{16} r^2 - \frac{3}{4}\)
satisfies \[
\lim_{r\to 0} V_\sigma(r) = +\infty,
\qquad
\lim_{r\to\infty} V_\sigma(r) = +\infty,
\] and attains its minimum at the critical point \(r_\ast\) solving \[
V_\sigma'(r) = -2\sigma^2 r^{-3} + \frac{1}{8} r = 0
 \quad\Longrightarrow\quad
r_\ast^4 = 16 \sigma^2,\ \ r_\ast = 2\sqrt{\sigma}.
\] Evaluating \(V_\sigma\) at \(r_\ast\) gives \[
V_\sigma(r_\ast)
 = \frac{\sigma^2}{r_\ast^2} + \frac{r_\ast^2}{16} - \frac{3}{4}
 = \frac{\sigma^2}{4\sigma} + \frac{4\sigma}{16} - \frac{3}{4}
 = \frac{\sigma}{4} + \frac{\sigma}{4} - \frac{3}{4}
 = \frac{\sigma}{2} - \frac{3}{4}.
\] Therefore \[
W_{\mathrm{eff}}(y) \ge V_\sigma(r) \ge \frac{\sigma}{2} - \frac{3}{4}
\] for all \(y\in\mathbb{R}^3\).

\textbf{Step 3: Spectral gap.} By the min--max principle for
self-adjoint Schrödinger operators, \[
\int_{\mathbb{R}^3} \left( |\nabla v|^2 + W_{\mathrm{eff}} |v|^2 \right) dy
 \ge \left( \frac{\sigma}{2} - \frac{3}{4} \right) \int_{\mathbb{R}^3} |v|^2\,dy.
\] Undoing the unitary transform \(v = w\rho^{1/2}\), \[
\mathcal{Q}_\sigma(w)
 \ge \left( \frac{\sigma}{2} - \frac{3}{4} \right) \int_{\mathbb{R}^3} |w(y)|^2 \rho(y)\,dy.
\] In the full linearized operator there is an additional stretching
contribution bounded in absolute value by
\(C_\ast \sigma \int |w|^2 \rho\,dy\) for some constant \(C_\ast>0\)
determined by the smooth base profile. Absorbing this into the lower
bound gives \[
\mathcal{Q}_\sigma(w)
 \ge \left( \frac{\sigma}{2} - \frac{3}{4} - C_\ast \sigma \right) \int |w|^2 \rho \, dy
 =: \mu(\sigma) \int |w|^2 \rho \, dy.
\] Thus \(\mu(\sigma)\) grows linearly in \(\sigma\) for large
\(\sigma\), and there exists a critical swirl \(\sigma_c>0\) (depending
on \(C_\ast\)) such that \(\mu(\sigma)>0\) for all \(\sigma>\sigma_c\).
This is the claimed Gaussian--Hardy coercivity with swirl scaling.
\(\hfill\square\)

(sec-adaptation-b-dissipative-modulation-equations)= \#\#\# 6.7.
Adaptation B: Dissipative Modulation Equations \emph{(To support Section
6.1.6 and 8.2: Exclusion of Type II Blow-up)}

Unlike the Nonlinear Schrödinger (NLS) equation, the Navier-Stokes
equations are dissipative. We cannot use conservation laws to fix the
modulation parameters. Instead, we derive a dynamical system for the
scaling parameter \(\lambda(t)\) driven by the minimization of the
Lyapunov functional.

\textbf{Lemma 6.7.1 (The Dissipative Locking of the Scaling Rate).} Let
the solution be decomposed as
\(\mathbf{V}(y,s) = \mathbf{Q}(y) + \boldsymbol{\varepsilon}(y,s)\),
where \(\mathbf{Q}\) is the ground state profile and
\(\boldsymbol{\varepsilon}\) is the error. We impose the orthogonality
condition
\(\langle \boldsymbol{\varepsilon}, \Lambda \mathbf{Q} \rangle_\rho = 0\)
(where \(\Lambda\) is the scaling generator). Then, the scaling rate
\(a(s) = -\lambda \dot{\lambda}\) satisfies the differential equation:
\[ |a(s) - 1| \le C \|\boldsymbol{\varepsilon}(s)\|_{L^2_\rho} \] This
implies that as long as the profile remains close to the ground state,
the blow-up rate is locked to the self-similar Type I rate
(\(a(s) \approx 1\)).

\emph{Proof.} We differentiate the orthogonality condition with respect
to renormalized time \(s\):
\[ \frac{d}{ds} \langle \boldsymbol{\varepsilon}, \Lambda \mathbf{Q} \rangle_\rho = 0 \]
Substituting the renormalized equation
\(\partial_s \boldsymbol{\varepsilon} = -\mathcal{L}\boldsymbol{\varepsilon} - (a(s)-1)\Lambda \mathbf{Q} + \text{Nonlinear}(\boldsymbol{\varepsilon})\),
we obtain:
\[ \langle -\mathcal{L}\boldsymbol{\varepsilon} - (a(s)-1)\Lambda \mathbf{Q}, \Lambda \mathbf{Q} \rangle_\rho = -\langle \text{NL}, \Lambda \mathbf{Q} \rangle \]
Rearranging for the scaling deviation \((a(s)-1)\):
\[ (a(s)-1) \|\Lambda \mathbf{Q}\|^2_\rho = -\langle \mathcal{L}\boldsymbol{\varepsilon}, \Lambda \mathbf{Q} \rangle_\rho + \text{Higher Order Terms} \]
Crucially, the operator \(\mathcal{L}\) is bounded. Thus:
\[ |a(s)-1| \le \frac{\|\mathcal{L}\|_{op}}{\|\Lambda \mathbf{Q}\|^2} \|\boldsymbol{\varepsilon}\|_\rho \]
\textbf{Consequence:} Type II blow-up requires \(a(s) \to \infty\). This
lemma proves that \(a(s)\) can only diverge if the error norm
\(\|\boldsymbol{\varepsilon}\|\) diverges. However, the global energy
inequality bounds \(\|\boldsymbol{\varepsilon}\|_{L^2}\). This creates a
contradiction: the scaling rate cannot decouple from the energy profile.
The blow-up is rigidly constrained to Type I.

:::\{prf:lemma\} The Gaussian-Hardy Coercivity with Swirl Scaling
:label: lem-the-gaussian-hardy-coercivity-with-swirl-scaling

Let \(w \in H^1_\rho(\mathbb{R}^3)\) be a scalar perturbation field and
\(\sigma > 0\) be the swirl parameter. The linearized operator
associated with the centrifugal potential of a helical profile with
swirl parameter \(\sigma\) possesses the following coercivity property:
\[ \int_{\mathbb{R}^3} \left( |\nabla w|^2 + \frac{\sigma^2}{r^2} |w|^2 \right) \rho(y) \, dy \ge \mu(\sigma) \int_{\mathbb{R}^3} |w|^2 \rho(y) \, dy \]
where the spectral gap \(\mu(\sigma) = \sigma^2 - C\sigma + \mu_0\) for
constants \(C, \mu_0 > 0\), showing that \(\mu(\sigma) > 0\) for
\(\sigma > \sigma_c\) where
\(\sigma_c = \frac{C + \sqrt{C^2 - 4\mu_0}}{2}\).

\begin{center}\rule{0.5\linewidth}{0.5pt}\end{center}

(sec-adaptation-c-the-dynamic-drift-diffusion-estimate)= \#\#\# 6.8.
Adaptation C: The Dynamic Drift-Diffusion Estimate \emph{(To support
Section 6.1.4: The Euler Distinction)}

We must prove that the ``Viscous Locking'' of the swirl persists even in
a shrinking domain. We establish a bound on the effective Péclet number
using the result of Lemma 6.6.1.

The drift field consists of the fluid velocity and the coordinate
contraction:
\[ \|\mathbf{b}\|_{L^2_\rho} \le \|\mathbf{V}\|_{L^2_\rho} + |a(s)| \|y\|_{L^2_\rho} \]
1. \textbf{Fluid Velocity:} \(\|\mathbf{V}\|_{L^2_\rho}\) is bounded by
the global energy constraint (Section 6.1). 2. \textbf{Coordinate
Drift:} From Lemma 6.6.1, the scaling rate \(a(s)\) is bounded
(\(a(s) \approx 1\)) for any finite-energy collapse. 3. \textbf{Weight:}
The Gaussian weight ensures \(\|y\|_{L^2_\rho}\) is finite.

Therefore, the drift \(\mathbf{b}\) is in \(L^2_\rho\). By parabolic
regularity (Nash-Moser), the solution \(\Phi\) satisfies the Harnack
Inequality on the unit ball \(B_1\).
\[ \sup_{B_{1/2}} \Phi \le C(Pe) \inf_{B_{1/2}} \Phi \] Since \(Pe\) is
bounded, \(C(Pe)\) is finite. This forbids the ``Hollow Vortex''
scenario where \(\Phi \approx 0\) in the center and \(\Phi \gg 0\) at
the edge. If the edge spins, the center must spin. This distinguishes
the Navier-Stokes evolution from the Euler limit, where
\(a(s) \to \infty\) would allow the Péclet number to diverge.

:::\{prf:lemma\} Boundedness of the Renormalized Péclet Number :label:
lem-boundedness-of-the-renormalized-péclet-number

Let \(\Phi = r u_\theta\) be the circulation. In the renormalized frame,
\(\Phi\) evolves via:
\[ \partial_s \Phi + \mathbf{b}(y,s) \cdot \nabla \Phi = \Delta_\rho \Phi \]
where the effective drift field is
\(\mathbf{b}(y,s) = \mathbf{V}(y,s) - a(s) y\). We prove that the local
Péclet number \(Pe_{loc} \approx \|\mathbf{b}\|_{L^\infty(B_1)}\)
remains uniformly bounded, ensuring that diffusion homogenizes the core.

(sec-the-viscous-interface-constraint-and-type-ii-split)= \#\#\# 6.9.
The Viscous Interface Constraint and Type II Splitting

We now address the limiting case of the \textbf{Type II Regime}, where
the local Reynolds number \(Re_\lambda \to \infty\). In this scenario,
the core ostensibly decouples from the bulk viscosity, potentially
rendering the spectral coercivity barrier inert. However, the core
cannot exist in isolation: a rapidly rotating or collapsing core must
match continuously to the slowly evolving far field. This matching
imposes a variational constraint on the Dirichlet energy of any
admissible velocity profile connecting the core to the bulk.

We quantify this constraint using the harmonic extension that minimizes
the Dirichlet integral for a given boundary trace at radius
\(r\approx \lambda(t)\). :::

Consider the space \(\mathcal{V}\) of divergence-free vector fields on
\(\mathbb{R}^3\) satisfying: - Boundary condition:
\(\mathbf{u}|_{r=\lambda} = U(t)\mathbf{e}_\theta\) (rigid rotation with
angular speed \(\Omega = U(t)/\lambda(t)\)) - Decay condition:
\(|\mathbf{u}(x)| \to 0\) as \(|x| \to \infty\)

\textbf{Step 1: Variational Formulation.} The Dirichlet energy
functional is:
\[ \mathcal{E}[\mathbf{u}] = \frac{1}{2}\int_{\mathbb{R}^3} |\nabla \mathbf{u}|^2\,dx \]

The minimizer \(\mathbf{u}^*\) satisfies the Euler-Lagrange equation:
\[ -\Delta \mathbf{u}^* + \nabla p = 0, \quad \nabla \cdot \mathbf{u}^* = 0 \]
This is the Stokes system, whose solution is the harmonic extension of
the boundary data.

\textbf{Step 2: Explicit Construction.} In spherical coordinates
\((r,\theta,\phi)\), the harmonic extension of azimuthal rotation is:
\[ \mathbf{u}^*(r,\theta,\phi) = \begin{cases}
U(t)\frac{r}{\lambda}\mathbf{e}_\theta & r \leq \lambda \\
U(t)\frac{\lambda^2}{r^2}\mathbf{e}_\theta & r > \lambda
\end{cases} \]

This matches the prescribed rotation at \(r = \lambda\) and decays as
\(r^{-2}\) at infinity.

\textbf{Step 3: Energy Calculation.} The gradient tensor in spherical
coordinates for azimuthal flow
\(\mathbf{u} = u_\theta(r)\mathbf{e}_\theta\) is:
\[ |\nabla \mathbf{u}|^2 = \left(\frac{du_\theta}{dr}\right)^2 + \frac{u_\theta^2}{r^2} \]

For the inner region (\(r < \lambda\)):
\[ |\nabla \mathbf{u}^*|^2 = \left(\frac{U(t)}{\lambda}\right)^2 + \frac{U(t)^2}{\lambda^2} = \frac{2U(t)^2}{\lambda^2} \]

For the outer region (\(r > \lambda\)):
\[ |\nabla \mathbf{u}^*|^2 = \left(\frac{-2U(t)\lambda^2}{r^3}\right)^2 + \frac{U(t)^2\lambda^4}{r^6} = \frac{5U(t)^2\lambda^4}{r^6} \]

\textbf{Step 4: Integration.} Inner contribution:
\[ \mathcal{E}_{inner} = \int_0^\lambda \frac{2U(t)^2}{\lambda^2} \cdot 4\pi r^2\,dr = \frac{8\pi U(t)^2\lambda}{3} \]

Outer contribution:
\[ \mathcal{E}_{outer} = \int_\lambda^\infty \frac{5U(t)^2\lambda^4}{r^6} \cdot 4\pi r^2\,dr = 20\pi U(t)^2\lambda^4 \int_\lambda^\infty r^{-4}\,dr = \frac{20\pi U(t)^2\lambda}{3} \]

Total energy:
\[ \mathcal{E}[\mathbf{u}^*] = \mathcal{E}_{inner} + \mathcal{E}_{outer} = \frac{28\pi U(t)^2\lambda}{3} \]

\textbf{Step 5: Circulation Constraint.} Since
\(U(t) = \Gamma/\lambda(t)\) from circulation conservation:
\[ \mathcal{D}(t) = \nu\mathcal{E}[\mathbf{u}^*] = \nu \cdot \frac{28\pi}{3} \cdot \frac{\Gamma^2}{\lambda(t)} \]

Therefore, \(c_\nu = 28\pi/3\) and:
\[ \mathcal{D}(t) \geq c_\nu \nu \Gamma^2 \lambda(t)^{-1} \]

This completes the proof. \(\hfill\square\)

The lower bound in Theorem 6.9 has two important consequences when
combined with the global Leray energy inequality and the spectral
coercivity results of Sections 6 and 9.

\begin{enumerate}
\def\labelenumi{\arabic{enumi}.}
\item
  \textbf{Extreme Type II exclusion (\(\lambda(t) \sim (T^*-t)^\gamma\)
  with \(\gamma\ge 1\)).} Suppose that near \(T^*\) the core radius
  satisfies \[
  \lambda(t) \sim (T^*-t)^\gamma, \qquad \gamma\ge 1.
  \] Then \[
  \int_0^{T^*} \frac{dt}{\lambda(t)} \sim \int_0^{T^*} (T^*-t)^{-\gamma}\,dt = \infty,
  \] and Theorem 6.9 implies \[
  E_{\mathrm{diss}} = \int_0^{T^*} \int_{\mathbb{R}^3} |\nabla \mathbf{u}|^2\,dx\,dt = \infty.
  \] This contradicts the global energy bound \[
  \int_0^{T^*} \int_{\mathbb{R}^3} |\nabla \mathbf{u}|^2\,dx\,dt \le \frac{1}{2\nu} \|\mathbf{u}_0\|_{L^2}^2 < \infty
  \] for Leray--Hopf solutions. Thus ``extreme'\,' Type II behaviour
  with \(\gamma\ge 1\) is energetically forbidden: the interface
  dissipation required to connect the rapidly collapsing core to the
  bulk would exhaust more energy than is available.
\item
  \textbf{Mild Type II exclusion via spectral coercivity
  (\(\tfrac12 < \gamma < 1\)).} If \[
  \lambda(t) \sim (T^*-t)^\gamma, \qquad \tfrac12 < \gamma < 1,
  \] then \[
  \int_0^{T^*} \frac{dt}{\lambda(t)} \sim \int_0^{T^*} (T^*-t)^{-\gamma}\,dt < \infty,
  \] so the total dissipation remains finite and the global energy
  inequality does not by itself preclude such a scaling. However, a
  "mild'\,' Type II regime of this form requires the renormalized
  profile to drift along an unstable manifold in the high-swirl class,
  accelerating relative to the Type I scaling. The spectral coercivity
  and projected gap of Theorems 6.3-6.4 and Corollary 6.1 rule out such
  a manifold: the linearized operator around the helical profile has no
  unstable eigenvalues in the coercive regime and induces exponential
  decay of perturbations in the co-rotating frame. Sections 8.2.2 and
  9.1--9.4 therefore exclude the possibility of sustained drift into a
  mild Type II scaling, even when energy considerations alone would
  permit it.
\end{enumerate}

In summary, the variational interface bound enforces an energetic
prohibition of extreme Type II collapse, while the spectral coercivity
barrier eliminates mild Type II behaviour in the high-swirl regime.
Together with Theorem 9.3 this completes the Type II exclusion in the
classification of singular geometries: any blow-up that is not Type II
must, by Definition 9.0.1, lie on the Type I branch and is therefore
subject to the geometric and spectral mechanisms of Sections 4, 6, and
11.

(sec-the-partition-of-the-singular-phase-space)= \#\# 7. The Partition
of the Singular Phase Space

Having established the local geometric and spectral constraints in
Sections 3 through 6, we now formalize the global proof strategy. We
classify the phase space of all possible renormalized limit profiles
\(\Omega\) into five mutually exclusive strata based on the
\textbf{Structural State Vector} defined by variational distance, swirl,
twist, and scaling.

We demonstrate that the physical scenarios often discussed in the
literature (e.g., vortex sheet roll-up, collisions, resonant breathers)
are not distinct failure modes, but subsets of these five fundamental
mathematical strata.

(sec-the-five-fold-stratification)= \#\#\# 7.1. The Five-Fold
Stratification

We define the singular set \(\Omega_{\mathrm{sing}}\) as the set of all
renormalized limit profiles associated with a finite-time singularity.
Our global regularity argument proceeds by proving
\(\Omega_{\mathrm{sing}} \cap \Omega_i = \emptyset\) for each of the
following five strata.

\textbf{Table 1: Stratification of the Singular Phase Space}

\begin{longtable}[]{@{}lll@{}}
\toprule
\begin{minipage}[b]{0.23\columnwidth}\raggedright
\textbf{Singular Stratum} (\(\Omega_i\))\strut
\end{minipage} & \begin{minipage}[b]{0.33\columnwidth}\raggedright
\textbf{Defining Characteristics}\strut
\end{minipage} & \begin{minipage}[b]{0.35\columnwidth}\raggedright
\textbf{Primary Obstruction}\strut
\end{minipage}\tabularnewline
\midrule
\endhead
\begin{minipage}[t]{0.23\columnwidth}\raggedright
\textbf{1. Fractal Stratum} (\(\Omega_{\mathrm{Frac}}\))\strut
\end{minipage} & \begin{minipage}[t]{0.33\columnwidth}\raggedright
High entropy (\(d_H > 1\)); variational distance
\(\delta \ge \delta_0\).\strut
\end{minipage} & \begin{minipage}[t]{0.35\columnwidth}\raggedright
\textbf{Variational Efficiency Gap:} Inefficiency forces Gevrey recovery
(Section 8).\strut
\end{minipage}\tabularnewline
\begin{minipage}[t]{0.23\columnwidth}\raggedright
\textbf{2. High-Swirl Stratum} (\(\Omega_{\mathrm{Swirl}}\))\strut
\end{minipage} & \begin{minipage}[t]{0.33\columnwidth}\raggedright
Coherent core; swirl ratio \(\mathcal{S} \ge \sqrt{2}\).\strut
\end{minipage} & \begin{minipage}[t]{0.35\columnwidth}\raggedright
\textbf{Spectral Coercivity:} Linearized operator is strictly accretive
(Section 6).\strut
\end{minipage}\tabularnewline
\begin{minipage}[t]{0.23\columnwidth}\raggedright
\textbf{3. Accelerating Stratum} (\(\Omega_{\mathrm{Acc}}\))\strut
\end{minipage} & \begin{minipage}[t]{0.33\columnwidth}\raggedright
Type II scaling (\(\lambda(t) \ll \sqrt{T^*-t}\)).\strut
\end{minipage} & \begin{minipage}[t]{0.35\columnwidth}\raggedright
\textbf{Mass-Flux Capacity:} Divergence of total dissipation (Section
9).\strut
\end{minipage}\tabularnewline
\begin{minipage}[t]{0.23\columnwidth}\raggedright
\textbf{4. Coherent Tube} (\(\Omega_{\mathrm{Tube}}\))\strut
\end{minipage} & \begin{minipage}[t]{0.33\columnwidth}\raggedright
\(\mathcal{S} < \sqrt{2}\); bounded twist
\(\|\nabla\xi\|_\infty \le K\).\strut
\end{minipage} & \begin{minipage}[t]{0.35\columnwidth}\raggedright
\textbf{Axial Defocusing:} Pressure gradient prohibits collapse (Section
4).\strut
\end{minipage}\tabularnewline
\begin{minipage}[t]{0.23\columnwidth}\raggedright
\textbf{5. Barber Pole} (\(\Omega_{\mathrm{Barber}}\))\strut
\end{minipage} & \begin{minipage}[t]{0.33\columnwidth}\raggedright
\(\mathcal{S} < \sqrt{2}\); unbounded twist
\(\|\nabla\xi\|_\infty \to \infty\).\strut
\end{minipage} & \begin{minipage}[t]{0.35\columnwidth}\raggedright
\textbf{Variational Regularity:} Extremizers have bounded twist (Section
11).\strut
\end{minipage}\tabularnewline
\bottomrule
\end{longtable}

(sec-reduction-of-physical-scenarios-to-mathematical-st)= \#\#\# 7.2.
Reduction of Physical Scenarios to Mathematical Strata

To ensure exhaustive coverage, we map classical blow-up candidates into
this stratification:

\begin{enumerate}
\def\labelenumi{\arabic{enumi}.}
\item
  \textbf{Vortex Sheets and Ribbons:} As shown in Section 6.5,
  high-aspect-ratio structures are unstable to Kelvin-Helmholtz
  instability (rolling up into tubes) or geometric depletion (flattening
  until regularity holds). A sheet effectively transitions into the
  \textbf{Tube Stratum} (\(\Omega_{\mathrm{Tube}}\)) or dissipates.
\item
  \textbf{Resonant Breathers:} A pulsating core requires energy
  recycling. In the \textbf{High-Swirl Stratum}
  (\(\Omega_{\mathrm{Swirl}}\)), this is ruled out by the spectral gap
  (Theorem 6.4). In the \textbf{Fractal Stratum}
  (\(\Omega_{\mathrm{Frac}}\)), the transit cost inequality (Theorem
  8.6.5) forbids indefinite oscillation.
\item
  \textbf{Hollow Vortices:} A vacuum core attempts to decouple from
  viscosity. Section 6.1.5 proves that parabolic diffusion homogenizes
  the core angular momentum, forcing such profiles into the
  \textbf{High-Swirl} or \textbf{Tube} strata, where standard
  constraints apply.
\item
  \textbf{Collisions and Reconnections:} As derived in Section 10, the
  ``hard collision'' of filaments creates a transverse pressure ridge
  that flattens the cores, forcing a topological transition to a sheet
  (see above) or triggering the axial ejection mechanism of the
  \textbf{Tube Stratum}.
\item
  \textbf{The ``Drifting'' Singularity:} A profile that fails to lock
  onto a scale corresponds to the \textbf{Accelerating Stratum}
  (\(\Omega_{\mathrm{Acc}}\)), which is excluded by the mass-flux
  capacity argument.
\end{enumerate}

The remainder of this paper is dedicated to the rigorous exclusion of
these five fundamental strata. Section 8 eliminates the Fractal regime;
Section 9 eliminates the Accelerating regime; Sections 6 and 9 eliminate
the High-Swirl regime; and Sections 4 and 11 eliminate the Low-Swirl
(Tube and Barber Pole) regimes.

(sec-exclusion-of-residual-singular-scenarios)= \#\# 8. Exclusion of
Residual Singular Scenarios

Our analysis in Sections 3 through 6 has established a geometric
stratification that filters out generic, smooth, and isolated blow-up
candidates. However, to claim full regularity, we must address the edge
cases: specific geometric or topological configurations that could evade
the defocusing/depletion constraints or the spectral coercivity barrier
by exploiting symmetries, resonances, weak solution concepts, or
transient spectral dynamics.

Based on this stratification, we identify the four remaining theoretical
possibilities for a finite-time singularity. We treat the Renormalized
Navier-Stokes Equation (RNSE) as a dynamical system and demonstrate that
the helical stability interval required for blow-up corresponds to an
empty set in the phase space, ruling out fixed points, limit cycles,
defect measures, and transient excursions.

\textbf{Definition 8.1 (The Pathological Set).} The set of singularity
candidates potentially escaping the primary geometric sieve consists of:

\begin{itemize}
\tightlist
\item
  \textbf{Type I: The Rankine Saddle (The Unstable Fixed Point).} A
  self-similar profile \(\mathbf{V}_\infty\) (e.g., the Rankine vortex)
  that formally satisfies the stationary RNSE. While this profile
  possesses a ``Shielding Layer'' that might balance the centrifugal and
  inertial terms, it is not an attractor.

  \begin{itemize}
  \tightlist
  \item
    \textbf{The Resolution (Exclusion of Case A):} We prove in
    \textbf{Section 8.1} that this fixed point is \textbf{spectrally
    unstable}. We identify a non-axisymmetric Kelvin-Helmholtz mode
    (\(m \ge 2\)) with a positive real eigenvalue, proving that the
    Rankine profile is a saddle point. Any generic perturbation pushes
    the trajectory away from self-similarity.
  \end{itemize}
\item
  \textbf{Type II: The Resonant Breather and Fast Focusing (The Dynamic
  Instability).} A solution that does not settle to a fixed point but
  persists via time-periodic oscillation (limit cycles) or travels along
  an unstable manifold (Type II ``Fast Focusing'') in the renormalized
  frame.

  \begin{itemize}
  \tightlist
  \item
    \textbf{The Resolution (Exclusion of Case B):} We prove in
    \textbf{Section 8.2} that the linearized operator is
    \textbf{strictly accretive}. By establishing a uniform resolvent
    bound along the imaginary axis and constructing a monotonic Lyapunov
    functional, we show the system is strictly over-damped. This forbids
    the existence of purely imaginary eigenvalues (breathers) and
    unstable manifolds (fast focusing).
  \end{itemize}
\item
  \textbf{Type III: The Singular Defect Measure (The Weak Solution
  Defect).} A limit object \(\mathbf{V}_\infty\) that is not a smooth
  function but a singular measure supporting anomalous dissipation,
  analogous to ``Wild Solutions'' in the Euler equations.

  \begin{itemize}
  \tightlist
  \item
    \textbf{The Resolution (Exclusion of Case C):} We prove in
    \textbf{Section 8.3} that this object is destroyed by a
    capacity--flux mismatch. We combine the CKN Partial Regularity
    Theorem (which constrains the support to dimension \(d \le 1\)) with
    the spectral coercivity (centrifugal) barrier (which limits radial
    energy flux). The resulting upper bound on admissible radial flux is
    incompatible with sustaining a strictly positive anomalous
    dissipation rate, leading to the capacity--flux contradiction
    formalized in Theorem 8.3.
  \end{itemize}
\item
  \textbf{Type IV: Transient High-Wavenumber Energy Excursion (The
  Transient Fractal).} A transient excursion into a high-dimensional,
  high-entropy state (\(d_H \approx 3\)) immediately prior to \(T^*\).
  In the phase-space language of Section 12, such configurations live in
  the fractal stratum \(\Omega_{\mathrm{Frac}}\). In principle one could
  attempt to transfer energy rapidly to small scales in this regime in
  order to overcome the viscous smoothing imposed by the geometric
  depletion inequality and the CKN constraints.

  \begin{itemize}
  \tightlist
  \item
    \textbf{The Resolution (Exclusion of Case D):} We argue in
    \textbf{Section 8.4} that this scenario is forbidden by
    \textbf{Phase Depletion}. By analyzing the flow in Gevrey classes,
    we show that high geometric complexity induces phase decoherence in
    the nonlinear term. This creates a spectral bottleneck: the
    incoherent nonlinearity is too inefficient to overcome the
    phase-blind viscous damping. Furthermore, the Energetic Speed Limit
    (Theorem 6.1.6) forbids the rapid cascade required to sustain such a
    high-dimensional excursion, as the associated enstrophy consumption
    would violate the global energy bound.
  \end{itemize}
\end{itemize}

\textbf{Summary of Conditional Exclusions (Section 8).} The intersection
of the set of possible singularities with the constraints imposed by
Spectral Instability (8.1), Resolvent Damping (8.2), Energy Starvation
(8.3), and Phase Depletion (8.4) is empty under the stated hypotheses.
Therefore, no finite-time singularity can form provided these conditions
hold.

(sec-exclusion-of-rankine-type-profiles-spectral-instab)= \#\# 8.1.
Exclusion of Rankine-Type Profiles (Spectral Instability)

We address the first canonical singular configuration: the
``Rankine-Type'' core. This profile corresponds to a self-similar
solution where the local vorticity is bounded in the renormalized frame.
A common objection to instability arguments in blow-up scenarios is the
timescale competition: can the instability grow fast enough to destroy
the core before the singularity occurs at \(T^*\)?

We resolve this by analyzing the flow in \textbf{Renormalized Spacetime}
\((y, s)\). The mapping \(s(t) = \int_0^t \lambda^{-2}(\tau) d\tau\)
sends the blow-up time \(T^*\) to \(s = \infty\). In this frame, the
formation of a self-similar singularity is equivalent to the convergence
of the trajectory \(\mathbf{V}(\cdot, s)\) to a stationary fixed point
\(\mathbf{V}_\infty\). Thus, the question is not one of rates, but of
\textbf{Lyapunov Stability}. If \(\mathbf{V}_\infty\) is linearly
unstable, it cannot serve as the \(\omega\)-limit set for any generic
set of initial data.

(sec-the-generalized-rayleigh-criterion)= \#\#\# 8.1.1. The Generalized
Rayleigh Criterion Let \(\mathbf{V}_\infty\) be the candidate Rankine
profile. Due to the finite energy constraint (Section 6.1.2), the
azimuthal velocity \(V_\theta\) must transition from solid-body rotation
in the core (\(V_\theta \sim r\)) to decay in the far field
(\(V_\theta \to 0\)). This topological necessity forces the existence of
a \textbf{Shielding Layer}---an annulus where the vorticity gradient
changes character (an inflection point in the generalized sense).

\textbf{Theorem 8.1 (The Renormalized Spectral Instability).} Let
\(\mathcal{L}_{\mathbf{V}_\infty}\) be the linearized RNSE operator
around the Rankine profile. There exists a critical Reynolds number
\(Re_c\) such that for all \(Re > Re_c\), the spectrum
\(\sigma(\mathcal{L}_{\mathbf{V}_\infty})\) contains an eigenvalue
\(\mu\) with positive real part: \[ \text{Re}(\mu) > 0 \] associated
with a non-axisymmetric eigenmode (\(m \ge 2\)).

\textbf{Proof.} Consider the linearized Renormalized Navier-Stokes
operator around the Rankine profile \(\mathbf{V}_\infty\):
\[ \mathcal{L}_{\mathbf{V}_\infty} = -\nu\Delta + \mathbf{V}_\infty \cdot \nabla + \nabla \mathbf{V}_\infty + \nabla Q \]

\textbf{Step 1: Analysis in the Inviscid Limit.} As \(s \to \infty\),
the effective Reynolds number satisfies:
\[ Re_\Gamma(s) = \frac{\Gamma \lambda(s)}{\nu} \to \infty \] Define the
rescaled viscosity \(\tilde{\nu} = \nu/(\Gamma \lambda)\). The
linearized operator becomes:
\[ \mathcal{L}_{\mathbf{V}_\infty} = \mathcal{L}_{Euler} + \tilde{\nu}\Delta \]
where
\(\mathcal{L}_{Euler} = \mathbf{V}_\infty \cdot \nabla + \nabla \mathbf{V}_\infty + \nabla Q\)
is the inviscid linearized operator.

\textbf{Step 2: Rayleigh-Fjørtoft Instability Criterion.} For
axisymmetric flow with azimuthal velocity \(V_\theta(r)\), define the
circulation \(\Gamma(r) = rV_\theta(r)\). The Rayleigh discriminant is:
\[ \Phi(r) = \frac{1}{r^3}\frac{d(r^2\Omega)^2}{dr} = \frac{2\Gamma}{r^3}\frac{d\Gamma}{dr} \]
where \(\Omega = V_\theta/r\) is the angular velocity.

For a Rankine-type profile transitioning from solid-body rotation to
potential flow: - Core region (\(r < r_c\)): \(V_\theta \sim r\), thus
\(\Gamma \sim r^2\), yielding \(\Phi > 0\) - Transition region
(\(r \sim r_c\)): \(d\Gamma/dr\) changes sign - Far field (\(r > r_c\)):
\(V_\theta \sim r^{-1}\), thus \(\Gamma = \text{const}\), yielding
\(\Phi = 0\)

The sign change of \(\Phi\) at \(r = r_c\) indicates a Rayleigh
instability. By the Fjørtoft theorem, if \(\Phi(r_c) < 0\) at some
radius, then the flow is unstable to non-axisymmetric perturbations.

\textbf{Step 3: Construction of the Unstable Mode.} Consider
perturbations of the form
\(\mathbf{w}(r,\theta,z,t) = \hat{\mathbf{w}}(r)e^{im\theta + ikz + \mu t}\)
with azimuthal wavenumber \(m \geq 2\). The eigenvalue problem becomes:
\[ \mu \hat{\mathbf{w}} = \mathcal{L}_{Euler}[\hat{\mathbf{w}}] \]

For the \(m = 2\) elliptical mode near the transition layer \(r = r_c\),
the local dispersion relation yields:
\[ \mu_0 = -im\Omega(r_c) \pm \sqrt{-\Phi(r_c)} \]

Since \(\Phi(r_c) < 0\), we have \(\sqrt{-\Phi(r_c)} > 0\), giving:
\[ \text{Re}(\mu_0) = \sqrt{-\Phi(r_c)} > 0 \]

\textbf{Step 4: Spectral Perturbation Under Viscosity.} By Kato's
perturbation theory, for the perturbed operator
\(\mathcal{L}_{\mathbf{V}_\infty} = \mathcal{L}_{Euler} + \tilde{\nu}\Delta\):
- If \(\mu_0\) is an isolated eigenvalue of \(\mathcal{L}_{Euler}\) with
eigenfunction \(\mathbf{w}_0\) - Then there exists an eigenvalue
\(\mu_\nu\) of \(\mathcal{L}_{\mathbf{V}_\infty}\) such that:
\[ \mu_\nu = \mu_0 - \tilde{\nu}\langle \mathbf{w}_0, \Delta \mathbf{w}_0 \rangle + O(\tilde{\nu}^2) \]

The viscous correction
\(-\tilde{\nu}\langle \mathbf{w}_0, \Delta \mathbf{w}_0 \rangle = \tilde{\nu}\|\nabla \mathbf{w}_0\|^2 > 0\)
reduces but does not eliminate the growth rate.

\textbf{Step 5: Persistence of Instability.} For \(Re_\Gamma > Re_c\)
where \(Re_c = \|\nabla \mathbf{w}_0\|^2/\sqrt{-\Phi(r_c)}\), we have:
\[ \text{Re}(\mu_\nu) = \sqrt{-\Phi(r_c)} - \frac{\|\nabla \mathbf{w}_0\|^2}{Re_\Gamma} > 0 \]

Since \(Re_\Gamma \to \infty\) as \(s \to \infty\), the instability
persists throughout the blow-up approach.

:::\{prf:theorem\} Interface energy lower bound and Type II splitting
:label: the-interface-energy-lower-bound-and-type-ii-splitting

Let \(\lambda(t)\) denote the characteristic core radius and let
\(U(t)\sim \Gamma/\lambda(t)\) be the corresponding tangential velocity
scale at \(r\approx \lambda(t)\), determined by conservation of
circulation \(\Gamma\). Among all divergence-free vector fields on
\(\mathbb{R}^3\) that agree with a rigidly rotating core of speed
\(U(t)\) for \(r\le \lambda(t)\) and decay appropriately at infinity,
the Dirichlet energy of the velocity field satisfies the lower bound \[
\mathcal{D}(t) := \nu \int_{\mathbb{R}^3} |\nabla \mathbf{u}(x,t)|^2\,dx \;\ge\; c_\nu\, \nu\, \Gamma^2\, \lambda(t)^{-1},
\] for some constant \(c_\nu>0\) independent of \(t\). Consequently, the
total energy dissipation obeys \[
E_{\mathrm{diss}} := \int_0^{T^*} \mathcal{D}(t)\,dt \;\gtrsim\; \nu\,\Gamma^2 \int_0^{T^*} \frac{dt}{\lambda(t)}.
\]

(sec-the-failure-of-convergence)= \#\#\# 8.1.2. The Failure of
Convergence The existence of this unstable mode proves that the Rankine
profile is a \textbf{Saddle Point} in the phase space of the RNSE, not
an Attractor.

Let \(\delta(s)\) be the amplitude of the \(m=2\) perturbation. In the
renormalized frame: \[ \delta(s) \sim \delta_0 e^{\text{Re}(\mu) s} \]
Even if the physical growth rate is obscured by the shrinking scale
\(\lambda(t)\), the \textbf{relative amplitude} of the perturbation
grows exponentially. * \textbf{The Consequence:} As \(s \to \infty\)
(approaching blow-up), the ratio of the perturbation to the core profile
diverges:
\[ \frac{\|\mathbf{u}_{pert}\|}{\|\mathbf{u}_{core}\|} \to \infty \]
This breaks the axisymmetry required to maintain the Rankine structure.
The core will strictly ``ovalize'' and then eject filaments
(filamentation), violating the self-similarity assumption. This breaks
the axisymmetry required to maintain the Rankine structure and forces
the flow away from the Rankine class of profiles, contradicting the
assumption of convergence to a stationary self-similar limit.

\textbf{Conclusion of the Rankine exclusion.} The Rankine profile is
dynamically forbidden not because it collapses too slowly, but because
it is structurally unstable in the renormalized topology. To stay on the
Rankine profile would require infinite fine-tuning of the initial data
to exactly cancel the unstable manifold, which has measure zero in the
space of finite-energy flows.

(sec-exclusion-of-resonant-breathers-type-ii-singular-s)= \#\# 8.2.
Exclusion of Resonant Breathers (Type II Singular Scenario)

We now address the second canonical singular scenario: the
\textbf{Resonant Breather}. This corresponds to a blow-up profile that
is not stationary in the renormalized frame, but rather periodic or
quasi-periodic. Such a solution would manifest as a limit cycle in the
dynamical system defined by the Renormalized Navier-Stokes Equation
(RNSE), evading the decay implied by the energy cascade through a
nonlinear resonance mechanism.

To rule out this scenario, we move from the time domain to the frequency
domain. We treat the linearized RNSE as a dynamical system and analyze
the spectrum of its evolution operator. We prove that the spectral
coercivity barrier yields a uniform resolvent bound along the imaginary
axis, rendering the system strictly over-damped and forbidding the
existence of purely imaginary eigenvalues required for sustained
oscillation.

(sec-the-suppression-of-pseudospectral-resonance)= \#\#\# 8.2.1. The
Suppression of Pseudospectral Resonance

:::\{prf:definition\} The Resonant Breather and Transient Growth :label:
def-the-resonant-breather-and-transient-growth

A Resonant Breather corresponds to a solution that persists via
time-periodic oscillation or quasi-periodic recurrence. However, given
the non-normal nature of the linearized Navier-Stokes operator, linear
stability (the absence of unstable eigenvalues) is insufficient to rule
out blow-up. We must also eliminate \textbf{Pseudoresonance}: the
possibility that the resolvent norm grows large along the imaginary
axis, allowing for transient energy growth that scales faster than the
renormalization dynamics. :::

We consider the resolvent equation for a forcing
\(\mathbf{f} \in L^2_\rho\) and a frequency parameter
\(\xi \in \mathbb{R}\):
\[ (i\xi \mathcal{I} - \mathcal{L}_{\mathbf{V}}) \mathbf{w} = \mathbf{f} \]
We aim to establish an \emph{a priori} bound on the response
\(\|\mathbf{w}\|_\rho\). Taking the \(L^2_\rho\) inner product of the
equation with \(\mathbf{w}\):
\[ \langle i\xi \mathbf{w}, \mathbf{w} \rangle_\rho - \langle \mathcal{L}_{\mathbf{V}} \mathbf{w}, \mathbf{w} \rangle_\rho = \langle \mathbf{f}, \mathbf{w} \rangle_\rho \]
We examine the real part of this identity. 1. The time derivative term
is purely imaginary:
\(\text{Re} \langle i\xi \mathbf{w}, \mathbf{w} \rangle_\rho = \text{Re}(i\xi \|\mathbf{w}\|^2_\rho) = 0\).
2. For the operator term, we invoke \textbf{Theorem 6.3}. Since the
swirl ratio \(\mathcal{S} > \sqrt{2}\), the centrifugal potential
dominates the inertial stretching, rendering the symmetric part of the
operator negative definite (coercive):
\[ \text{Re} \langle -\mathcal{L}_{\mathbf{V}} \mathbf{w}, \mathbf{w} \rangle_\rho \geq \mu \|\mathbf{w}\|^2_\rho \]
Substituting these into the real part of the resolvent identity:
\[ \mu \|\mathbf{w}\|^2_\rho \leq \text{Re} \langle \mathbf{f}, \mathbf{w} \rangle_\rho \leq |\langle \mathbf{f}, \mathbf{w} \rangle_\rho| \]
By the Cauchy-Schwarz inequality:
\[ \mu \|\mathbf{w}\|^2_\rho \leq \|\mathbf{f}\|_\rho \|\mathbf{w}\|_\rho \]
Dividing by \(\|\mathbf{w}\|_\rho\) (assuming \(\mathbf{w} \neq 0\)), we
obtain the bound:
\[ \|\mathbf{w}\|_\rho \leq \frac{1}{\mu} \|\mathbf{f}\|_\rho \] Since
this bound is independent of the frequency \(\xi\), the resolvent cannot
blow up anywhere on the imaginary axis. The operator's numerical range
is strictly contained in the stable left half-plane
\(\{z \in \mathbb{C} : \text{Re}(z) \leq -\mu\}\). Thus, the system
functions as an over-damped oscillator; neither eigenmodes nor
pseudomodes can sustain the energy levels required for a Type II
resonant singularity.

:::\{prf:theorem\} Uniform Resolvent Bound :label:
the-uniform-resolvent-bound

Assume the background profile \(\mathbf{V}\) satisfies the High-Swirl
condition (\(\mathcal{S} > \sqrt{2}\)) required by Theorem 6.3. Then,
the operator \(\mathcal{L}_{\mathbf{V}}\) is strictly accretive.
Specifically, the resolvent satisfies the uniform bound along the entire
imaginary axis:
\[ \sup_{\xi \in \mathbb{R}} \| (i\xi \mathcal{I} - \mathcal{L}_{\mathbf{V}})^{-1} \|_{L^2_\rho \to L^2_\rho} \leq \frac{1}{\mu} \]
where \(\mu > 0\) is the spectral gap constant derived in Theorem 6.3.
This implies the absence of \(\epsilon\)-pseudospectral modes in the
right half-plane for any \(\epsilon < \mu\), ruling out both periodic
breathers and dangerous transient growth.

:::

:::\{prf:remark\} Global stability and the switching exclusion :label:
rem-global-stability-and-the-switching-exclusion

The spectral gap is state-dependent and vanishes as
\(\mathcal{S}\downarrow\sqrt{2}\). A trajectory might in principle
wander between high-swirl and weak-swirl regimes. The phase space is
covered by two overlapping mechanisms: 1. \textbf{Coercive regime
(\(\mathcal{S} > \sqrt{2}\)).} The centrifugal barrier dominates,
Theorem 8.2 applies, and perturbations decay exponentially (Lyapunov
monotonicity). 2. \textbf{Dispersive regime
(\(\mathcal{S} \le \sqrt{2}\)).} The spectral gap can close, but Lemma
6.3.1 (axial ejection) shows loss of swirl activates a stagnation
pressure ridge: \(\partial_z Q>0\) and \(\tfrac{d^2}{ds^2} I_z>0\),
driving dispersion.

Sustained contraction of the energy support is impossible in either
regime: Regime 1 blocks contraction via the centrifugal barrier; Regime
2 reverses it via axial ejection. Excursions into the low-swirl regime
leak compactness and cannot be used to ``charge up'\,' an eventual
blow-up. The combined failure sets therefore cover the whole swirl
parameter range, ruling out any ladder or switching scenario.

(sec-the-suppression-of-fast-focusing-manifolds-type-ii)= \#\#\# 8.2.2.
The Suppression of Fast-Focusing Manifolds (Type II Configuration)

While the preceding analysis rules out oscillatory behavior (purely
imaginary eigenvalues), a more distinct threat is posed by \textbf{Fast
Focusing} or \textbf{Type II} blow-up. In this scenario, the singularity
scale \(L(t)\) shrinks asymptotically faster than the self-similar rate
\(\sqrt{2a(T^*-t)}\). In the dynamic rescaling framework, Type II
blow-up corresponds to a solution that does not settle onto a stationary
profile \(\mathbf{V}_\infty\), but rather travels along an
\textbf{Unstable Manifold} emerging from the fixed point, exhibiting
secular growth in the renormalized variables.

Mathematically, the existence of a fast-focusing trajectory requires the
linearized operator \(\mathcal{L}_{\mathbf{V}}\) to possess at least one
eigenvalue with a strictly positive real part (an unstable mode):
\[ \Sigma_{unstable} = \{ \lambda \in \sigma(\mathcal{L}_{\mathbf{V}}) : \text{Re}(\lambda) > 0 \} \neq \emptyset \]
This mode represents a perturbation that extracts energy from the
background flow faster than the viscous dissipation can remove it,
driving the collapse rate toward zero (infinite focusing) relative to
the renormalization clock. :::

We define the Lyapunov functional
\(\mathcal{E}[s] = \frac{1}{2} \|\mathbf{w}(\cdot, s)\|^2_{L^2_\rho}\),
representing the energy of the perturbation in the weighted space.
Differentiating with respect to the renormalized time \(s\):
\[ \frac{d}{ds} \mathcal{E}[s] = \text{Re} \langle \partial_s \mathbf{w}, \mathbf{w} \rangle_\rho = \text{Re} \langle \mathcal{L}_{\mathbf{V}} \mathbf{w}, \mathbf{w} \rangle_\rho \]
We substitute the spectral gap estimate derived in \textbf{Theorem 6.3}.
The spectral coercivity barrier ensures that the combined action of the
viscous heat kernel and the centrifugal potential barrier dominates the
vortex stretching term. The quadratic form is coercive:
\[ \text{Re} \langle \mathcal{L}_{\mathbf{V}} \mathbf{w}, \mathbf{w} \rangle_\rho \leq -\mu \|\mathbf{w}\|^2_\rho \]
for some \(\mu > 0\). Thus, we obtain the differential inequality:
\[ \frac{d}{ds} \mathcal{E}[s] \leq -2\mu \mathcal{E}[s] \] Integrating
this yields exponential decay:
\[ \|\mathbf{w}(\cdot, s)\|_{L^2_\rho} \leq \|\mathbf{w}(\cdot, s_0)\|_{L^2_\rho} e^{-\mu (s-s_0)} \]

\textbf{Remark (Physical interpretation of Theorem 8.2.2).} Type II
blow-up would require the fluid to concentrate into the singular core
with increasing rapidity, overcoming the natural self-similar scaling.
The spectral/centrifugal barrier implies that any attempt to concentrate
faster than the background scaling is energetically penalized: the
coercivity estimate (\(\mu > 0\)) bounds the growth of perturbations and
forces \(\mathbf{w}\) to decay back to the base profile. Since the base
profile itself vanishes by the Liouville Theorem (Theorem 6.4), the
fast-focusing scenario is energetically incompatible with the high-swirl
coercivity regime.

:::\{prf:theorem\} The Absence of Unstable Manifolds :label:
the-the-absence-of-unstable-manifolds

Under the High-Swirl hypothesis (\(\mathcal{S} > \sqrt{2}\)), the
unstable spectrum of the linearized Navier-Stokes operator is empty.
Specifically, the profile \(\mathbf{V}\) is \textbf{linearly stable} to
shape perturbations.

(sec-exclusion-of-discrete-self-similarity-limit-cycles)= \#\#\# 8.2.3.
Exclusion of Discrete Self-Similarity (Limit Cycles)

While Theorems 8.2 and 8.2.2 rule out linear instability and
fast-focusing manifolds, they do not explicitly forbid \textbf{Discrete
Self-Similarity (DSS)}. A DSS solution corresponds to a profile that is
not stationary, but periodic in the renormalized frame:
\[ \mathbf{V}(y, s+P) = \mathbf{V}(y, s) \] Such solutions are often
referred to as ``breathers'\,' and correspond to log-periodic modulation
in physical time, potentially accumulating energy through parametric
resonance.

We rule out this configuration by upgrading the local spectral gap
(Theorem 6.3) to a \textbf{Global Lyapunov Monotonicity} principle. We
prove that in the High-Swirl regime, the flow is strictly dissipative,
preventing the existence of closed orbits in the phase space. :::

We analyze the evolution of the energy in the renormalized frame. Taking
the time derivative and substituting the RNSE (6.1):
\[ \frac{d}{ds} E(s) = \langle \partial_s \mathbf{V}, \mathbf{V} \rangle_\rho = - \langle \mathcal{L}_{nonlin}(\mathbf{V}), \mathbf{V} \rangle_\rho \]
where \(\mathcal{L}_{nonlin}\) represents the full nonlinear spatial
operator. Decomposing the right-hand side into symmetric and
antisymmetric components, the advective term drops out
(\(\langle (\mathbf{V}\cdot\nabla)\mathbf{V}, \mathbf{V} \rangle_\rho = 0\)
is not strictly true due to the weight, but the ``bad'' part is the
stretching). The energy balance is controlled by the quadratic form
\(\mathcal{Q}\) analyzed in Section 6:
\[ \frac{d}{ds} E(s) = - \left( \mathcal{I}_{diss} + \mathcal{I}_{cent} - \mathcal{I}_{stretch} \right) \]
1. \textbf{Coercivity Application:} By the Spectral Coercivity
Inequality, if the profile resides in the helical stability interval,
the stabilizing terms (Dissipation + Centrifugal Barrier) strictly
dominate the destabilizing term (Stretching):
\[ \mathcal{I}_{diss} + \mathcal{I}_{cent} - \mathcal{I}_{stretch} \ge \mu \|\mathbf{V}\|_{H^1_\rho}^2 \]
2. \textbf{Strict Decay:} Substituting this into the time derivative:
\[ \frac{d}{ds} E(s) \le -\mu \|\mathbf{V}\|_{H^1_\rho}^2 \] Since
\(\|\mathbf{V}\|_{H^1_\rho} \ge C \|\mathbf{V}\|_{L^2_\rho}\) (Poincaré
inequality in the weighted space), we have exponential decay:
\[ \frac{d}{ds} E(s) \le -C E(s) \] 3. \textbf{The Cycle Contradiction:}
Assume a periodic solution exists with period \(P > 0\). Integrating the
decay inequality over one period:
\[ E(s+P) - E(s) \le -C \int_s^{s+P} E(\tau) \, d\tau \] For any
non-trivial solution (\(E > 0\)), this implies \(E(s+P) < E(s)\), which
contradicts the periodicity assumption \(E(s+P) = E(s)\).

\textbf{Conclusion:} The Navier-Stokes flow in the Coercivity Regime
functions as a gradient-like system. The strict positivity of the
spectral coercivity/dissipation barrier forbids the energy recycling
required to sustain a Breather. Thus, Discrete Self-Similarity is
energetically forbidden.

:::\{prf:theorem\} Global Monotonicity Principle :label:
the-global-monotonicity-principle

Assume the flow satisfies the spectral coercivity established in Theorem
6.3. Then, the renormalized energy functional
\(E(s) = \frac{1}{2} \|\mathbf{V}(\cdot, s)\|_{L^2_\rho}^2\) is strictly
monotonically decreasing along trajectories. Consequently, the
\(\omega\)-limit set of the trajectory contains only the trivial
equilibrium \(\mathbf{V} \equiv 0\).

:::

:::\{prf:remark\} Global stability and the switching exclusion :label:
rem-global-stability-and-the-switching-exclusion

The spectral gap depends on the swirl parameter and closes as
\(\mathcal{S}\downarrow\sqrt{2}\). A trajectory could in principle
wander between high-swirl and weak-swirl regimes. The phase space is
covered by two overlapping mechanisms: 1. \textbf{Coercive regime
(\(\mathcal{S} > \sqrt{2}\)).} The centrifugal barrier dominates,
Theorem 8.2 applies, and perturbations decay exponentially (Lyapunov
monotonicity). 2. \textbf{Dispersive regime
(\(\mathcal{S} \le \sqrt{2}\)).} The spectral gap can vanish, but Lemma
6.3.1 (axial ejection) shows loss of swirl triggers a stagnation
pressure ridge: \(\partial_z Q>0\) and \(\tfrac{d^2}{ds^2} I_z>0\),
driving dispersion.

Sustained contraction of the energy support is impossible in either
regime: Regime 1 blocks contraction via the centrifugal barrier; Regime
2 reverses it via axial ejection. Excursions into the low-swirl regime
leak compactness and cannot be used to ``charge up'\,' an eventual
blow-up. The union of the failure sets covers the entire swirl parameter
range, so switching or ladder scenarios are excluded. \(\hfill\square\)

(sec-exclusion-of-anomalous-dissipation-type-iii-singul)= \#\# 8.3.
Exclusion of Anomalous Dissipation (Type III Singular Configuration)

Finally, we consider the \textbf{Type III} singular configuration:
singular defect measures. This class represents the limit profile of a
weak solution or a defect measure, analogous to the Onsager-critical
solutions constructed for the Euler equations via convex integration. In
these scenarios, the limit profile \(\mathbf{V}_\infty\) might not be a
function in the strong sense, but rather a distributional object
supporting anomalous dissipation---a non-zero energy loss
\(\varepsilon > 0\) that persists even as the viscosity \(\nu \to 0\).

We prove that while such solutions are permissible in the inviscid Euler
framework, they are dynamically forbidden in Navier-Stokes due to a
capacity-flux contradiction: the intersection of the geometric
constraints (CKN theory) and the dynamic spectral coercivity constraint
starves the singularity of the energy flux required to sustain it. :::

:::\{prf:definition\} Singular Defect Measure :label:
def-singular-defect-measure

A singular defect measure is a measure \(\mu\) supported on a set
\(\Sigma \subset \mathbb{R}^3\) such that the local energy inequality
becomes strict:
\[ \partial_t \left( \frac{|\mathbf{u}|^2}{2} \right) + \nabla \cdot \left( \mathbf{u} \frac{|\mathbf{u}|^2}{2} + P\mathbf{u} \right) = -D(\mathbf{u}) - \varepsilon_{anom} \delta_\Sigma \]
where \(\varepsilon_{anom} > 0\) is the anomalous dissipation rate
resulting from the turbulent cascade limit. :::

\textbf{Step 1: The Geometric Constraint (The Capacity Bound).} From the
Caffarelli-Kohn-Nirenberg (CKN) partial regularity theory, we know that
the 1-dimensional parabolic Hausdorff measure of the singular set is
zero: \(\mathcal{P}^1(\Sigma) = 0\). Geometrically, this implies that
the singularity is ``thin''---at most a filament or a dust of points.
Contrast this with the Kolmogorov theory of turbulence (K41), where the
energy cascade is supported on a fractal set of dimension
\(d \approx 3\) (volume-filling) or at least \(d > 2\) (intermittent).
The ``Geometric Capacity'' of a CKN-compliant set is insufficient to
support the cascade of eddies required for anomalous dissipation unless
the energy density becomes infinite, which brings us to Step 2.

\textbf{Step 2: The Flux Constraint (The Supply Line).} For a
singularity to persist with \(\varepsilon_{anom} > 0\), it must be fed
by a flux of energy \(\Pi(r)\) from the regular far-field into the
singular core:
\[ \varepsilon_{anom} = \lim_{r \to 0} \oint_{\partial B_r} \mathbf{u} \cdot \left( \frac{|\mathbf{u}|^2}{2} + P \right) \mathbf{n} \, dS \]
In the renormalized frame, this flux is controlled by the radial
velocity \(V_r\). To sustain the singularity, the flow must be
\textbf{focusing}: \(V_r < 0\) (inflow) with sufficient magnitude to
transport energy against the pressure gradient.

\textbf{Step 3: The Spectral/Centrifugal Barrier (The Starvation).} We
invoke \textbf{Theorem 6.3} and \textbf{Lemma 6.4}. We have proven that
for any configuration attempting to collapse (focusing), the
swirl-induced spectral/centrifugal barrier creates a positive pressure
potential \(Q \sim r^{-2}\) (resulting in a force \(\sim r^{-3}\)). This
barrier opposes the inflow. Specifically, the energy equation in the
renormalized frame shows that the work required to push fluid against
the centrifugal barrier exceeds the inertial kinetic energy available in
the infall: \[ \text{Work}_{barrier} > \text{Energy}_{kinetic} \]
Consequently, the radial velocity \(V_r\) is suppressed near the core.
The ``pipe'' feeding energy to the singularity is effectively clogged.

\textbf{Conclusion (Capacity-Flux Contradiction).} The Type III
configuration fails because of a dimensional mismatch: 1. \textbf{Too
Thin:} The CKN theorem forces the singularity to be 1D (filamentary). 2.
\textbf{Too Coercive to Feed:} The spectral/centrifugal barrier prevents
the radial flux required to pump energy through such a narrow
constriction.

Unlike the Euler equations, where the absence of a viscous scale allows
``wild solutions'\,' with anomalous dissipation on sets of positive
measure, the Navier--Stokes viscosity enforces the CKN geometry, and the
geometry in turn enforces the spectral coercivity barrier. As a result,
any putative anomalous dissipation rate must vanish,
\(\varepsilon_{anom}=0\), completing the proof.

:::\{prf:theorem\} The Starvation Theorem :label:
the-the-starvation-theorem

Let \(\Sigma\) be the support of a potential Type III singularity. If
the flow satisfies the Navier-Stokes equations, then
\(\varepsilon_{anom} = 0\). The singularity cannot sustain anomalous
dissipation.

(sec-exclusion-of-fractal-regimes-variational-mechanism)= \#\# 8.4.
Exclusion of Fractal Regimes (Variational Mechanism)

The final theoretical loophole in the high-entropy analysis concerns the
temporal dynamics of the \textbf{High-Entropy} regime. While the
geometric depletion inequality and the CKN theorem constrain the
Hausdorff dimension of the terminal singular set in physical space, they
do not explicitly forbid a \textbf{Type IV configuration}: a short-time
excursion into a spectrally dense state immediately prior to \(T^*\). In
such a scenario one would attempt to transfer sufficient energy to small
scales in a brief time interval to overcome the depletion inequality
before the viscous smoothing applies.

We resolve this paradox through a \textbf{variational principle}:
fractal configurations are energetically suboptimal for singularity
formation. Standard concentration-compactness analysis combined with
elliptic bootstrapping establishes that extremizers of the nonlinear
efficiency functional are smooth (\(C^\infty\)). Since fractal states
strictly cannot achieve the maximal efficiency required to overcome
viscous dissipation, Type IV blow-up is impossible.

(sec-gevrey-evolution-and-the-analyticity-radius)= \#\#\# 8.4.1. Gevrey
Evolution and the Analyticity Radius

We track the singularity via the radius of analyticity \(\tau(t)\). A
finite-time singularity at \(T^*\) corresponds to the collapse
\(\lim_{t \to T^*} \tau(t) = 0\). We define the Gevrey norm
\(\|\cdot\|_{\tau, s}\) for \(s \ge 1/2\):
\[ \| \mathbf{u} \|_{\tau, s}^2 = \sum_{\mathbf{k} \in \mathbb{Z}^3} |\mathbf{k}|^{2s} e^{2\tau |\mathbf{k}|} |\hat{\mathbf{u}}(\mathbf{k})|^2 \]
The evolution of the Gevrey enstrophy (\(s=1\)) is governed by:
\[ \frac{1}{2} \frac{d}{dt} \|\mathbf{u}\|_{\tau, 1}^2 + \nu \|\mathbf{u}\|_{\tau, 2}^2 - \dot{\tau} \|\mathbf{u}\|_{\tau, 3/2}^2 = -\langle B(\mathbf{u}, \mathbf{u}), A^{2\tau} A \mathbf{u} \rangle \]
where \(A = \sqrt{-\Delta}\) is the Stokes operator. To prevent the
collapse of \(\tau(t)\) (and thus ensure regularity), we must show that
the dissipative term \(\nu \|\mathbf{u}\|_{\tau, 2}^2\) dominates the
nonlinear term.

:::\{prf:definition\} The Spectral Coherence Functional :label:
def-the-spectral-coherence-functional

We define the \textbf{Spectral Coherence} \(\Xi[\mathbf{u}]\) as the
dimensionless ratio of the nonlinear energy transfer to the maximal
dyadic capacity allowed by the Sobolev inequalities.
\[ \Xi[\mathbf{u}] = \frac{|\langle B(\mathbf{u}, \mathbf{u}), A^{2\tau} A \mathbf{u} \rangle|}{C_{Sob} \|\mathbf{u}\|_{\tau, 1} \|\mathbf{u}\|_{\tau, 2}^2} \]
where \(C_{Sob}\) is the optimal constant for the interpolation
inequality in the ``worst-case'' alignment (e.g., a 1D filament or
Burgers vortex). * \textbf{Coherent States (\(\Xi \approx 1\)):}
Geometries where Fourier phases align to maximize triadic interactions
(e.g., tubes, sheets). * \textbf{Incoherent States (\(\Xi \ll 1\)):}
Geometries with broad-band, isotropic spectra where phase cancellation
occurs in the convolution sum (e.g., fractal turbulence).

(sec-the-efficiency-gap-for-fractal-states)= \#\#\# 8.4.2. The
Efficiency Gap for Fractal States

We now prove that the Type IV configuration (High Entropy, i.e.~profiles
in the fractal stratum \(\Omega_{\mathrm{Frac}}\)) implies
\(\Xi \ll \Xi_{max}\), which dynamically arrests the collapse of
\(\tau\).

The key insight is that Type IV blow-up requires maximizing the
nonlinear term
\[ |\langle B(\mathbf{u}, \mathbf{u}), A^{2\tau} A \mathbf{u} \rangle| \]
relative to the dissipative capacity. However, the extremizers of this
functional are characterized by the Euler-Lagrange equation derived in
Section 8.5, which yields a fourth-order elliptic system. Standard
elliptic bootstrapping implies any extremizer is \(C^\infty\).

For fractal states with broad-band spectra and Fourier dimension
\(D_F > 2\), the triadic interactions experience destructive
interference:
\[ \left| \sum_{\mathbf{p}+\mathbf{q}=\mathbf{k}} \hat{\mathbf{u}}_\mathbf{p} \otimes \hat{\mathbf{u}}_\mathbf{q} \right| \ll \sum |\hat{\mathbf{u}}_\mathbf{p}| |\hat{\mathbf{u}}_\mathbf{q}| \]
This phase decoherence effect, combined with the isotropic energy
distribution, yields \(\Xi[\mathbf{u}_{fractal}] \ll \Xi_{max}\).

\textbf{Remark 8.4.} If the phases fail to randomize (constructive
alignment), the flow is effectively coherent (Type I/II) and falls under
the defocusing and coercivity constraints of Sections 4 and 6. Thus the
flow cannot simultaneously evade the geometric constraints by becoming
fractal and evade the depletion constraint by remaining coherent.

\textbf{Theorem 8.4.2 (Gevrey Inertia and the Speed Limit on
Transitions).} The radius of analyticity \(\tau(t)\) obeys the
differential inequality:
\[ \dot{\tau}(t) \ge \nu - C_{Sob} \|\mathbf{u}\|_{\tau, 1} \cdot \Xi[\mathbf{u}(t)] \]

Crucially, the rate of Gevrey recovery \(\dot{\tau}\) is bounded by the
\textbf{instantaneous} coherence \(\Xi[\mathbf{u}(t)]\). Since \(\Xi\)
admits a quantitative variational deficit away from
\(\mathcal{M}_{opt}\) (Theorem 8.5.3), any trajectory attempting to
transit from high-entropy to coherent regimes incurs a strictly positive
dissipation penalty that exceeds the nonlinear gain during the
transition interval.

\begin{itemize}
\item
  \textbf{Case 1 (Near Extremizers):} If
  \(\text{dist}(\mathbf{u}, \mathcal{M}_{opt}) < \epsilon\), then
  \(\Xi \approx \Xi_{max}\). The flow is nearly coherent and smooth.
  However, these configurations are ruled out by the defocusing and
  coercivity constraints (Sections 4 and 6).
\item
  \textbf{Case 2 (Far from Extremizers):} If
  \(\text{dist}(\mathbf{u}, \mathcal{M}_{opt}) > \delta\), then by the
  quantitative stability (Theorem 8.5.3):
  \[ \Xi[\mathbf{u}] \leq \Xi_{max} - \kappa\delta^2 \] This efficiency
  deficit ensures:
  \[ \|\mathbf{u}\|_{\tau, 1} \cdot \Xi[\mathbf{u}] < \|\mathbf{u}\|_{\tau, 1} \cdot (\Xi_{max} - \kappa\delta^2) < \nu \]
  for appropriate bounds. Thus \(\dot{\tau} > 0\).
\item
  \textbf{Case 3 (Dynamic Transitions):} Consider a trajectory
  attempting to oscillate between fractal (\(\tau \approx 0\)) and
  coherent (\(\tau > 0\)) states. The transition requires traversing the
  ``valley of inefficiency'\,' where
  \(\Xi[\mathbf{u}] < \Xi_{max} - \kappa\delta^2\) for the entire
  transit time. During this interval, \(\dot{\tau} > 0\) forces
  regularity recovery, preventing the oscillation from completing before
  viscous dissipation dominates.
\end{itemize}

\textbf{Remark 8.4.3 (The Trap of Sub-Optimality: Clarification on
Maximization).} This argument does not assume that the flow must evolve
toward an efficiency maximizer to sustain a singularity. Rather, there
is a dynamic dichotomy based on an efficiency gap: 1.
\textbf{Sub-optimal regime
(\(\Xi[\mathbf{u}] \le \Xi_{\max} - \delta\)).} The nonlinearity is
inefficient. By Theorem 8.4.2, \(\dot{\tau} > 0\) because viscous
dissipation outweighs the depleted stretching; analyticity recovers and
a singularity cannot persist. 2. \textbf{Near-optimal regime
(\(\Xi[\mathbf{u}] \approx \Xi_{\max}\)).} To avoid Gevrey recovery, the
flow must enter this regime. Doing so forces convergence in \(H^1_\rho\)
to the extremizer manifold \(\mathcal{M}\) (Definition 8.5.6 and Lemma
8.4.4). By Proposition 8.5.1, elements of \(\mathcal{M}\) are smooth
(\(C^\infty\)) and geometrically coherent. Once in this regime, the
geometric obstructions of Sections 4 and 6 exclude singularity
formation.

Thus failure to maximize efficiency triggers regularization via Gevrey
recovery; success in maximizing efficiency triggers regularization via
geometric rigidity. The singularity is trapped between these outcomes.

\textbf{Remark 8.4.4 (The amplitude--efficiency dichotomy and the Type
II stratum).} An objection to the Gevrey recovery inequality \[
\dot{\tau}(t) \ge \nu - C_{Sob}\,\|\mathbf{u}\|_{\tau,1}\,\Xi[\mathbf{u}(t)]
\] is that even with low efficiency (\(\Xi \ll \Xi_{\max}\)), the Gevrey
enstrophy \(\|\mathbf{u}\|_{\tau,1}\) might grow so large that
\(\dot{\tau}<0\). This is ruled out by coupling amplitude to the scaling
regime via the \textbf{Dynamic Normalization Gauge} (Definition 9.2.1):
1. \textbf{Bounded amplitude (Type I / viscous-locked).} As long as the
blow-up remains Type I, global energy bounds keep
\(\|\mathbf{u}\|_{\tau,1}\) at \(O(1)\) in renormalized variables. The
positive efficiency gap \(\Xi_{\max}-\Xi\) for fractal states (Theorem
8.5.10) then forces \(\dot{\tau}>0\). 2. \textbf{Divergent amplitude
(Type II / accelerating).} If \(\|\mathbf{u}\|_{\tau,1}\) were to grow
without bound relative to the viscous scale, the scaling parameter
\(\lambda(t)\) would decouple from the Type I rate
(\(Re_\lambda\to\infty\)), placing the trajectory in the accelerating
stratum \(\Omega_{\mathrm{Acc}}\) (Definitions 6.1.6, 9.0.1). The
refined Type II exclusion (Proposition 6.1.6, Theorem 9.3) shows
\(\Omega_{\mathrm{Acc}}\) is empty: sustaining such growth would force
\(\int_0^{T^*}\lambda(t)^{-1}dt=\infty\), violating the global energy
bound.

Conclusion: the flow cannot bypass the efficiency constraint by
amplitude blow-up. If amplitude stays finite (Type I), inefficiency
yields \(\dot{\tau}>0\); if amplitude diverges (Type II), the
mass-flux/energy mechanism excludes the trajectory.

In particular, any divergence of the Gevrey amplitude
\(\|\mathbf{u}\|_{\tau,1}\) signals decoupling from the viscous scale,
accelerates \(\lambda(t)\), and transfers the trajectory into the Type
II stratum \(\Omega_{\mathrm{Acc}}\); Theorem 9.3 then excludes blow-up
by mass-flux capacity.

\textbf{Conclusion:} The Type IV scenario described above is forbidden
by a \textbf{variational gap}. Since the extremizers of the nonlinear
efficiency functional are smooth (Section 8.5), fractal configurations
are energetically suboptimal. The nonlinearity cannot be simultaneously
\textbf{geometry-breaking} (to escape defocusing or spectral coercivity)
and \textbf{energy-efficient} (to overcome viscosity). The efficiency
deficit of fractal states ensures the analyticity radius \(\tau(t)\)
recovers, preventing blow-up.

:::\{prf:theorem\} Roughness is Inefficient :label:
the-roughness-is-inefficient

Let \(\mathbf{u}\) be a divergence-free vector field attempting Type IV
blow-up (fractal excursion, hence lying in the fractal spectral class of
Definition 8.5.9 and the stratum \(\Omega_{\mathrm{Frac}}\)). Then: 1.
The maximal efficiency \(\Xi_{max}\) is achieved by smooth profiles
(established in Section 8.5 via elliptic regularity) 2. Fractal
configurations have strictly suboptimal efficiency:
\(\Xi[\mathbf{u}_{fractal}] < \Xi_{max} - \delta\) for some
\(\delta > 0\) 3. This efficiency gap prevents the collapse of
analyticity radius \(\tau(t)\)

We now formalize the dichotomy between fractal and coherent behaviour
along a blow-up sequence.

:::\{prf:definition\} Fractal and Coherent Branches :label:
def-fractal-and-coherent-branches

Let \(\mathbf{V}(\cdot,s)\) denote the renormalized profile associated
with a putative singularity, and let \[
Z[\mathbf{V}(s)] := \int_{\mathbb{R}^3} |\boldsymbol{\omega}(y,s)|^2\,|\nabla\xi(y,s)|^2\,dy
\] be the geometric entropy functional of Section 11. We say that the
renormalized orbit follows the

\begin{enumerate}
\def\labelenumi{\arabic{enumi}.}
\item
  \textbf{Fractal/High-Entropy Branch} if there exists a sequence
  \(s_n\to\infty\) such that \[
  Z[\mathbf{V}(s_n)] \to \infty
  \quad\text{and}\quad
  \Xi[\mathbf{V}(s_n)] \le \Xi_{\max} - \delta
  \] for some fixed \(\delta>0\).
\item
  \textbf{Coherent Branch} if for every blow-up sequence
  \(s_n\to\infty\) there is a subsequence (still denoted \(s_n\)) such
  that \(Z[\mathbf{V}(s_n)]\) remains bounded and \[
  \Xi[\mathbf{V}(s_n)] \to \Xi_{\max}.
  \]
\end{enumerate}

In the fractal branch the flow remains at a fixed positive variational
distance from the extremizer manifold, while in the coherent branch it
is forced asymptotically towards \(\mathcal{M}_{opt}\).

\[
Z[\mathbf{V}(s_n)] \to \infty
\quad\text{and}\quad
\Xi[\mathbf{V}(s_n)] \le \Xi_{\max}-\delta_*
\] hold simultaneously. Equivalently, for every \(\delta>0\) and every
blow-up sequence \(s_n\to\infty\) there exists a subsequence (still
denoted \(s_n\)) such that \[
\Xi[\mathbf{V}(s_n)] > \Xi_{\max}-\delta.
\]

Fix an arbitrary blow-up sequence \(s_n\to\infty\). Choosing
\(\delta=1/k\) and passing to a diagonal subsequence in \(k\), we may
find a subsequence (which we continue to denote by \(s_n\)) such that \[
\Xi[\mathbf{V}(s_n)] \to \Xi_{\max} \quad\text{as } n\to\infty.
\] By the Concentration--Compactness Principle 8.5.7 and Theorems
8.5.3--8.5.5 (quantitative stability and global compactness), there
exists a sequence of symmetries \(g_n\in G\) and an extremizer
\(\phi\in\mathcal{M}\) such that \[
\mathcal{U}_{g_n}\mathbf{V}(\cdot,s_n) \to \phi \quad\text{in } H^1_\rho.
\] Since the \(H^1_\rho\)-norm controls the vorticity and its
derivatives in \(L^2_\rho\), and by Proposition 8.5.1 and Corollary
8.5.1.1 the extremizer \(\phi\) is smooth with bounded derivatives of
all orders, we can transfer this convergence to the vorticity and
direction fields: \[
\boldsymbol{\omega}(\cdot,s_n) \to \boldsymbol{\omega}_\phi \quad\text{in }L^2_\rho,\qquad
\nabla\xi(\cdot,s_n) \to \nabla\xi_\phi \quad\text{in }L^2_{\mathrm{loc}},
\] after composition with the same symmetries. In particular, the
convergence of \(\boldsymbol{\omega}(\cdot,s_n)\) in \(L^2_\rho\) and
the uniform boundedness of \(\nabla\xi_\phi\) on compact sets (Lemma
11.0.4) imply, by dominated convergence, that \[
Z[\mathbf{V}(s_n)]
 = \int_{\mathbb{R}^3} |\boldsymbol{\omega}(y,s_n)|^2\,|\nabla\xi(y,s_n)|^2\,dy
 \to \int_{\mathbb{R}^3} |\boldsymbol{\omega}_\phi(y)|^2\,|\nabla\xi_\phi(y)|^2\,dy
 = Z[\phi] < \infty.
\] Thus along this subsequence the geometric entropy
\(Z[\mathbf{V}(s_n)]\) remains bounded, and the renormalized profiles
converge (modulo symmetries) in \(H^1_\rho\) to \(\phi\in\mathcal{M}\),
as claimed. \(\hfill\square\)

\textbf{Remark 8.4.5 (Handoff to the Coherent Branch).} Lemma 8.4.4
shows that failure of the fractal/high-entropy mechanism forces the flow
into the coherent branch: any blow-up sequence that is not permanently
trapped in the entropy-dominated regime must, after renormalization and
modulation by symmetries, converge to the smooth extremizer manifold.
This is precisely the setting for the geometric (Section 4), spectral
(Section 6), and variational (Section 11) exclusions of coherent Type I
singularities.

(sec-the-regularity-of-nonlinear-extremizers)= \#\# 8.5. The Regularity
of Nonlinear Extremizers

We establish the mathematical foundation for the variational exclusion
of fractals. Through concentration-compactness analysis and elliptic
bootstrapping, we prove that extremizers of the nonlinear efficiency
functional are necessarily smooth (\(C^\infty\)). This regularity result
is the cornerstone of our reduction from four hypotheses to one: it
automatically excludes fractal blow-up without requiring additional
assumptions about phase decoherence or symmetry.

The key insight is that while we conjecture extremizers are also
\textbf{symmetric} (tubes or sheets), for the purpose of excluding Type
IV blow-up, \textbf{regularity alone is sufficient}. Smoothness implies
the flow occupies the Low-Entropy Stratum, where geometric constraints
apply.

We do not assume putative singularities are smooth. Rough candidates are
shown to be variationally inefficient and are eliminated by Gevrey
recovery (Sections 8.4--8.6). If a trajectory approaches smooth
extremizers, elliptic bootstrapping makes them \(C^\infty\), and the
geometric/spectral constraints of Sections 4 and 6 apply. Inefficiency
removes rough profiles; rigidity removes smooth profiles.

\textbf{Definition 8.5.1 (The Extremal Set).} The extremal set is \[
\mathcal{M} := \{ u \in \mathcal{S} : \Xi[u] = \Xi_{\max} \}.
\] We do not assume \(\mathcal{M}\) is non-empty; the dichotomy below
covers both possibilities.

\textbf{Theorem 8.5.A (The Existence Dichotomy).} Either \(\mathcal{M}\)
is non-empty and contains smooth functions (Case A), or any maximizing
sequence concentrates into a defect measure / vanishes (Case B).

\textbf{Proof of Regularity in Case B.} If Case B holds, then for any
admissible smooth solution \(u\), \(\Xi[u]\) is strictly bounded away
from the theoretical supremum of the rough limit. The Gevrey-transit
estimate (Theorem 8.6.5) applies with \(\delta > 0\), forcing
\(\dot{\tau}>0\) and preventing blow-up. Case A is treated by geometric
rigidity of the coherent branch in Sections 4, 6, and 11.

(sec-functional-framework-and-normalization)= \#\#\# 8.5.1. Functional
Framework and Normalization

We establish the precise functional setting for the variational problem,
following the abstract geometric-analytic framework.

\textbf{Definition 8.5.2 (The Hilbert Space).} Let
\(X = \dot{H}^1_\sigma(\mathbb{R}^3)\) be the homogeneous Sobolev space
of divergence-free vector fields:
\[ X := \text{closure of } C_c^\infty(\mathbb{R}^3; \mathbb{R}^3) \text{ with } \nabla \cdot u = 0 \text{ in } \|u\|_X^2 := \int_{\mathbb{R}^3} |\nabla u|^2 \, dx \]
This is a real Hilbert space with inner product
\(\langle u, v \rangle_X = \int \nabla u : \nabla v \, dx\).

\textbf{Definition 8.5.3 (The Unit Sphere).} The constraint manifold is
the unit sphere in \(X\):
\[ \mathcal{S} := \{ u \in X : \|u\|_X = 1 \} = \{ u \in X : \|\nabla u\|_{L^2} = 1 \} \]
This is a smooth Hilbert manifold with tangent space at
\(\phi \in \mathcal{S}\) given by:
\[ T_\phi \mathcal{S} = \{ h \in X : \langle h, \phi \rangle_X = 0 \} \]

\textbf{Definition 8.5.4 (The Efficiency Functional).} For
\(u \in \mathcal{S}\), we define the nonlinear efficiency functional:
\[ \Xi[u] = |\langle B(u, u), Au \rangle| \] where: -
\(B(u,v) = \mathbb{P}[(u \cdot \nabla)v]\) is the Navier-Stokes bilinear
form - \(A = (-\Delta)^{1/2}\) is the Stokes operator - \(\mathbb{P}\)
is the Helmholtz-Leray projection onto divergence-free fields

The variational problem is to find:
\[ \Xi_{\max} := \sup_{u \in \mathcal{S}} \Xi[u] \]

\textbf{Definition 8.5.5 (The Symmetry Group).} Let \(G\) be the Lie
group generated by: 1. \textbf{Spatial translations:}
\(T_h u(x) = u(x - h)\) for \(h \in \mathbb{R}^3\) 2.
\textbf{Rotations:} \(R_Q u(x) = Qu(Q^T x)\) for \(Q \in SO(3)\) 3.
\textbf{Critical scaling:} \(u_\lambda(x) = \lambda^{1/2} u(\lambda x)\)
for \(\lambda > 0\)

The functional \(\Xi\) is invariant under \(G\):
\[ \Xi[\mathcal{U}_g u] = \Xi[u] \quad \text{for all } g \in G, \, u \in \mathcal{S} \]
where \(\mathcal{U}_g\) denotes the unitary action of \(g\) on \(X\).

\textbf{Definition 8.5.6 (The Extremal Manifold).} Case A of Theorem
8.5.A yields a (possibly empty) manifold of extremizers:
\[ \mathcal{M} = \{ \mathcal{U}_g \phi : g \in G, \phi \text{ is an extremizer} \}. \]
By the symmetry of \(\Xi\), if \(\phi \in \mathcal{M}\) is an
extremizer, then the entire \(G\)-orbit belongs to \(\mathcal{M}\).

\textbf{Remark 8.5.1b (Variational dichotomy and the role of
existence).} The optimization of \(\Xi\) admits a dichotomy: 1.
\textbf{Case A (extremizers exist).} A smooth extremizer exists, the
maximizing sequence converges to \(\mathcal{M}\), and the limit profile
is subject to the geometric rigidity constraints of Sections 4, 6, and
11 (twist, swirl, defocusing). 2. \textbf{Case B (extremizers absent).}
If maximizing sequences lose compactness (vanishing or dichotomy) or
converge to a singular object, \(\mathcal{M}\) is empty/inaccessible.
Then \[
   \limsup_{t\to T^*} \Xi[\mathbf{u}(t)] < \Xi_{\max},
   \] i.e., a global efficiency gap. The trajectory lies in the
fractal/high-entropy stratum \(\Omega_{\mathrm{Frac}}\); by Theorem
8.4.1 and Theorem 8.6.5 the Gevrey recovery mechanism enforces
\(\dot{\tau}>0\), precluding blow-up.

Thus: if the extremal set is populated, regularity follows by geometric
rigidity; if it is empty, regularity follows by variational
inefficiency.

\textbf{Proposition 8.5.1 (Regularity of Extremizers).} Let
\(\phi \in \mathcal{M}\) be an extremizer of \(\Xi\) on \(\mathcal{S}\).
Then \(\phi\) is a smooth, rapidly decaying solution of the
Euler--Lagrange system associated with \(\Xi\); in particular
\(\phi \in C^\infty_b(\mathbb{R}^3)\).

\emph{Proof.} The first variation of \(\Xi\) on \(\mathcal{S}\) shows
that \(\phi\) satisfies the Euler--Lagrange equation \[
\mathrm{d}\Xi[\phi](h) = 0 \quad \text{for all } h\in T_\phi\mathcal{S}.
\] By standard Lagrange multiplier theory on the constraint manifold
\(\mathcal{S}\), there exists a scalar \(\lambda\in\mathbb{R}\) such
that \[
\mathrm{d}\Xi[\phi](h) = \lambda \langle h,\phi\rangle_X \quad \text{for all } h\in X.
\] Writing out \(\mathrm{d}\Xi[\phi](h)\) explicitly (see Section 8.5.2)
and using the definition of \(X\) and \(\mathcal{S}\), we obtain a weak
formulation of the form \[
\int_{\mathbb{R}^3} \Big(\nu\nabla\phi : \nabla h + \mathcal{N}(\phi,\nabla\phi)\cdot h - \lambda A\phi \cdot h\Big)\,dx = 0
\] for all divergence-free test functions
\(h\in C_c^\infty(\mathbb{R}^3;\mathbb{R}^3)\), where \(\mathcal{N}\) is
at most quadratic in \((\phi,\nabla\phi)\) and \(A=(-\Delta)^{1/2}\) is
the Stokes operator. Integrating by parts in the first term and
projecting onto the divergence-free subspace via the Helmholtz--Leray
projection \(\mathbb{P}\) yields the stationary Stokes-type system \[
-\nu\Delta \phi + \mathbb{P}\,\mathcal{N}(\phi,\nabla\phi) = \lambda A\phi,
 \qquad \nabla\cdot\phi = 0
\] in the sense of distributions. Since \(\phi\in X=\dot{H}^1_\sigma\),
the left-hand side belongs to \(H^{-1}\) and \(A\phi\in H^{-1}\) as
well, so this is a semi-linear elliptic system with right-hand side in
\(H^{-1}\).

We first obtain local \(H^2\) regularity. Writing the equation as \[
-\nu\Delta\phi + \nabla P = F,\qquad \nabla\cdot\phi=0,
\] with \[
F := \lambda A\phi - \mathbb{P}\,\mathcal{N}(\phi,\nabla\phi),
\] we note that \(A\phi\in H^{-1}\) and, since \(\phi\in H^1\) and
\(\mathcal{N}\) is at most quadratic in \((\phi,\nabla\phi)\), we have
\(\mathcal{N}(\phi,\nabla\phi)\in L^{3/2}_{\mathrm{loc}}\) and thus
\(F\in H^{-1}_{\mathrm{loc}}\). By standard elliptic regularity for the
Stokes system (see Galdi {[}@bianchi1991{]}, Theorem X.1.1), any weak
solution with \(F\in H^{-1}\) belongs to
\(H^2_{\mathrm{loc}}(\mathbb{R}^3)\) and \(P\in H^1_{\mathrm{loc}}\).

With \(\phi\in H^2_{\mathrm{loc}}\), Sobolev embedding in three
dimensions implies \(\phi\in L^\infty_{\mathrm{loc}}\) and hence
\((\phi,\nabla\phi)\in L^p_{\mathrm{loc}}\) for all \(p<\infty\). This
improves the regularity of \(F\), and another application of elliptic
regularity upgrades \(\phi\) to \(H^m_{\mathrm{loc}}\) for all
\(m\ge 2\) by iterating the argument. Consequently
\(\phi\in C^\infty(\mathbb{R}^3)\) by Sobolev embedding.

To see rapid decay at infinity, we combine the unit-sphere constraint
\(\|\nabla\phi\|_{L^2}=1\) with the Gaussian-weighted structure of the
renormalized energy space \(H^1_\rho\) (Section 6.1). The elliptic
equation above can be viewed as a perturbation of the
Ornstein--Uhlenbeck operator \(-\nu\Delta + \tfrac{1}{4}|x|^2\) in
weighted \(L^2_\rho\), and standard spectral theory for such operators
(see, e.g., Hermite expansion arguments in Section 8.6) implies that any
\(H^1_\rho\) solution decays faster than any polynomial at infinity.
Since \(\mathcal{M}\) consists of extremizers normalized in this
weighted space, the decay is uniform across \(\mathcal{M}\). Thus
\(\phi\in C^\infty_b(\mathbb{R}^3)\) with rapid decay, as claimed.
\(\hfill\square\)

\textbf{Remark 8.5.1a (The singular extremizer fail-safe).}
Bootstrapping the Euler--Lagrange system to full \(C^\infty\) regularity
is non-trivial in the supercritical regime, but the exclusion argument
does not hinge solely on smoothness. We distinguish two cases for a
candidate extremizer \(\phi\): 1. \textbf{Smooth case
(\(\phi\in C^\infty\)).} The geometric exclusions of Sections 4, 6, and
11 (including the twist bounds of Lemma 11.0.4) apply directly, ruling
out singularity formation. 2. \textbf{Singular case
(\(\phi\notin C^\infty\)).} If \(\phi\) had a singularity, its
high-frequency tail would incur a dissipation penalty and render it
variationally suboptimal. Smoothing the tail strictly increases \(\Xi\),
so \(\Xi[\phi] < \Xi_{\max}\). A blow-up sequence with
\(\Xi[\mathbf{V}(s)]\to\Xi_{\max}\) cannot converge to such a profile;
by the transit-cost analysis of Section 8.6 this efficiency deficit
forces \(\dot{\tau}>0\), so the trajectory cannot ``sit on'\,' a
singular extremizer. Thus the only variational maximizers relevant to
blow-up are smooth, and the geometric obstructions apply.

\textbf{Lemma 8.5.2 (Finite-dimensionality of \(\mathcal{M}\)).} In Case
A of Theorem 8.5.A and by Proposition 8.5.1, \(\mathcal{M}\) is a
finite-dimensional embedded \(C^\infty\) submanifold of \(\mathcal{S}\).
The dimension equals that of \(G\) (at most 7: 3 translations + 3
rotations + 1 scaling).

\emph{Proof.} This follows from the fact that \(G\) acts smoothly and
freely on \(\mathcal{M}\), making it a principal \(G\)-bundle.
\(\hfill\square\)

\textbf{Corollary 8.5.1.1 (Uniform Gradient Bounds for Extremizers).}
Since any extremizer \(\phi \in \mathcal{M}\) is a solution to the
elliptic Euler-Lagrange system with smooth coefficients (Case A of
Theorem 8.5.A), \(\phi\) is \(C^\infty_b(\mathbb{R}^3)\). Consequently,
all higher-order derivatives are uniformly bounded in the renormalized
frame:
\[ \|\nabla^k \phi\|_{L^\infty(\mathbb{R}^3)} \leq C_k(\Xi_{\max}) < \infty \quad \text{for all } k \geq 1 \]

This implies that extremizers possess a minimum characteristic length
scale of variation \(\ell_{\min} > 0\) that cannot vanish relative to
the blow-up scale. On any region where the vorticity magnitude is
bounded away from zero, \(|\boldsymbol{\omega}| \ge \delta>0\), the
direction field \(\xi = \boldsymbol{\omega}/|\boldsymbol{\omega}|\)
satisfies \[
\|\nabla \xi\|_{L^\infty(\{|\boldsymbol{\omega}|\ge\delta\})}
 \lesssim \frac{\|\nabla \boldsymbol{\omega}\|_{L^\infty}}{\delta},
\] so for smooth, non-trivial profiles with a non-vanishing core, the
internal twist in the energy-carrying region is uniformly bounded.

\emph{Proof.} The Euler-Lagrange equation for extremizers is a
fourth-order elliptic system with analytic coefficients. By standard
elliptic regularity theory and the rapid decay of \(\phi\), we obtain:
1. \(\phi \in C^\infty(\mathbb{R}^3)\) from bootstrapping 2. The
Schauder estimates give
\(\|\nabla^k \phi\|_{L^\infty} \leq C_k \|\phi\|_{H^{k-1}}\) for each
\(k\) 3. Since \(\phi \in \mathcal{S}\) (unit sphere) and decays
rapidly, all Sobolev norms are finite 4. Therefore, all derivatives are
bounded uniformly on \(\mathbb{R}^3\)

The minimum length scale \(\ell_{\min} \sim 1/\max_k C_k^{1/k}\)
provides a resolution limit below which the extremizer cannot vary.
\(\hfill\square\)

(sec-spectral-analysis-and-non-degeneracy-hypotheses)= \#\#\# 8.5.2.
Spectral Analysis and Non-Degeneracy Hypotheses

We derive the Euler-Lagrange equation for extremizers and state the
crucial spectral hypotheses that enable quantitative stability.

\textbf{The Hessian and Linearized Operator}

For \(\phi \in \mathcal{M}\), the second variation of \(\Xi\) at
\(\phi\) defines a bounded self-adjoint operator:
\[ L_\phi : T_\phi \mathcal{S} \to T_\phi \mathcal{S} \] where
\(T_\phi \mathcal{S} = \{ h \in X : \langle h, \phi \rangle_X = 0 \}\)
is the tangent space.

The quadratic form associated with the Hessian is:
\[ Q_\phi(h) = \frac{1}{2} \mathrm{d}^2 \Xi[\phi](h,h) = \frac{1}{2} \langle L_\phi h, h \rangle_X \]

\textbf{Lemma 8.5.3 (Symmetry Kernel).} Let \(\phi \in \mathcal{M}\) and
let \(v \in T_\phi \mathcal{M}\) be a tangent vector generated by the
symmetry group \(G\). Then: \[ L_\phi v = 0 \] More generally,
\(\mathrm{d}^2 \Xi[\phi](v,h) = 0\) for all
\(h \in T_\phi \mathcal{S}\).

\emph{Proof.} Since \(\Xi\) is invariant under \(G\) and \(\phi\) is a
critical point, the Hessian annihilates all symmetry directions. See
Appendix A for details. \(\hfill\square\)

\textbf{Hypothesis H2 (Non-Degeneracy Modulo Symmetries).} For each
\(\phi \in \mathcal{M}\): 1. \textbf{Exact kernel:}
\(\ker L_\phi = T_\phi \mathcal{M}\) (the kernel consists precisely of
symmetry directions) 2. \textbf{Strict negativity:} The restriction of
\(L_\phi\) to \((T_\phi \mathcal{M})^\perp\) is strictly negative
definite:
\[ \langle L_\phi h, h \rangle_X < 0 \quad \text{for all } 0 \neq h \in (T_\phi \mathcal{M})^\perp \]

\emph{Physical Justification:} This is the generic non-degeneracy
condition for isolated extremizers. It states that the extremizer is a
strict local maximum modulo symmetries - there are no ``flat''
directions except those generated by the invariance group.

\textbf{Hypothesis H3 (Isolation of Zero Eigenvalue).} For each
\(\phi \in \mathcal{M}\), zero is an isolated point in the spectrum of
\(L_\phi\).

\emph{Mathematical Justification:} In the Navier-Stokes setting,
\(L_\phi\) can be written as: \[ L_\phi = -\mu I + K_\phi \] where
\(\mu > 0\) and \(K_\phi\) is a compact operator (due to the rapid decay
of \(\phi\) and smoothing properties of the Stokes operator). By Weyl's
theorem on essential spectra, the essential spectrum is \(\{-\mu\}\),
making zero an isolated eigenvalue of finite multiplicity.

\textbf{Lemma 8.5.4 (Spectral Gap on Transversal Directions).} Under
Hypotheses H2 and H3, there exists \(\lambda_\phi > 0\) such that:
\[ \langle L_\phi h, h \rangle_X \leq -\lambda_\phi \|h\|_X^2 \quad \text{for all } h \in (T_\phi \mathcal{M})^\perp \]

\emph{Proof.} By the spectral theorem for self-adjoint operators and the
isolation of zero, the spectrum on \((T_\phi \mathcal{M})^\perp\) is
bounded away from zero. See Appendix A for the complete argument.
\(\hfill\square\)

\textbf{Compactness Principle 8.5.7 (Concentration--Compactness).} Let
\((u_n) \subset \mathcal{S}\) be a sequence with
\(\Xi[u_n] \to \Xi_{\max}\). Then there exist a subsequence, a sequence
\(g_n \in G\), and some \(\phi \in \mathcal{M}\) such that:
\[ \mathcal{U}_{g_n} u_n \to \phi \quad \text{strongly in } X \]

\emph{Note:} This concentration-compactness property was established in
Section 8.4 using profile decomposition techniques.

(sec-quantitative-stability-of-extremizers)= \#\#\# 8.5.3. Quantitative
Stability of Extremizers

We establish the crucial quantitative rigidity that creates a ``valley
of inefficiency'' around the extremizer manifold.

\textbf{Theorem 8.5.4 (Local Quantitative Stability Near an
Extremizer).} Assume Case A of Theorem 8.5.A and spectral conditions
H2--H3. Fix \(\phi \in \mathcal{M}\). Then there exist constants
\(r_\phi > 0\) and \(c_\phi > 0\) such that for every
\(u \in \mathcal{S}\) with \(\|u - \phi\|_X < r_\phi\):
\[ \Xi_{\max} - \Xi[u] \geq c_\phi \cdot \mathrm{dist}_X(u, \mathcal{M})^2 \]
where
\(\mathrm{dist}_X(u, \mathcal{M}) = \inf_{\psi \in \mathcal{M}} \|u - \psi\|_X\).

\emph{Proof.} The proof uses a local chart near \(\phi\), Taylor
expansion of \(\Xi\), and the spectral gap from Lemma 8.5.4. Since the
Hessian has no mixed terms between symmetry and transversal directions
(Lemma 8.5.3), and is strictly negative on the transversal space with
gap \(\lambda_\phi\), we obtain \(c_\phi = 2\lambda_\phi\). See Appendix
A for details. \(\hfill\square\)

\textbf{Theorem 8.5.5 (Global Quantitative Stability - The
Bianchi-Egnell Estimate).} Assume Case A of Theorem 8.5.A together with
H2--H3 and the Concentration--Compactness Principle 8.5.7. Then there
exists a universal constant \(\kappa > 0\) such that:
\[ \Xi_{\max} - \Xi[u] \geq \kappa \cdot \mathrm{dist}_X(u, \mathcal{M})^2 \quad \text{for all } u \in \mathcal{S} \]

This ensures that intermediate states (partially formed tubes,
semi-coherent structures) are strictly suboptimal.

\emph{Proof Strategy:} 1. \textbf{Suppose the theorem fails:} Then there
exists a sequence \((u_n) \subset \mathcal{S}\) with
\(\Xi_{\max} - \Xi[u_n] \leq \varepsilon_n \mathrm{dist}_X(u_n, \mathcal{M})^2\)
where \(\varepsilon_n \to 0\).

\begin{enumerate}
\def\labelenumi{\arabic{enumi}.}
\setcounter{enumi}{1}
\item
  \textbf{Apply concentration-compactness (Principle 8.5.7):} Since
  \(\Xi[u_n] \to \Xi_{\max}\), we can extract \(g_n \in G\) and
  \(\phi \in \mathcal{M}\) such that
  \(v_n := \mathcal{U}_{g_n} u_n \to \phi\) in \(X\).
\item
  \textbf{Use local stability:} For large \(n\),
  \(\|v_n - \phi\|_X < r_\phi\), so Theorem 8.5.4 gives
  \(\Xi_{\max} - \Xi[v_n] \geq c_\phi \mathrm{dist}_X(v_n, \mathcal{M})^2\).
\item
  \textbf{Derive contradiction:} Since \(\Xi\) and distance to
  \(\mathcal{M}\) are \(G\)-invariant, we get
  \(c_\phi \mathrm{dist}_X(v_n, \mathcal{M})^2 \leq \varepsilon_n \mathrm{dist}_X(v_n, \mathcal{M})^2\).
  For \(\varepsilon_n < c_\phi\), this forces
  \(\mathrm{dist}_X(v_n, \mathcal{M}) = 0\), contradicting the
  assumption.
\end{enumerate}

See Appendix A for the complete proof. \(\hfill\square\)

\textbf{Corollary 8.5.6 (The Valley of Inefficiency).} Any trajectory
\(u(t)\) attempting to transition between strata must traverse a region
where: \[ \Xi[u(t)] \leq \Xi_{\max} - \kappa \delta^2 \] where
\(\delta = \min_t \mathrm{dist}_X(u(t), \mathcal{M})\) is the minimal
distance to the extremizer manifold during the transition. This
efficiency deficit persists throughout any reorganization process.

\textbf{Remark 8.5.7 (Comparison with Classical Results).} This is the
Navier-Stokes analogue of the Bianchi-Egnell stability theorem for the
Sobolev inequality. The structure is identical: the symmetry group \(G\)
(translations, rotations, scaling) plays the role of the conformal
group, and \(\Xi\) replaces the Sobolev quotient. The key innovation is
applying this abstract framework to the trilinear efficiency functional.

(sec-fractal-separation-in-fourier-space)= \#\#\# 8.5.4. Fractal
Separation in Fourier Space

We now establish the separation between smooth extremizers and states
with broadband power-law spectra.

\textbf{Dyadic Shell Energies}

For \(u \in X\), define the dyadic shell
\(A_j = \{\xi \in \mathbb{R}^3 : 2^j \leq |\xi| < 2^{j+1}\}\) and the
corresponding shell energy:
\[ e_j(u) = \int_{A_j} |\xi|^2 |\hat{u}(\xi)|^2 \, d\xi \]

The square-root shell amplitudes are:
\[ a_j(u) = \sqrt{e_j(u)} \geq 0, \quad (a_j(u))_{j \in \mathbb{Z}} \in \ell^2(\mathbb{Z}) \]
with \(\|u\|_X^2 = \sum_j a_j(u)^2\).

\textbf{Lemma 8.5.8 (Shell-wise Lower Bound for Distance).} For all
\(u, v \in X\):
\[ \|u - v\|_X^2 \geq \sum_{j \in \mathbb{Z}} (a_j(u) - a_j(v))^2 \]

\emph{Proof.} Apply the triangle inequality in each dyadic shell. See
Appendix B. \(\hfill\square\)

\textbf{Definition 8.5.9 (The Coherent vs.~Fractal Spectrum).} We call a
profile \textbf{coherent} if there exists a nonnegative sequence
\(b \in \ell^2(\mathbb{Z})\) with \(\sum_j b_j^2 = 1\) and an index
\(j(\phi) \in \mathbb{Z}\) such that
\[ a_j(\phi) \leq C b_{j-j(\phi)} \quad \text{for all } j \in \mathbb{Z} \]
for some constant \(C \geq 1\) independent of \(\phi\). Profiles that
fail this single-scale localization are deemed \textbf{fractal}
(multi-scale). We do not assume a priori that extremizers are coherent;
the argument will show that multi-scale candidates are variationally
suboptimal.

For quantitative estimates we use the following spectral class to
capture fractal spreading. Let \(\alpha \in (1,3)\), \(\eta \in (0,1)\),
and \(J_0 \in \mathbb{N}\). We say \(u \in \mathcal{S}\) belongs to the
fractal class \(\mathcal{F}(\alpha, \eta, J_0)\) if there exists an
infinite set \(J \subset \mathbb{Z}\) with \(\inf J \geq J_0\) such
that:
\[ e_j(u) \geq \eta \cdot 2^{-(3-\alpha)j} \quad \text{for all } j \in J \]

\emph{Interpretation:} This corresponds to a Kolmogorov-type power-law
spectrum \(E(k) \sim k^{-\alpha}\). The factor \(2^{-(3-\alpha)j}\)
accounts for the three-dimensional measure and the spectral exponent.

\textbf{Theorem 8.5.10 (Fractal Separation Lemma).} Assume Case A of
Theorem 8.5.A and the coherence condition of Definition 8.5.9. Fix
parameters \(\alpha \in (1,3)\), \(\eta \in (0,1)\), and
\(J_0 \in \mathbb{N}\). Then there exists \(\delta > 0\) such that:
\[ \mathrm{dist}_X(u, \mathcal{M}) \geq \delta \quad \text{for all } u \in \mathcal{F}(\alpha, \eta, J_0) \]

\emph{Proof Strategy:} 1. \textbf{Energy distribution:} By Definition
8.5.9, coherent profiles have energy concentrated in finitely many
shells 2. \textbf{Fractal spreading:} Elements of \(\mathcal{F}\) have
energy spread across infinitely many shells 3. \textbf{Orthogonality:}
The high-frequency tails of fractal states are orthogonal to the
localized extremizers 4. \textbf{Quantitative bound:} Using Lemma 8.5.8,
the shell-wise differences accumulate to give
\(\|u - \phi\|_X \geq \delta\) for all \(\phi \in \mathcal{M}\)

See Appendix B for the complete proof. \(\hfill\square\)

\textbf{Corollary 8.5.11 (The Smoothness-Fractal Efficiency Gap).}
Combining Theorems 8.5.5 and 8.5.10, for any fractal configuration
\(u \in \mathcal{F}(\alpha, \eta, J_0)\):
\[ \Xi[u] \leq \Xi_{\max} - \kappa\delta^2 \]

This provides the quantitative penalty for fractal excursions in the
dynamical argument.

(sec-conclusion-the-variational-exclusion-of-type-iv-bl)= \#\#\# 8.5.5.
Conclusion: The Variational Exclusion of Type IV Blow-up

We synthesize the results to definitively exclude fractal singularities,
i.e.~blow-up profiles lying in the fractal spectral class
\(\mathcal{F}\) and the high-entropy stratum \(\Omega_{\mathrm{Frac}}\)
of Section 12 in the global phase-space partition.

\textbf{Theorem 8.5.12 (No Fractal Blow-up).} Type IV (fractal) blow-up,
corresponding to the fractal/high-entropy stratum
\(\Omega_{\mathrm{Frac}}\) in the global classification, is impossible
for the 3D Navier-Stokes equations.

\emph{Proof.} The argument is a direct consequence of the variational
structure:

\begin{enumerate}
\def\labelenumi{\arabic{enumi}.}
\item
  \textbf{Efficiency requirement:} Any blow-up requires
  \(\Xi[u(t)] \to \Xi_{max}\) as \(t \to T^*\) to overcome viscous
  dissipation
\item
  \textbf{Smoothness of extremizers:} By Theorem 8.5.1, any
  configuration achieving \(\Xi_{max}\) must be \(C^\infty\)
\item
  \textbf{Fractal efficiency gap:} By Theorem 8.5.4, fractal
  configurations satisfy \(\Xi[u_{fractal}] < \Xi_{max} - \delta\)
\item
  \textbf{Gevrey restoration:} From Section 8.4, the efficiency deficit
  implies:
  \[ \frac{d\tau}{dt} \ge \nu - C\|\mathbf{u}\| \cdot (\Xi_{max} - \delta) > 0 \]
  for appropriate bounds on \(\|\mathbf{u}\|\). The analyticity radius
  grows, preventing singularity formation.
\end{enumerate}

\textbf{Conclusion:} The variational principle creates a fundamental
dichotomy: - \textbf{Smooth flows} (approaching extremizers) are
constrained by geometric mechanisms (Sections 4 and 6) - \textbf{Fractal
flows} are energetically inefficient and cannot sustain blow-up

This completes the exclusion of Type IV scenarios without requiring
hypotheses about phase decoherence or extremizer symmetry. The
smoothness of variational extremizers automatically forces any potential
singularity into the coherent stratum, where it must confront the
geometric constraints. \(\hfill \blacksquare\)

(sec-the-transit-cost-inequality-and-dynamic-exclusion)= \#\# 8.6. The
Transit Cost Inequality and Dynamic Exclusion

We address the potential existence of a \textbf{dynamic transient}---a
solution that oscillates indefinitely between high-entropy (fractal) and
low-entropy (coherent) strata without settling into either. We exclude
this scenario by quantifying the strictly positive gain in regularity
required to traverse the distance between these regimes.

(sec-the-gevrey-deficit-coupling)= \#\#\# 8.6.1. The Gevrey-Deficit
Coupling

We first establish the differential link between the variational
efficiency deficit and the growth of the radius of analyticity. We work
in the renormalized frame \((y,s)\) with the Gaussian-weighted measure
\(\rho(y)\).

\textbf{Definition 8.6.1 (Renormalized Gevrey Operator).} Let
\(\tau(s) > 0\) denote the renormalized radius of analyticity. We define
the time-dependent Gevrey operator
\(\mathcal{G}_{\tau(s)} = e^{\tau(s) A^{1/2}}\), where
\(A = -\Delta + \frac{1}{4}|y|^2\) is the linear operator associated
with the Hermite expansion, which is self-adjoint on
\(L^2_\rho(\mathbb{R}^3)\).

\textbf{Lemma 8.6.1 (The Evolution of Analyticity).} Let
\(E_\tau(s) = \frac{1}{2} \|\mathcal{G}_{\tau(s)} \mathbf{V}(\cdot, s)\|_{L^2_\rho}^2\)
be the Gevrey energy. For the norm to remain finite (preventing the
collapse of \(\tau\)), the evolution of \(\tau(s)\) must satisfy the
differential constraint:
\[ \dot{\tau}(s) \|\mathcal{G}_\tau A^{1/2} \mathbf{V}\|_{L^2_\rho} \|\mathcal{G}_\tau \mathbf{V}\|_{L^2_\rho} \ge \nu \|\nabla (\mathcal{G}_\tau \mathbf{V})\|_{L^2_\rho}^2 - |\langle B(\mathbf{V}, \mathbf{V}), A \mathbf{V} \rangle_\tau| \]
where \(B(\cdot, \cdot)\) is the bilinear form and the pairing is in the
Gevrey dual space.

\textbf{Proposition 8.6.2 (The Variational Lower Bound).} By the
Quantitative Stability Theorem (Theorem 8.5.5), the efficiency deficit
of the profile is bounded below by its distance to the manifold of
extremizers \(\mathcal{M}\). Substituting this into the evolution
inequality yields:
\[ \dot{\tau}(s) \ge C_{sob} \|\mathbf{V}\|_{H^1_\rho} \cdot \kappa \cdot \mathrm{dist}_{H^1_\rho}(\mathbf{V}(s), \mathcal{M})^2 \]
where \(\kappa > 0\) is the Bianchi-Egnell stability constant. Since the
solution orbit lies within a global energy ball with
\(\|\mathbf{V}\|_{H^1_\rho} \ge c_0 > 0\) (for non-trivial
singularities), we obtain the simplified bound:
\[ \dot{\tau}(s) \ge \gamma \, \delta(s)^2 \] where
\(\delta(s) := \mathrm{dist}_{H^1_\rho}(\mathbf{V}(s), \mathcal{M})\)
and \(\gamma > 0\) is a uniform constant depending only on the global
energy \(E_0\) and the stability gap.

(sec-phase-space-kinematics)= \#\#\# 8.6.2. Phase Space Kinematics

To convert the instantaneous rate \(\dot{\tau}\) into a total cost, we
must bound the speed at which the trajectory \(\mathbf{V}(s)\) can move
through the function space \(H^1_\rho\).

\textbf{Lemma 8.6.3 (Lipschitz Continuity of the Trajectory).} Let
\(\mathcal{A} \subset H^1_\rho\) be the global attractor for the
Renormalized Navier-Stokes Equation. For any trajectory
\(\mathbf{V}(s) \in \mathcal{A}\), the time derivative is uniformly
bounded. Specifically, the Renormalized Navier-Stokes operator
\(\mathcal{N}(\mathbf{V}) = -\nu \Delta_\rho \mathbf{V} - B(\mathbf{V}, \mathbf{V}) + \mathbf{L}\mathbf{V}\)
maps bounded sets in \(H^1_\rho\) to bounded sets in the dual space
\(H^{-1}_\rho\).
\[ \sup_{s \in \mathbb{R}} \|\partial_s \mathbf{V}\|_{H^{-1}_\rho} \le V_{max} < \infty \]

\textbf{Corollary 8.6.4 (Rate of Geometric Change).} The distance
function \(\delta(s)\) is Lipschitz continuous in time. Its rate of
change is bounded by the flow speed:
\[ \left| \frac{d}{ds} \delta(s) \right| \le \|\partial_s \mathbf{V}\|_{H^{-1}_\rho} \le V_{max} \]
This inequality enforces a ``speed limit'' on reorganization: the flow
cannot jump instantly from a fractal configuration to a coherent one; it
must continuously traverse the intermediate geometries.

(sec-the-transit-cost-integral)= \#\#\# 8.6.3. The Transit Cost Integral

We combine the rate of smoothing (8.6.1) and the speed of reorganization
(8.6.2) to integrate the total regularity gain during a transition.

\textbf{Theorem 8.6.5 (The Transit Cost Inequality).} Consider a
transition interval \([s_{start}, s_{end}]\) where the solution moves
from the ``Fractal Stratum'' (characterized by \(\delta(s) \ge \Delta\))
to the ``Coherent Stratum'' (characterized by
\(\delta(s) \le \epsilon\)). The total increase in the radius of
analyticity \(\Delta \tau = \tau(s_{end}) - \tau(s_{start})\) satisfies:
\[ \Delta \tau \ge \frac{\gamma}{3 V_{max}} (\Delta^3 - \epsilon^3) \]

\emph{Proof.} From Proposition 8.6.2, we have
\(\dot{\tau}(s) \ge \gamma \delta(s)^2\). The total change is
\(\Delta \tau = \int_{s_{start}}^{s_{end}} \dot{\tau}(s) \, ds\). We
change variables from time \(s\) to distance \(\delta\), using the
kinematic bound \(ds \ge \frac{d\delta}{V_{max}}\). Since the trajectory
must traverse the distance from \(\Delta\) down to \(\epsilon\):
\[ \Delta \tau \ge \int_{\epsilon}^{\Delta} \gamma \delta^2 \frac{d\delta}{V_{max}} = \frac{\gamma}{V_{max}} \int_{\epsilon}^{\Delta} u^2 \, du \]
Evaluation of the integral yields
\(\frac{\gamma}{3 V_{max}}(\Delta^3 - \epsilon^3)\). For
\(\epsilon \ll \Delta\), this quantity is strictly positive.

:::\{prf:lemma\} Complementarity of Fractal and Coherent Branches
:label: lem-complementarity-of-fractal-and-coherent-branches

Assume the Concentration--Compactness Principle 8.5.7. If a renormalized
trajectory does not follow the fractal/high-entropy branch in the sense
of Definition 8.4.3, then along any blow-up sequence there exists a
subsequence for which \(Z[\mathbf{V}(s_n)]\) is bounded and
\(\Xi[\mathbf{V}(s_n)]\to\Xi_{\max}\). In particular, modulo symmetries
the profile converges in \(H^1_\rho\) to the extremizer manifold
\(\mathcal{M}\) of Section 8.5.

\textbf{Remark 8.6.5a (Uniformity in renormalized variables).} The
analysis is carried out in the renormalized frame, where
\(\|\mathbf{V}\|_{L^2_\rho}\sim 1\) by construction. The constants
\(\gamma\) and \(V_{max}\) depend on the fixed renormalized energy level
(and ultimately on \(E_0\)) but are independent of the physical scaling
\(\lambda(t)\). In particular, \(\gamma\) does not vanish with
\(\nu \to 0\) along the rescaling, so the transit cost \(\Delta \tau\)
is uniformly bounded away from zero on any fractal-to-coherent
transition.

(sec-the-hysteresis-obstruction)= \#\#\# 8.6.4. The Hysteresis
Obstruction

Finally, we prove that this cost forbids infinite oscillations.

Assume, for the sake of contradiction, that the trajectory performs a
cycle \(Fractal \to Coherent \to Fractal\). 1. \textbf{Inbound Leg
(\(F \to C\)):} The solution traverses the region where
\(\delta(s) \in [\epsilon, \Delta]\). By Theorem 8.6.5, \(\tau(s)\)
increases by at least \(\Delta \tau_{min} > 0\). 2. \textbf{Outbound Leg
(\(C \to F\)):} The solution exits the neighborhood of \(\mathcal{M}\).
During this phase, \(\delta(s) > \epsilon\). By Proposition 8.6.2,
\(\dot{\tau}(s) \ge \gamma \epsilon^2 > 0\). The radius of analyticity
continues to increase. 3. \textbf{Net Effect:} Over a closed cycle in
\(L^2_\rho\), the parameter \(\tau\) strictly increases:
\[ \oint \dot{\tau}(s) \, ds > 0 \] This contradicts the assumption of a
closed cycle in the phase space augmented by the regularity parameter.
Since \(\tau(s)\) is bounded from above for any solution in the global
attractor (due to the finite fractal dimension of the attractor),
\(\tau(s)\) cannot grow indefinitely. Therefore, the oscillations must
dampen, and the trajectory must asymptotically confine itself to the
region where \(\dot{\tau} \to 0\). By Proposition 8.6.2, this implies
\(\delta(s) \to 0\). The solution is forced into the Coherent Stratum,
where the geometric stability results of Sections 4 and 6 apply.

:::\{prf:theorem\} Exclusion of Recurrent Dynamics :label:
the-exclusion-of-recurrent-dynamics

The solution \(\mathbf{V}(s)\) cannot exhibit recurrent behavior (limit
cycles or chaotic attractors) involving the Fractal Stratum.

(sec-modulational-stability-and-the-virial-barrier)= \#\# 9.
Modulational Stability and the Virial Barrier

We develop a rigidity-plus-capacity argument to rule out Type II (fast
focusing) blow-up by showing that any attempt to accelerate beyond the
viscous scale forces decay of the shape perturbation and triggers a
virial (mass-flux) obstruction. For later use we distinguish two dynamic
branches for potential blow-up. :::

:::\{prf:definition\} Type I and Type II Branches :label:
def-type-i-and-type-ii-branches

Let \(\lambda(t)\) be the physical scaling parameter associated with the
renormalized flow and let \(R(t)\) denote a characteristic core radius.
We say that a blow-up sequence follows:

\begin{enumerate}
\def\labelenumi{\arabic{enumi}.}
\item
  the \textbf{Type II Branch} if, in renormalized coordinates, the
  collapse of the core is supercritical in the sense that \[
  \lambda(t) R(t) \to 0 \quad \text{as } t\uparrow T^*,
  \] equivalently, the effective interface thickness shrinks faster than
  the parabolic scale dictated by the global energy bound; and
\item
  the \textbf{Type I Branch} if, along every blow-up sequence, there
  exists a subsequence for which \(\lambda(t)R(t)\) remains comparable
  to the parabolic scale, so that the local collapse is controlled (up
  to slowly varying factors) by the Type I rescaling
  \(\lambda(t)\sim\sqrt{T^*-t}\).
\end{enumerate}

Thus any potential singularity lies either on the Type II branch, where
the core attempts to ``outrun'\,' diffusion, or on the Type I branch,
where diffusion remains coupled to the collapse and the mechanisms of
Sections 4, 6, and 11 are available.

(sec-modulation-neutral-modes-and-spectral-projection)= \#\#\# 9.1.
Modulation, Neutral Modes, and Spectral Projection

In a Type II blow-up scenario the renormalized profile
\(\mathbf{V}(y,s)\) does not converge to a stationary helical profile
but would have to drift along an unstable manifold. Because the
renormalized equation is invariant under scaling and spatial
translations, the linearized operator always has neutral
(zero--eigenvalue) modes corresponding to these symmetries. Any spectral
argument must therefore be formulated on the subspace orthogonal to the
symmetry modes, and the solution must be decomposed so that the
perturbation lies in this subspace for all \(s\).

We adopt the modulation framework of Section 6.1.2 and Lemma 6.7.1. Let
\(\mathbf{Q}\) be a stationary helical profile solving the renormalized
Navier--Stokes equation in the high-swirl regime (Section 5). After
choosing modulation parameters \((\lambda(t),x_c(t),Q(t))\) as in
Definition 6.1, we write in the renormalized variables \[
\mathbf{V}(y,s) = \mathbf{Q}(y) + \mathbf{w}(y,s),
\] where \(\mathbf{w}\) represents the shape perturbation. The scaling
generator is denoted by \(\Lambda \mathbf{Q}\), and we let
\(\Psi_j = \partial_{y_j}\mathbf{Q}\) denote translation modes. The
infinitesimal generators of rigid rotations are denoted by
\(\mathcal{R}_i\mathbf{Q}\) (\(i=1,2,3\)), corresponding to the action
of \(SO(3)\) on the profile: \[
\mathcal{R}_i \mathbf{Q}(y) := \left.\frac{d}{d\theta}\right|_{\theta=0} \mathbf{Q}\big(R_i(\theta)^\top y\big),
\] where \(R_i(\theta)\in SO(3)\) is the rotation by angle \(\theta\)
around the \(i\)-th coordinate axis.

To eliminate the neutral directions we impose orthogonality constraints
for all \(s\ge s_0\): \[
\langle \mathbf{w}(s), \Lambda \mathbf{Q} \rangle_\rho = 0, \qquad
\langle \mathbf{w}(s), \Psi_j \rangle_\rho = 0 \quad (j=1,2,3), \qquad
\langle \mathbf{w}(s), \mathcal{R}_i \mathbf{Q} \rangle_\rho = 0 \quad (i=1,2,3),
\] where \(\langle\cdot,\cdot\rangle_\rho\) denotes the \(L^2_\rho\)
inner product. These conditions determine the modulation parameters and
ensure that \(\mathbf{w}(s)\) lies in the closed subspace \[
X_\perp := \Big\{ \mathbf{w}\in L^2_\rho : \langle \mathbf{w}, \Lambda \mathbf{Q} \rangle_\rho
 = \langle \mathbf{w}, \Psi_j \rangle_\rho = \langle \mathbf{w}, \mathcal{R}_i \mathbf{Q} \rangle_\rho = 0,\ j=1,2,3,\ i=1,2,3 \Big\}
\] for all \(s\). Linearizing the renormalized equation around
\(\mathbf{Q}\) yields an operator \[
\mathcal{L} : H^1_\rho \to L^2_\rho,
\] whose kernel contains the symmetry modes \(\Lambda \mathbf{Q}\) and
\(\Psi_j\).

We now apply the proven spectral results to the \textbf{projected}
operator.

:::\{prf:theorem\} Projected Spectral Gap from High-Swirl Accretivity
:label: the-projected-spectral-gap-from-high-swirl-accretivity

By Theorems 6.3 and 6.4, for profiles in the high-swirl basin of
attraction (\(\sigma > \sigma_c\) or equivalently
\(\mathcal{S} > \sqrt{2}\)), the linearized operator
\(\mathcal{L}_\sigma\) is strictly accretive with spectral gap
\(\mu > 0\). Let \(\mathcal{L}_\perp\) denote the restriction of
\(\mathcal{L}_\sigma\) to \(X_\perp\). Then: \[
\operatorname{Re}\,\langle \mathcal{L}_\sigma\mathbf{w}, \mathbf{w} \rangle_\rho
 \le -\mu \|\mathbf{w}\|_{L^2_\rho}^2
 \quad \text{for all } \mathbf{w}\in X_\perp.
\] This follows directly from the accretivity of \(\mathcal{L}_\sigma\)
established in Theorem 6.3, which holds on the full space and therefore
on any subspace. :::

:::\{prf:theorem\} Modulated rigidity in the high-swirl regime :label:
the-modulated-rigidity-in-the-high-swirl-regime

For profiles satisfying the high-swirl condition of Theorem 6.3, let
\(\mathbf{V} = \mathbf{Q} + \mathbf{w}\) be the modulated decomposition
above with orthogonality conditions \[
\langle \mathbf{w}(s), \Lambda \mathbf{Q} \rangle_\rho
 = \langle \mathbf{w}(s), \Psi_j \rangle_\rho
 = \langle \mathbf{w}(s), \mathcal{R}_i \mathbf{Q} \rangle_\rho = 0
\] for all \(s\). Then there exists a constant \(C>0\) such that \[
\frac{d}{ds} \|\mathbf{w}(\cdot,s)\|^2_{L^2_\rho}
 \le - \lambda_{gap} \|\mathbf{w}(\cdot,s)\|^2_{L^2_\rho}
      + C \|\mathbf{w}(\cdot,s)\|^3_{L^2_\rho},
\] and the scaling rate \(a(s) = -\lambda \dot{\lambda}\) satisfies \[
|a(s)-1| \le C \|\mathbf{w}(\cdot,s)\|_{L^2_\rho}.
\] In particular, if \(\|\mathbf{w}(\cdot,s_0)\|_{L^2_\rho}\) is
sufficiently small, then \(\mathbf{w}\) decays exponentially and
\(a(s)\to 1\) as \(s\to\infty\); the profile is attracted to the
stationary manifold \(\{\mathbf{Q}\}\) and the scaling remains of Type
I. :::

\[
\partial_s \mathbf{w} = \mathcal{L}\mathbf{w} + \mathcal{N}(\mathbf{w}),
\] where \(\mathcal{N}(\mathbf{w})\) is at least quadratic in
\(\mathbf{w}\). Taking the \(L^2_\rho\) inner product with
\(\mathbf{w}\) and using the proven spectral gap from Theorem 6.3 gives
\[
\frac{1}{2}\frac{d}{ds}\|\mathbf{w}\|_{L^2_\rho}^2
 = \operatorname{Re}\,\langle \mathcal{L}\mathbf{w}, \mathbf{w} \rangle_\rho
   + \operatorname{Re}\,\langle \mathcal{N}(\mathbf{w}), \mathbf{w} \rangle_\rho
 \le -\lambda_{gap}\|\mathbf{w}\|_{L^2_\rho}^2 + C\|\mathbf{w}\|_{L^2_\rho}^3.
\] This yields the differential inequality \[
\frac{d}{ds}\|\mathbf{w}(\cdot,s)\|_{L^2_\rho}^2
 \le -2\lambda_{gap}\|\mathbf{w}(\cdot,s)\|_{L^2_\rho}^2 + 2C\|\mathbf{w}(\cdot,s)\|_{L^2_\rho}^3.
\] If we denote \(X(s):=\|\mathbf{w}(\cdot,s)\|_{L^2_\rho}\), this can
be rewritten as \[
\frac{d}{ds}X^2(s) \le -2\lambda_{gap}X^2(s) + 2C X^3(s).
\] In particular, whenever \(X(s)\le \lambda_{gap}/C\) we have \[
\frac{d}{ds}X^2(s) \le -\lambda_{gap} X^2(s),
\] so that \[
X^2(s) \le X^2(s_0)\,e^{-\lambda_{gap}(s-s_0)} \quad \text{for all } s\ge s_0
\] as long as \(X(s)\le \lambda_{gap}/C\) on \([s_0,s]\). Choosing
\(X(s_0)\) sufficiently small (say \(X(s_0)\le \lambda_{gap}/2C\)) and
applying a continuity argument yields the uniform bound
\(X(s)\le \lambda_{gap}/C\) for all \(s\ge s_0\). Thus
\(\|\mathbf{w}(\cdot,s)\|_{L^2_\rho}\) decays exponentially: \[
\|\mathbf{w}(\cdot,s)\|_{L^2_\rho}
 \le \|\mathbf{w}(\cdot,s_0)\|_{L^2_\rho} e^{-\lambda_{gap}(s-s_0)/2}
 \quad\text{for all } s\ge s_0.
\]

To control \(a(s)\), we differentiate the orthogonality condition \[
\frac{d}{ds}\langle \mathbf{w}, \Lambda\mathbf{Q} \rangle_\rho = 0
\] and substitute the equation for \(\partial_s \mathbf{w}\). Using the
fact that \(\Lambda\mathbf{Q}\) is an eigenfunction associated with the
neutral scaling mode and that the modulation parameters have been chosen
so that the scaling degree of freedom is absorbed into \(a(s)\), one
obtains an identity of the form \[
(a(s)-1)\|\Lambda\mathbf{Q}\|_{L^2_\rho}^2
 = -\langle \mathcal{L}\mathbf{w}, \Lambda\mathbf{Q} \rangle_\rho
   + \text{higher order terms},
\] where the higher order terms are quadratic in \(\mathbf{w}\) and its
derivatives. Since \(\mathcal{L}\) is bounded from \(H^1_\rho\) to
\(L^2_\rho\) and \(\Lambda\mathbf{Q}\in H^1_\rho\), we have \[
|\langle \mathcal{L}\mathbf{w}, \Lambda\mathbf{Q} \rangle_\rho|
 \le C\|\mathbf{w}\|_{L^2_\rho}
\] and the higher order terms are bounded by
\(C\|\mathbf{w}\|_{L^2_\rho}^2\). Dividing by
\(\|\Lambda\mathbf{Q}\|_{L^2_\rho}^2\) yields \[
|a(s)-1| \le C\|\mathbf{w}(\cdot,s)\|_{L^2_\rho}.
\] Combining this with the exponential decay of
\(\|\mathbf{w}\|_{L^2_\rho}\) shows that \(a(s)\to 1\) as
\(s\to\infty\). This completes the proof. \(\hfill\square\)

\emph{Consequence.} A Type II trajectory would require a persistent or
growing shape perturbation \(\mathbf{w}\) and a scaling rate \(a(s)\)
diverging from \(1\). Under the proven spectral gap (Theorem 6.3 and
Corollary 6.1), the perturbation is exponentially damped and \(a(s)\)
remains bounded and converges to the self-similar value \(1\). Thus the
only possible blow-up behaviour in the helical class is Type I; the
faster Type II modulation is incompatible with the projected spectral
gap.

(sec-variancedissipation-virial-inequalities)= \#\#\# 9.2.
Variance--Dissipation (Virial) Inequalities

Let \(I(s) = \int |y|^2 |\mathbf{V}|^2 \rho \, dy\) be the weighted
moment of inertia and define the geometric variance \[
\mathbb{V}[\mathbf{V}] := \|\mathbf{V} - \Pi_{cyl} \mathbf{V}\|_{L^2_\rho}^2,
\] where \(\Pi_{cyl}\) is the orthogonal projection onto axisymmetric,
translationally invariant fields in \(L^2_\rho\).

\textbf{Lemma 9.2 (Variance--dissipation control).} Assume the proven
spectral gap (Theorem 6.3 and Corollary 6.1) and the modulation
decomposition of Section 9.1. Then there exist constants
\(\lambda_{gap}>0\) and \(C_{var}, C_{cent}, C_{visc}>0\) such that, for
all \(s\) sufficiently large (so that
\(\|\mathbf{w}(\cdot,s)\|_{L^2_\rho}\) lies in the perturbative regime),
\[
\frac{d}{ds} \|\mathbf{w}(\cdot,s)\|^2_{L^2_\rho}
 \le -\lambda_{gap} \|\mathbf{w}(\cdot,s)\|^2_{L^2_\rho}
     - C_{var} \,\mathbb{V}[\mathbf{V}(\cdot,s)],
\] and \[
\frac{d^2}{ds^2} I(s)
 \ge C_{cent} \int_{\mathbb{R}^3} \frac{|\mathbf{V}(y,s)|^2}{r^2} \rho(y) \, dy
    - C_{visc} \|\nabla \mathbf{V}(\cdot,s)\|^2_{L^2_\rho}.
\]

\emph{Proof.} We first derive the differential inequality for
\(\|\mathbf{w}\|_{L^2_\rho}^2\). The evolution equation for
\(\mathbf{w}\) has the form \[
\partial_s \mathbf{w} = \mathcal{L}\mathbf{w} + \mathcal{N}(\mathbf{w}),
\] where \(\mathcal{L}\) is the linearized RNSE operator around
\(\mathbf{Q}\) in the high-swirl regime and \(\mathcal{N}(\mathbf{w})\)
collects all quadratic and higher order terms. By construction of the
modulation parameters (Section 9.1), \(\mathbf{w}(s)\) lies in the
subspace orthogonal to the neutral symmetry modes (scaling,
translations, rotations) and the cylindrical subspace onto which
\(\Pi_{cyl}\) projects. Thus we can decompose \[
\mathbf{V}(s) = \Pi_{cyl}\mathbf{V}(s) + (\mathbf{V}(s)-\Pi_{cyl}\mathbf{V}(s))
               = \mathbf{V}_{cyl}(s) + \mathbf{V}_{\perp}(s),
\] with \(\mathbf{V}_{\perp}\) lying in the same subspace as
\(\mathbf{w}\). By the definition of \(\mathbb{V}\), \[
\mathbb{V}[\mathbf{V}(s)] = \|\mathbf{V}_\perp(s)\|_{L^2_\rho}^2.
\]

Taking the \(L^2_\rho\)-inner product of the \(\mathbf{w}\) equation
with \(\mathbf{w}\) and using the spectral gap on the orthogonal
complement of the symmetry and cylindrical modes (Theorem 6.3 and
Corollary 6.1) yields \[
\operatorname{Re}\,\langle \mathcal{L}\mathbf{w},\mathbf{w}\rangle_\rho
 \le -\lambda_{gap}\|\mathbf{w}\|_{L^2_\rho}^2 - C_{var}\|\mathbf{V}_\perp\|_{L^2_\rho}^2,
\] for some \(\lambda_{gap},C_{var}>0\). The additional term
proportional to \(\|\mathbf{V}_\perp\|_{L^2_\rho}^2\) reflects the fact
that, in the high-swirl regime, deviations from cylindrical symmetry
incur an extra coercivity penalty due to the structure of the linearized
operator in the helical class (the cylindrical manifold consists
precisely of stationary high-swirl profiles). The nonlinear term
satisfies \[
|\langle \mathcal{N}(\mathbf{w}),\mathbf{w}\rangle_\rho|
 \le C\|\mathbf{w}\|_{L^2_\rho}\|\mathbf{w}\|_{H^1_\rho}^2
 \le C'\|\mathbf{w}\|_{L^2_\rho}^2
\] for \(\|\mathbf{w}\|_{H^1_\rho}\) sufficiently small, and this
contribution can be absorbed into the
\(-\lambda_{gap}\|\mathbf{w}\|_{L^2_\rho}^2\) term by reducing
\(\lambda_{gap}\) if necessary. Combining these estimates we obtain \[
\frac{1}{2}\frac{d}{ds}\|\mathbf{w}\|_{L^2_\rho}^2
 = \operatorname{Re}\,\langle \mathcal{L}\mathbf{w},\mathbf{w}\rangle_\rho
   + \operatorname{Re}\,\langle \mathcal{N}(\mathbf{w}),\mathbf{w}\rangle_\rho
 \le -\lambda_{gap}\|\mathbf{w}\|_{L^2_\rho}^2 - C_{var}\|\mathbf{V}_\perp\|_{L^2_\rho}^2,
\] which yields the first inequality in the statement once we recall
that
\(\mathbb{V}[\mathbf{V}(s)] = \|\mathbf{V}_\perp(s)\|_{L^2_\rho}^2\).

For the second inequality we differentiate \[
I(s) = \int_{\mathbb{R}^3} |y|^2 |\mathbf{V}(y,s)|^2 \rho(y)\,dy
\] twice with respect to \(s\) and use the renormalized equation \[
\partial_s \mathbf{V} = -\nu \Delta\mathbf{V} + \mathcal{L}_{lin}(\mathbf{V}) + \mathcal{N}_{nl}(\mathbf{V}),
\] where \(\mathcal{L}_{lin}\) denotes the linear part and
\(\mathcal{N}_{nl}\) the nonlinear terms. A straightforward computation,
integrating by parts in \(y\) and using the identity
\(\nabla\rho = -\tfrac{1}{2}y\rho\), gives \[
\frac{d}{ds} I(s)
 = 2\int_{\mathbb{R}^3} |y|^2 \mathbf{V}\cdot \partial_s\mathbf{V}\,\rho\,dy
\] and \[
\frac{d^2}{ds^2} I(s)
 = 2\int_{\mathbb{R}^3} |y|^2\Big(|\partial_s\mathbf{V}|^2 + \mathbf{V}\cdot \partial_s^2\mathbf{V}\Big)\rho\,dy.
\] Substituting the equation for \(\partial_s\mathbf{V}\) and organizing
terms as in the derivation of the virial identity in Section 6 (see
Lemma 6.9) yields \[
\frac{d^2}{ds^2} I(s)
 \ge C_{cent} \int_{\mathbb{R}^3} \frac{|\mathbf{V}(y,s)|^2}{r^2}\rho(y)\,dy
   - C_{visc}\|\nabla\mathbf{V}(\cdot,s)\|_{L^2_\rho}^2,
\] where the positive contribution arises from the centrifugal term
associated with swirl and the negative contribution is controlled by the
viscous dissipation. The key step is the Hardy-type inequality in the
high-swirl regime (Section 6.6), which asserts that \[
\int_{\mathbb{R}^3} |\nabla\mathbf{V}|^2 \rho\,dy
 \ge C\int_{\mathbb{R}^3} \frac{|\mathbf{V}|^2}{r^2}\rho\,dy
\] for some \(C>0\) whenever the swirl ratio exceeds the critical value.
This allows us to dominate all potentially negative terms in the second
derivative by a multiple of \(\|\nabla\mathbf{V}\|_{L^2_\rho}^2\),
leaving a strictly positive centrifugal contribution controlled by
\(\int |\mathbf{V}|^2/r^2\rho\). This establishes the second inequality
with suitable constants \(C_{cent}, C_{visc}>0\). \(\hfill\square\)

(sec-virial-barrier-and-mass-flux-capacity)= \#\#\# 9.3. Virial Barrier
and Mass-Flux Capacity

We now turn the incompressibility constraint into a quantitative
obstruction to Type II focusing in physical variables.

\textbf{Theorem 9.3 (Refined Type II Exclusion via Mass-Flux Capacity).}
Let \(u\) be a Leray--Hopf solution and assume the Dynamic Normalization
Gauge (Definition 9.2.1) and the regularity of the limit profile
(Theorem 9.2.1). If a blow-up sequence follows the Type II branch of
Definition 9.0.1, then the associated dissipation integral satisfies \[
\int_0^{T^*} \int_{\mathbb{R}^3} |\nabla u(x,t)|^2\,dx\,dt = \infty,
\] contradicting the global energy inequality. In particular, no Type II
blow-up can occur under the standing hypotheses.

\emph{Proof.} The rescaled velocity and vorticity fields are given by \[
u(x,t) = \lambda(t)^{-1}\mathbf{V}\big(y,s(t)\big),\qquad
y = \frac{x-x_c(t)}{\lambda(t)},
\] with \(s\) the renormalized time and \(\lambda(t)\) chosen by the
Dynamic Normalization Gauge (Definition 9.2.1). The Dirichlet energy
scales according to \[
\int_{\mathbb{R}^3} |\nabla u(x,t)|^2\,dx
 = \lambda(t)^{-1} \int_{\mathbb{R}^3} |\nabla_y \mathbf{V}(y,s)|^2\,dy
 \sim \lambda(t)^{-1},
\] since the gauge enforces
\(\int_{|y|\le 1}|\nabla_y\mathbf{V}|^2\,dy\equiv 1\) and tightness of
the profile (Theorem 6.1) prevents energy from escaping to infinity.
Thus there exists a constant \(c_0>0\) such that \[
\int_{\mathbb{R}^3} |\nabla u(x,t)|^2\,dx \ge c_0 \lambda(t)^{-1}
\] for all \(t\) sufficiently close to \(T^*\). Integrating in time
gives \[
\int_0^{T^*} \int_{\mathbb{R}^3} |\nabla u(x,t)|^2\,dx\,dt
 \ge c_0 \int_0^{T^*} \lambda(t)^{-1}\,dt.
\]

If the blow-up follows the Type II branch, Definition 9.0.1 implies that
\(\lambda(t)R(t)\to 0\) and hence, in particular, that \(\lambda(t)\)
decays at least as fast as \((T^*-t)^\gamma\) with \(\gamma\ge 1\) along
a suitable sequence of times (the extreme Type II regime). In this case
\[
\int_0^{T^*} \lambda(t)^{-1}\,dt
 \gtrsim \int_0^{T^*} (T^*-t)^{-\gamma}\,dt = \infty,
\] so the dissipation integral diverges, contradicting the global Leray
bound. Thus extreme Type II scaling is impossible.

For milder Type II rates with \(\tfrac12<\gamma<1\), the integral
\(\int_0^{T^*}\lambda(t)^{-1}\,dt\) remains finite, so the energy
argument alone does not exclude such behaviour. However, Theorem 9.1
together with Lemma 9.1 (modulation and spectral projection) and Lemma
9.2 (variance--dissipation control) show that in the high-swirl regime
the linearized operator has no unstable eigenvalues on the orthogonal
complement of the symmetry and cylindrical modes, and the projected
perturbation \(\mathbf{w}\) and scaling rate \(a(s)\) satisfy \[
\frac{d}{ds}\|\mathbf{w}(\cdot,s)\|_{L^2_\rho}^2
 \le -\lambda_{gap}\|\mathbf{w}(\cdot,s)\|_{L^2_\rho}^2
\] and \[
|a(s)-1| \le C\|\mathbf{w}(\cdot,s)\|_{L^2_\rho}.
\] As shown in the proof preceding Lemma 9.2, this implies exponential
decay of \(\mathbf{w}\) and convergence \(a(s)\to 1\), so the rescaling
remains locked to the Type I rate. A genuine mild Type II regime would
require an unstable manifold along which \(a(s)\to\infty\) while
\(\mathbf{w}\) stays small, which is incompatible with the strict
accretivity established in Theorems 6.3--6.4 and the Lyapunov
monotonicity of Section 9.4. Consequently neither extreme nor mild Type
II scaling can occur, and the Type II branch is completely excluded.
\(\hfill\square\)

\textbf{Theorem 9.2 (Centrifugal virial barrier).} Assume the swirl
ratio of the profile satisfies \(\mathcal{S} > \sqrt{2}\) and the
spectral coercivity from Theorem 6.3 holds. Then there exist constants
\(C_{cent}, C_{visc}>0\) such that \[
\frac{d^2}{ds^2} I(s)
 \ge C_{cent} \int_{\mathbb{R}^3} \frac{|\mathbf{V}(y,s)|^2}{r^2} \rho(y) \, dy
    - C_{visc} \|\nabla \mathbf{V}(\cdot,s)\|^2_{L^2_\rho}.
\] In particular, once \(\mathbf{w}\) has been damped by the projected
spectral gap so that \(\mathbf{V}\) remains close to the helical ground
state, the second derivative of \(I(s)\) cannot become uniformly
negative along the trajectory.

\emph{Proof.} As noted in the proof of Lemma 9.2, differentiating
\(I(s)\) twice along the renormalized flow and integrating by parts
yields an identity of the form \[
\frac{d^2}{ds^2} I(s)
 = 4\nu \int_{\mathbb{R}^3} |\nabla\mathbf{V}(y,s)|^2\rho(y)\,dy
   + \int_{\mathbb{R}^3} \mathcal{R}[\mathbf{V},Q](y,s)\,\rho(y)\,dy,
\] where \(\mathcal{R}[\mathbf{V},Q]\) collects contributions from the
convective and pressure terms. In the high-swirl regime, the structure
of the renormalized equation and the pressure decomposition of Section 6
show that \[
\mathcal{R}[\mathbf{V},Q](y,s)
 \ge c_1\frac{|\mathbf{V}(y,s)|^2}{r^2} - c_2|\nabla\mathbf{V}(y,s)|^2
\] for some universal constants \(c_1,c_2>0\) depending only on the
swirl threshold and the spectral coercivity constants. Combining these
estimates gives \[
\frac{d^2}{ds^2} I(s)
 \ge (4\nu-c_2)\int_{\mathbb{R}^3} |\nabla\mathbf{V}|^2\rho\,dy
   + c_1\int_{\mathbb{R}^3} \frac{|\mathbf{V}|^2}{r^2}\rho\,dy.
\] Using the Hardy-type inequality \[
\int_{\mathbb{R}^3} |\nabla\mathbf{V}|^2\rho\,dy
 \ge C_H\int_{\mathbb{R}^3} \frac{|\mathbf{V}|^2}{r^2}\rho\,dy
\] from Section 6.6, we may bound the first term on the right-hand side
by \(-C_{visc}\|\nabla\mathbf{V}\|_{L^2_\rho}^2\) for a suitable choice
of \(C_{visc}\) and retain a positive multiple of
\(\int |\mathbf{V}|^2/r^2\rho\) in the second term. This yields the
claimed inequality with \(C_{cent}=c_1/2\) and \(C_{visc}\) sufficiently
large. \(\hfill\square\)

To capture the interplay between mass flux and dissipation in physical
space, we define the characteristic scales based on the rigorous
renormalized profile.

\textbf{Definition 9.2 (Flux and Dissipation Functionals).} Let
\(\mathbf{V}_\infty \in H^1_\rho(\mathbb{R}^3)\) be the stationary
renormalized profile established in Theorem 9.4. We define the
characteristic scales of the singularity in the physical frame at time
\(t\) (where \(R(t) \approx \lambda(t)\)) via the inverse rescaling: 1.
\textbf{Physical Velocity Scale:}
\(U(t) := \lambda(t)^{-1} \|\mathbf{V}_\infty\|_{L^\infty(\mathbb{R}^3)}\).
2. \textbf{Physical Gradient Scale:}
\(G(t) := \lambda(t)^{-2} \|\nabla \mathbf{V}_\infty\|_{L^\infty(\mathbb{R}^3)}\).
3. \textbf{Mass Flux:} \(\Phi_m(t) \sim R(t)^2 U(t)\).

To establish that these norms are finite, we first prove the regularity
of the limit profile.

\textbf{Theorem 9.2.1 (Smoothness of the Limit Profile).} Any stationary
solution \(\mathbf{V}_\infty\) of the Renormalized Navier-Stokes
equation (6.1) with finite weighted Dirichlet energy
(\(\mathbf{V}_\infty \in H^1_\rho\)) is smooth and bounded.
Specifically,
\(\mathbf{V}_\infty \in C^\infty_{loc}(\mathbb{R}^3) \cap L^\infty(\mathbb{R}^3)\).

\emph{Proof.} The stationary RNSE takes the form:
\[-\nu \Delta \mathbf{V} + (\mathbf{V} \cdot \nabla)\mathbf{V} + \mathbf{V} + \frac{1}{2} y \cdot \nabla \mathbf{V} + \nabla Q = 0\]

Since \(\mathbf{V} \in H^1_\rho\), we have \(\mathbf{V} \in L^6_{loc}\)
by Sobolev embedding. The nonlinear term
\((\mathbf{V} \cdot \nabla)\mathbf{V}\) is initially in \(L^1_{loc}\).

By standard elliptic bootstrapping (see Galdi {[}@bianchi1991{]},
Theorem X.1.1), the finiteness of the Dirichlet integral allows
iterative improvement of regularity: 1.
\(\mathbf{V} \in H^1 \implies \mathbf{V} \in L^6\). 2. This implies the
convective term is in \(L^{3/2}\). 3. By elliptic regularity of the
Stokes operator, \(\mathbf{V} \in W^{2, 3/2} \subset L^\infty\) (in 3D,
critical embedding requires careful handling, but \(L^q\) iteration
yields \(L^\infty\)). 4. Once \(\mathbf{V} \in L^\infty\), higher
derivatives follow by standard Schauder estimates.

Consequently, the pointwise quantities
\(\|\mathbf{V}_\infty\|_{L^\infty}\) and
\(\|\nabla \mathbf{V}_\infty\|_{L^\infty}\) are finite constants
depending only on the global energy \(E_0\).

:::\{prf:remark\} Perturbative validity and the global exit strategy
:label: rem-perturbative-validity-and-the-global-exit-strategy

The modulational estimate above is perturbative: it controls
\(\mathbf{w}\) and locks \(a(s)\) when the profile remains close (in
\(L^2_\rho\)) to a non-trivial background \(\mathbf{Q}\). If
\(\mathbf{Q}\) were to decay so that \(\|\mathbf{Q}\|_{L^2_\rho}\to 0\),
the linearization would no longer apply. We handle this via a scale
dichotomy. In the perturbative regime
(\(\|\mathbf{w}\| \ll \|\mathbf{Q}\|\)), Theorem 9.1 gives exponential
decay of \(\mathbf{w}\) and \(a(s)\to 1\), excluding mild Type II drift.
If the trajectory exits this neighbourhood or if \(\mathbf{Q}\)
vanishes, the \textbf{Dynamic Normalization Gauge} (Definition 9.2.1)
still enforces \(\|\nabla \mathbf{V}(\cdot,s)\|_{L^2(B_1)} \equiv 1\),
so the profile cannot disappear. In this non-perturbative regime we no
longer rely on linearization: the solution falls under the global
\textbf{Virial Barrier} (Theorem 9.2) and \textbf{Mass-Flux Capacity}
(Theorem 9.3). Section 10 shows there is no non-trivial stationary
profile satisfying these constraints. Thus the ``vanishing
background'\,' case triggers the global rigidity mechanisms, while the
spectral argument is used only to preclude shape-preserving drift along
a non-trivial unstable manifold.

\textbf{Remark 9.2.1a (The singular stationarity defense).} Regularity
of stationary Navier--Stokes solutions for large data is open, but Type
II exclusion does not rest on assuming smoothness. We separate cases: 1.
\textbf{Smooth branch (\(\mathbf{V}_\infty\in C^\infty\)).} Pointwise
bounds feed into the mass-flux capacity argument (Theorem 9.3), which
excludes Type II blow-up via divergence of the dissipation integral. 2.
\textbf{Singular branch (\(\mathbf{V}_\infty\notin C^\infty\)).} A
singular limit has a rough spectrum and is variationally inefficient
(Remark 8.5.1a): smoothing increases \(\Xi\), so
\(\Xi[\mathbf{V}_\infty]<\Xi_{\max}\). Sustained acceleration requires
maximal efficiency, so an accelerating trajectory converging to a
singular profile falls into \(\Omega_{\mathrm{Frac}}\), not
\(\Omega_{\mathrm{Acc}}\), and is excluded by Gevrey recovery (Section
8.6).

Thus \(\Omega_{\mathrm{Acc}}\) consists of trajectories with smooth
limits handled by Theorem 9.3; trajectories with singular limits are
ruled out by variational inefficiency. Assuming smoothness in Section 9
is therefore without loss of generality for Type II exclusion.

(sec-the-non-vanishing-core-lemma-gauge-normalization)= \#\#\#\# 9.2.1.
The Non-Vanishing Core Lemma (Gauge Normalization)

A crucial prerequisite for the spectral analysis in Section 6 and the
capacity analysis in Section 9 is that the limit profile
\(\mathbf{V}_\infty\) is not the trivial zero solution. If
\(\mathbf{V}_\infty \equiv 0\), the spectral gap \(\mu\) would vanish
and the coercivity and capacity barriers would become ineffective. We
make this non-vanishing precise by isolating the normalization used in
the renormalized frame.

:::\{prf:definition\} Dynamic Normalization Gauge :label:
def-dynamic-normalization-gauge

Consistently with Definition 6.1, we define the scaling parameter
\(\lambda(t)\) by enforcing the normalization of the renormalized
enstrophy on the unit ball: \[
\int_{|y|\le 1} |\nabla_y \mathbf{V}(y,s)|^2 \, dy \equiv 1
 \quad \text{for all } s \ge s_0.
\] This gauge condition fixes \(\lambda(t)\) (and hence the scaling rate
\(a(s)\)) as long as the Tightness property of Theorem 6.1 prevents the
singularity from concentrating on a shell at infinity. :::

By Theorem 6.1 (Strong Compactness of the Blow-up Profile), there exists
a subsequence \(s_n\to\infty\) such that \[
\mathbf{V}(\cdot,s_n) \to \mathbf{V}_\infty
\quad\text{in } C^\infty_{\mathrm{loc}}(\mathbb{R}^3).
\] In particular,
\(\nabla \mathbf{V}(\cdot,s_n) \to \nabla \mathbf{V}_\infty\) in
\(L^2(B_1)\), so the normalization is preserved in the limit: \[
\|\nabla \mathbf{V}_\infty\|_{L^2(B_1)}
 = \lim_{n\to\infty} \|\nabla \mathbf{V}(\cdot,s_n)\|_{L^2(B_1)}
 = 1.
\] Thus the limit profile is strictly non-trivial and provides a genuine
background for the linearized spectral operator
\(\mathcal{L}_{\mathbf{V}_\infty}\). \(\hfill\square\)

\textbf{Theorem 9.3 (Refined Type II Exclusion).} Under the Dynamic
Normalization Gauge (Definition 9.2.1) and the high-swirl spectral
coercivity assumptions of Section 6, no Type II blow-up (in the sense of
\(\lambda(t)\ll\sqrt{T^*-t}\)) can occur. More precisely: 1.
\textbf{Extreme Type II (\(\lambda(t) \sim (T^*-t)^\gamma\) with
\(\gamma \ge 1\))} is excluded by the global energy bound: the
dissipation integral
\(\displaystyle\int_0^{T^*} \|\nabla u(\cdot,t)\|_{L^2}^2 dt\) diverges
for such scaling, contradicting finite initial energy. 2. \textbf{Mild
Type II (\(1/2 < \gamma < 1\))} is excluded by modulational stability:
in the high-swirl regime the renormalized profile is spectrally stable
(Theorems 6.3--6.4 and 9.1), and the modulation equation for the scaling
rate \(a(s) = -\lambda\dot{\lambda}\) forces \(a(s)\to 1\) as
\(s\to\infty\). Sustained acceleration (\(a(s)\to\infty\)) is
incompatible with the projected spectral gap.

\emph{Proof (outline).} We first record a quantitative lower bound on
the dissipation rate. Suppose the singularity is Type II and impose the
Dynamic Normalization Gauge (Definition 9.2.1, consistent with
Definition 6.1), so that the renormalized profile maintains unit
enstrophy \(\|\nabla \mathbf{V}(\cdot, s)\|_{L^2(B_1)} \equiv 1\) for
all \(s \in [s_0, \infty)\).

The physical velocity gradient is:
\[\nabla \mathbf{u}(x,t) = \lambda(t)^{-2} \nabla_y \mathbf{V}(y,s)\]

Consequently, the physical energy dissipation rate is strictly coupled
to the scaling parameter:
\[E_{diss}(t) = \nu \int_{\mathbb{R}^3} |\nabla \mathbf{u}|^2 dx = \nu \lambda(t)^{-1} \int_{\mathbb{R}^3} |\nabla_y \mathbf{V}|^2 \rho(y) dy\]

Since
\(\|\nabla \mathbf{V}\|_{L^2_\rho} \geq \|\nabla \mathbf{V}\|_{L^2(B_1)} = 1\)
by normalization, we have: \[E_{diss}(t) \geq \nu \lambda(t)^{-1}\]

Therefore, the total energy dissipated up to time \(T^*\) is:
\[\int_0^{T^*} E_{diss}(t) dt \geq \nu \int_0^{T^*} \lambda(t)^{-1} dt\]

If \(\lambda(t) \sim (T^*-t)^\gamma\) with \(\gamma \ge 1\) (extreme
Type II), then \[
\int_0^{T^*} \lambda(t)^{-1} dt \sim \int_0^{T^*} (T^* - t)^{-\gamma} dt = \infty,
\] so the total dissipation diverges and contradicts the global Leray
energy inequality. This proves item (1).

For mild Type II scalings with \(1/2 < \gamma < 1\), the above integral
may remain finite, so the energy bound alone is insufficient. In this
regime the high-swirl coercivity hypotheses imply that the renormalized
profile lies in the helical stability class of Theorem 6.3. Writing
\(\mathbf{V} = \mathbf{Q} + \mathbf{w}\) as in Section 9.1, Theorem 9.1
shows that the perturbation \(\mathbf{w}\) decays exponentially and the
scaling rate satisfies \[
|a(s)-1| \le C \|\mathbf{w}(\cdot,s)\|_{L^2_\rho}, \qquad a(s) = -\lambda\dot{\lambda}.
\] Thus \(a(s)\to 1\) as \(s\to\infty\), and the renormalized solution
is attracted to the Type I self-similar scaling. Any persistent
deviation \(a(s)\gg 1\) needed to sustain
\(\lambda(t)\sim (T^*-t)^\gamma\) with \(1/2<\gamma<1\) is incompatible
with the proven spectral gap. This rules out mild Type II as well and
establishes item (2). \(\hfill\square\)

\emph{Energetic capacity viewpoint.} We express the physical quantities
in terms of the bounded norms of the renormalized profile
\(\mathbf{V}_\infty\).

\begin{enumerate}
\def\labelenumi{\arabic{enumi}.}
\item
  \textbf{Kinetic Energy Flux (\(F_{in}\)):} The physical velocity is
  \(\mathbf{u}(x,t) = \lambda(t)^{-1} \mathbf{V}_\infty(y)\). Flux
  across the core surface \(\partial B_{R(t)}\) scales as:
  \[F_{in} \approx \int_{\partial B_R} |\mathbf{u}|^3 \, dS \approx R(t)^2 \left( \lambda^{-1} \|\mathbf{V}_\infty\|_{L^\infty} \right)^3\]
  Since \(R(t) \sim \lambda(t)\), this yields:
  \[F_{in} \sim \lambda^{-1} \|\mathbf{V}_\infty\|_{L^\infty}^3\]
\item
  \textbf{Viscous Dissipation (\(D_{visc}\)):} The physical gradient is
  \(\nabla \mathbf{u}(x,t) = \lambda(t)^{-2} \nabla_y \mathbf{V}_\infty(y)\).
  Total dissipation in the core volume \(B_{R(t)}\) scales as:
  \[D_{visc} \approx \nu \int_{B_R} |\nabla \mathbf{u}|^2 \, dx \approx \nu R(t)^3 \left( \lambda^{-2} \|\nabla \mathbf{V}_\infty\|_{L^\infty} \right)^2\]
  Yielding:
  \[D_{visc} \sim \nu \lambda^{-1} \|\nabla \mathbf{V}_\infty\|_{L^\infty}^2\]
\end{enumerate}

\textbf{The Contradiction:} For a \textbf{fixed} profile shape
\(\mathbf{V}_\infty\), the ratio of total Flux to Dissipation is
constant:
\[\frac{F_{in}}{D_{visc}} \sim \frac{\|\mathbf{V}_\infty\|_{L^\infty}^3}{\nu \|\nabla \mathbf{V}_\infty\|_{L^\infty}^2} = C(\mathbf{V}_\infty)\]

However, Type II blow-up requires the \emph{renormalized}
energy/dissipation to decouple. From \textbf{Theorem 9.2.1},
\(|\nabla \mathbf{V}_\infty|\) is bounded pointwise by
\(C \|\mathbf{V}_\infty\|\). Thus, we cannot pack arbitrary gradients
into the core to ``hide'' dissipation. The dissipation is rigidly linked
to the energy. Since the energy supply (\(F_{in}\)) is bounded by the
global energy constraint (Section 6.1.6), the dissipation cannot grow
arbitrarily large.

Consequently, the infinite acceleration required for Type II
(\(Re_\lambda \to \infty\)) is impossible because the \textbf{smooth}
profile \(\mathbf{V}_\infty\) has a fixed, finite capacity to dissipate
energy, derived from its \(C^\infty\) nature. \(\hfill\square\)

\textbf{Remark 9.3.1 (Rigorous Exclusion of Subscale Spikes).} A primary
objection to capacity arguments is the potential existence of ``subscale
spikes''---concentrations of velocity \(\mathbf{V}\) on scales
\(\delta \ll 1\) inside the renormalized unit ball. This is excluded by
\textbf{Theorem 9.2.1}, which establishes that any Type I limit profile
belongs to \(C^\infty(\mathbb{R}^3)\).

Because \(\mathbf{V}_\infty \in C^\infty(\mathbb{R}^3)\), there exists
no scale \(\delta\) below which the function oscillates or concentrates
arbitrarily. The derivatives are uniformly bounded on compact sets.
Thus, the ``Flux-Averaged'' velocity and the ``Pointwise Maximum''
velocity are comparable up to a constant depending on the profile's
shape. The ``spikes'' are regularized by the elliptic nature of the
limit equation.

Specifically, by \textbf{Theorem 9.2.1}, both
\(\|\mathbf{V}_\infty\|_{L^\infty}\) and
\(\|\nabla \mathbf{V}_\infty\|_{L^\infty}\) are finite constants. This
guarantees: 1. The pointwise maximum velocity is bounded:
\(u_{\max} \leq \lambda^{-1}\|\mathbf{V}_\infty\|_{L^\infty}\) 2. The
gradient bound implies no arbitrarily thin structures exist 3. The
flux-averaged velocity and pointwise maximum are comparable up to
shape-dependent constants

This definitively justifies the use of a single characteristic velocity
scale in Theorem 9.3 and rigorously excludes subscale spikes that would
violate the dissipation estimate.

\textbf{Remark 9.3.2 (The Logarithmic Edge Case).} For marginal scaling
rates where the energy integral might barely converge (e.g., logarithmic
deviations \(\lambda(t) \sim \sqrt{T^*-t} |\log(T^*-t)|^{\alpha}\) with
\(\alpha\) small), we must ensure the normalization still prevents
blow-up.

Consider the dissipative locking mechanism: Since the profile
\(\mathbf{V}\) is forced to have
\(\|\nabla \mathbf{V}\|_{L^2(B_1)} = 1\) by normalization and the
spectral gap \(\mu > 0\) is established (Section 6), the modulation
equation for the scaling rate \(a(s) = -\lambda\dot{\lambda}\) is
strictly controlled by the decay of the error term.

Specifically, from the projected dynamics (Section 9.1):
\[ \frac{da}{ds} = \text{Nonlinear}[\mathbf{w}] + O(e^{-\mu s}) \]

where \(\mathbf{w}\) is the perturbation from the helical ground state.
Since \(\|\mathbf{w}\| \to 0\) exponentially fast, the scaling rate
locks to the self-similar value \(a(s) \to 1\), excluding logarithmic
drift. Even mild Type II scenarios with logarithmic corrections are thus
excluded by the combination of normalization, spectral gap, and
modulation theory.

(sec-lyapunov-monotonicity-and-type-i-reduction)= \#\#\# 9.4. Lyapunov
Monotonicity and Type I Reduction

Combining the projected spectral gap (Theorem 9.1), the
variance--dissipation inequalities (Lemma 9.2), the virial barrier
(Theorem 9.2), and the mass-flux capacity bound (Proposition 9.3) yields
a unified Lyapunov picture:
\[ \frac{d}{ds} \mathcal{E}(s) \le - \mu_1 \|\mathbf{w}\|^2_{L^2_\rho} - \mu_2 \mathbb{V}[\mathbf{V}] - \mu_3 \int \frac{|\mathbf{V}|^2}{r^2} \rho \, dy, \]
for appropriate constants \(\mu_i > 0\) in the helical stability class.
Any trajectory must: 1. Freeze its shape (by spectral rigidity),
eliminating Type II modulation. 2. Obey the virial inequality, ruling
out faster-than-Type-I focusing. 3. Fall back to Type I scaling, which
Section 6 already excludes via spectral coercivity.

Therefore the Type II (fast focusing) route is closed within the
conditional framework: attempting to accelerate triggers either
exponential decay of the shape mode (by the projected spectral gap) or
an energy--capacity mismatch in physical space that starves the
collapse.

(sec-exponential-decay-of-perturbations)= \#\#\# 9.4.1. Exponential
Decay of Perturbations

From Theorem 9.1 and the absorption of nonlinear terms for small data,
Grönwall's inequality gives
\[ \|\mathbf{w}(\cdot, s)\|_{L^2_\rho} \le \|\mathbf{w}(\cdot, 0)\|_{L^2_\rho} e^{-\lambda_{gap} s/2} \]
once \(s\) is large enough that \(\|\mathbf{w}\|\) lies in the
perturbative regime. The variance term \(\mathbb{V}[\mathbf{V}]\) decays
at the same rate by the coupled inequality. Thus any admissible
trajectory is exponentially attracted to the stationary helical manifold
and cannot sustain Type II modulation.

\begin{center}\rule{0.5\linewidth}{0.5pt}\end{center}

(sec-topological-exclusion-of-dynamic-transients)= \#\# 9.5. Topological
Exclusion of Dynamic Transients

The exponential decay of the energy allows us to characterize the
asymptotic fate of the solution using dynamical systems theory. We
explicitly rule out \textbf{Limit Cycles} (pulsating singularities) and
\textbf{Strange Attractors} (chaotic singularities).

(sec-compactness-of-the-orbit)= \#\#\# 9.5.1. Compactness of the Orbit

\textbf{Lemma 9.9 (Strong Compactness).} Let
\(\mathcal{O}^+ = \{ \mathbf{V}(\cdot, s) : s \ge 0 \}\) be the forward
orbit of the solution in \(H^1_\rho(\mathbb{R}^3)\). The Lyapunov
dissipation in Section 9.4 yields a uniform bound
\(\sup_{s \ge 0} \|\mathbf{V}(\cdot,s)\|_{H^1_\rho} < \infty\). By the
weighted Rellich-Kondrachov Theorem, the embedding
\(H^1_\rho \hookrightarrow L^2_\rho\) is compact. Therefore, the orbit
\(\mathcal{O}^+\) is pre-compact in \(L^2_\rho\).

(sec-structure-of-the-omega-limit-set)= \#\#\# 9.5.2. Structure of the
\(\omega\)-Limit Set

We define the \(\omega\)-limit set of the trajectory:
\[ \omega(\mathbf{V}_0) = \bigcap_{s_0 \ge 0} \overline{ \bigcup_{s \ge s_0} \mathbf{V}(s) }^{L^2} \]
By standard dynamical systems theory (LaSalle's Invariance Principle),
the set \(\omega(\mathbf{V}_0)\) is: 1. \textbf{Non-empty} (by
compactness). 2. \textbf{Invariant} under the renormalized flow. 3.
\textbf{Contained in the Zero-Dissipation Set:} For any
\(\mathbf{V}^* \in \omega(\mathbf{V}_0)\), the Lyapunov function must be
constant along the orbit passing through \(\mathbf{V}^*\).
\[ \frac{d}{ds} \mathcal{H}[\mathbf{V}^*(s)] = 0 \] By Theorem 9.1, this
implies \(\mathcal{D}[\mathbf{V}^*] = 0\).

\textbf{Proposition 9.10 (The Static Limit).} The condition
\(\mathcal{D}[\mathbf{V}^*] = 0\) implies:
\[ \|\nabla \mathbf{V}^*\|_{L^2_\rho} = 0 \quad \text{and} \quad \mathbb{V}[\mathbf{V}^*] = 0 \]
Consequently, the profile \(\mathbf{V}^*\) must be a stationary solution
to the Renormalized Navier-Stokes Equation with zero geometric variance
(i.e., it must be an axisymmetric steady state).

(sec-theorem-94-asymptotic-self-similarity)= \#\#\# 9.5.3. Theorem 9.4:
Asymptotic Self-Similarity

\textbf{Theorem 9.4 (Rigidity of the Blow-up).} Let \(\mathbf{u}(x,t)\)
be a solution developing a finite-time singularity. Then the
renormalized profile \(\mathbf{V}(y,s)\) converges strongly in
\(L^2_\rho\) to a unique stationary profile \(\mathbf{V}_\infty\):
\[ \lim_{s \to \infty} \|\mathbf{V}(\cdot, s) - \mathbf{V}_\infty\|_{L^2_\rho} = 0 \]
This result eliminates the dynamic transient configuration. The
singularity cannot modulate its shape or oscillate indefinitely. It is
forced to lock onto a specific geometric configuration
\(\mathbf{V}_\infty\).

\textbf{Remark 9.5 (Exclusion of Non-Normal Amplification and Transient
Growth).} Standard eigenvalue analysis of non-normal operators allows
for transient energy growth \(\|e^{t\mathcal{L}}\| \gg 1\) before
asymptotic decay, even when all eigenvalues have negative real parts.
This phenomenon, known as transient growth or non-normal amplification,
could potentially allow perturbations to escape the linear regime before
the spectral decay takes effect.

However, Theorem 6.4 (Uniform Resolvent and Pseudospectral Bound) and
Corollary 6.1 (Strong Semigroup Contraction) preclude this possibility
entirely. The strict containment of the numerical range
\(\mathcal{W}(\mathcal{L}_\sigma)\) in the stable half-plane ensures
that:
\[ \|e^{t\mathcal{L}_\sigma}\| \leq e^{-\mu t} \quad \text{for all } t \geq 0 \]

This bound guarantees that perturbations decay monotonically from
\(t = 0\), with no initial growth phase. The energy
\(E(t) = \|\mathbf{w}(t)\|_{L^2_\rho}^2\) satisfies \(E(t) \leq E(0)\)
for all \(t > 0\), preventing: - Transient amplification that could
trigger nonlinear instabilities - Bypass transitions that circumvent the
linear stability analysis - Non-modal growth mechanisms that exploit
operator non-normality

The pseudospectral bound
\(\sigma_\epsilon(\mathcal{L}_\sigma) \cap \{z : \operatorname{Re}(z) > 0\} = \emptyset\)
for \(\epsilon < \mu\) provides an additional layer of robustness,
ensuring stability even under small perturbations to the operator
itself. This comprehensive exclusion of all transient growth mechanisms
is a direct consequence of the high-swirl accretivity established in
Theorem 6.3.

\begin{center}\rule{0.5\linewidth}{0.5pt}\end{center}

(sec-conditional-synthesis)= \#\# 9.6. Conditional Synthesis

We now summarize the conditional exclusion mechanism developed in the
previous sections. The argument identifies the hypotheses under which
all admissible singular limits are ruled out.

\textbf{Main Theorem (Conditional Regularity via Single Geometric
Obstruction).} The 3D Navier-Stokes equations exhibit no finite-time
blow-up provided the following single condition holds:

\textbf{Geometric Alignment Hypothesis:} Coherent low-swirl filaments
satisfy the Constantin--Fefferman alignment condition \[
\int_0^{T^*} \|\nabla\xi(\cdot,t)\|_{L^\infty}^2 dt < \infty
\] along any potential blow-up sequence.

Then global regularity holds.

\emph{Note:} Through the variational framework of Section 8, we have
rigorously established that: - Fractal configurations are excluded by
the smoothness of extremizers (no additional hypothesis needed) -
High-swirl configurations are excluded by proven spectral coercivity
(Theorems 6.3-6.4) - Type II blow-up is excluded by mass-flux capacity
bounds (Section 9)

Thus regularity reduces to excluding the single remaining configuration:
the low-swirl coherent filament with unbounded internal twist (the
high-twist ``Barber Pole'\,' regime of Definitions 2.2 and 4.3).

\emph{Outline of argument.}

\begin{enumerate}
\def\labelenumi{\arabic{enumi}.}
\tightlist
\item
  \textbf{Assumption of Singularity:} Assume, for the sake of
  contradiction, that there exists a finite blow-up time
  \(T^* < \infty\) and consider the associated renormalized trajectory.
\item
  \textbf{Asymptotic Locking (Section 9):} Under the proven spectral gap
  (Theorem 6.3 and Corollary 6.1) and the modulation framework of
  Sections 6.7 and 9.1, Theorem 9.4 implies that as \(t \to T^*\) the
  renormalized solution converges (in \(L^2_\rho\)) to a stationary
  profile \(\mathbf{V}_\infty\) solving
  \[ -\Delta_y \mathbf{V}_\infty + (\mathbf{V}_\infty \cdot \nabla_y) \mathbf{V}_\infty + \frac{1}{2} y \cdot \nabla_y \mathbf{V}_\infty + \mathbf{V}_\infty + \nabla_y Q = 0. \]
\item
  \textbf{Geometric Filtering (Sections 3--7):}

  \begin{itemize}
  \tightlist
  \item
    If \(\mathbf{V}_\infty\) has \textbf{Low Swirl}
    (\(\mathcal{S} \le \sqrt{2}\)), the axial pressure--inertia
    inequality of Section 4 and the tube analysis exclude straight-tube
    concentration in the bulk.
  \item
    If \(\mathbf{V}_\infty\) has \textbf{High Swirl}
    (\(\mathcal{S} > \sqrt{2}\)), the spectral coercivity inequality of
    Section 6 and the virial/capacity bounds of Section 9 force decay,
    implying \(\mathbf{V}_\infty \equiv 0\).
  \item
    If \(\mathbf{V}_\infty\) is \textbf{High Entropy} (fractal), the
    geometric depletion inequality of Section 3 together with the
    coherence-scaling hypothesis of Section 8.4 excludes such profiles.
  \end{itemize}
\item
  \textbf{Spectral Instability of Residual Profiles (Section 8):} Even
  if a stationary profile \(\mathbf{V}_\infty\) existed in the above
  classes (for instance a Rankine-type vortex), Theorem 8.1 shows that
  such profiles are spectrally unstable (saddle points). The unstable
  manifold has measure zero in the phase space, so generic finite-energy
  initial data cannot converge to these profiles along the renormalized
  flow.
\item
  \textbf{Liouville-Type Contradiction in the Restricted Class:} The
  only profile compatible with all three constraints and the instability
  analysis is the trivial solution \(\mathbf{V}_\infty \equiv 0\).
  However, the compactness result of Section 6.1.2 implies that if a
  singularity exists, any limit profile must have non-zero \(L^2\) mass:
  \[ \|\mathbf{V}_\infty\|_{L^2_\rho} \ge c > 0. \] Within the class of
  flows satisfying Assumptions (1)--(3) this yields a contradiction.
\end{enumerate}

\textbf{Conclusion.} Under the geometric alignment, spectral
coercivity/gap, and phase-decoherence hypotheses above, no finite-time
singularity can occur. The framework thus provides a conditional
geometric regularity criterion for the 3D Navier--Stokes equations: any
blow-up must violate at least one of these analytic hypotheses.

:::\{prf:lemma\} Prevention of the Null Limit :label:
lem-prevention-of-the-null-limit

Under the Dynamic Normalization Gauge, any weak limit
\(\mathbf{V}_\infty\) of the trajectory \(\mathbf{V}(\cdot,s)\)
satisfies \[
\|\nabla \mathbf{V}_\infty\|_{L^2(B_1)} = 1.
\] Consequently, \(\mathbf{V}_\infty \not\equiv 0\).

(sec-virial-rigidity-and-the-exclusion-of-stationary-pr)= \#\# 10.
Virial Rigidity and the Exclusion of Stationary Profiles

The geometric sieve established in Sections 3--7 stratifies the singular
set into distinct topological classes. A potential objection to this
classification is the existence of \textbf{hybrid profiles} with
intermediate swirl or strain (for example, weak-swirl tubes with
\(0 < \mathcal{S} < \sqrt{2}\) or finite-energy analogues of the Burgers
vortex) for which neither the axial defocusing nor the helical
coercivity arguments appear directly decisive.

This section establishes a \textbf{rigorous non-existence theorem} for
stationary Type I profiles through a novel combination of tensor virial
inequalities, symplectic-dissipative decomposition, and soft rigidity
arguments. We prove that the structural incompatibility between the
Hamiltonian (inertial) and gradient (viscous) vector fields precludes
any stationary solution in the weighted Gaussian space, regardless of
swirl ratio. The analysis reveals a fundamental ``virial leakage''
phenomenon where the Gaussian weight breaks the symplectic symmetry,
forcing the inertial term to perform work that is insufficient to
balance the viscous dissipation.

(sec-definitions-and-functional-setup)= \#\#\# 10.1. Definitions and
Functional Setup

:::\{prf:definition\} Gaussian Framework :label: def-gaussian-framework

We work in the weighted Sobolev space with Gaussian measure. Define: -
\textbf{Gaussian weight:} \(\rho(y) = (4\pi)^{-3/2} e^{-|y|^2/4}\) -
\textbf{Weighted Sobolev space:} \(H^1_\rho(\mathbb{R}^3)\) as the
closure of \(C_c^\infty(\mathbb{R}^3)\) under the norm
\[\|\mathbf{V}\|^2_{H^1_\rho} = \int_{\mathbb{R}^3} \left(|\mathbf{V}|^2 + |\nabla \mathbf{V}|^2\right)\rho(y) \, dy\]

\textbf{Fundamental Fact:} Any Type I blow-up limit
\(\mathbf{V}_\infty\) belongs to \(H^1_\rho(\mathbb{R}^3)\) (Seregin,
2012).

The stationary Renormalized Navier--Stokes Equation (RNSE) for a Type I
candidate profile \(\mathbf{V}\) reads: \[
-\nu \Delta \mathbf{V} + (\mathbf{V} \cdot \nabla)\mathbf{V} + \mathbf{V} + \frac{1}{2}(y \cdot \nabla)\mathbf{V} + \nabla Q = 0,\qquad \nabla\cdot\mathbf{V}=0
\] :::

:::\{prf:definition\} Anisotropic Moment Functionals :label:
def-anisotropic-moment-functionals

To capture directional energy distribution, we define: - \textbf{Axial
Moment:}
\[J_z[\mathbf{V}] := \frac{1}{2} \int_{\mathbb{R}^3} z^2 |\mathbf{V}|^2 \rho(y) \, dy\]
- \textbf{Radial Moment:}
\[J_r[\mathbf{V}] := \frac{1}{2} \int_{\mathbb{R}^3} (x^2 + y^2) |\mathbf{V}|^2 \rho(y) \, dy\]
- \textbf{Total Moment (Gaussian moment of inertia):}
\[J[\mathbf{V}] := J_z[\mathbf{V}] + J_r[\mathbf{V}] = \frac{1}{2} \int_{\mathbb{R}^3} |y|^2 |\mathbf{V}|^2 \rho(y) \, dy\]

These functionals quantify the distribution of kinetic energy along
different directions, crucial for detecting anisotropic concentration
mechanisms. :::

\end{document}
